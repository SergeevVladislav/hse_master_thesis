\documentclass[14pt,a4paper, oneside]{extreport}

%%%%%%%%%% Програмный код %%%%%%%%%%
% \usepackage{minted}
% Включает подсветку команд в программах!
% Нужно, чтобы на компе стоял питон, надо поставить пакет Pygments, в котором он сделан, через pip.

% Для Windows: Жмём win+r, вводим cmd, жмём enter. Открывается консоль.
% Прописываем pip install Pygments
% Заходим в настройки texmaker и там прописываем в PdfLatex или XelaTeX:
% pdflatex -shell-escape -synctex=1 -interaction=nonstopmode %.tex

% Для Linux: Открываем консоль. Убеждаемся, что у вас установлен pip командой pip --version
% Если он не установлен, ставим его: sudo apt-get install python-pip
% Ставим пакет sudo pip install Pygments

% Для Mac: Всё то же самое, что на Linux, но через brew.

% После всего этого вы должны почувствовать себя тру-программистами!
% Документация по пакету хорошая. Сам читал, погуглите!

%%%%%%%%%% Математика %%%%%%%%%%
\usepackage{amsmath,amsfonts,amssymb,amsthm,mathtools}
% Показывать номера только у тех формул, на которые есть \eqref{} в тексте.
%\mathtoolsset{showonlyrefs=true}
%\usepackage{leqno} % Нумерация формул слева
%\usepackage{tipa} %Для формулки из логитов


\usepackage{hyphenat}

%%%%%%%%%% Шрифты %%%%%%%%
\usepackage[english]{babel} % выбор языка для документа
\usepackage[utf8]{inputenc} % задание utf8 кодировки исходного tex файла
\usepackage[X2,T2A]{fontenc}        % кодировка

\usepackage{fontspec}         % пакет для подгрузки шрифтов
\setmainfont{Times New Roman}       % задаёт основной шрифт документа

\usepackage{unicode-math}      % пакет для установки математического шрифта
\setmathfont{Asana-Math.otf}    % шрифт для математики

% Конкретный символ из конкретного шрифта
% \setmathfont[range=\int]{Neo Euler}

\usepackage{listings}
\usepackage{xcolor}
\lstset{
	basicstyle=\ttfamily\small,
	keywordstyle=\color{blue},
	commentstyle=\color{gray},
	stringstyle=\color{green!50!black},
	showstringspaces=false,
	breaklines=true,
	frame=single,
	numbers=left,
	numberstyle=\tiny\color{gray},
	tabsize=4
}



%%%%%%%%%% Работа с картинками %%%%%%%%%
\usepackage{graphicx}                  % Для вставки рисунков
\usepackage{graphics}
\graphicspath{{images/}{pictures/}}    % можно указать папки с картинками
\usepackage{wrapfig}                   % Обтекание рисунков и таблиц текстом


%%%%%%%%%% Работа с таблицами %%%%%%%%%%
\usepackage{tabularx}            % новые типы колонок
\usepackage{tabulary}            % и ещё новые типы колонок
\usepackage{array,delarray}      % Дополнительная работа с таблицами
\usepackage{longtable}           % Длинные таблицы
\usepackage{multirow}            % Слияние строк в таблице
\usepackage{float}               % возможность позиционировать объекты в нужном месте

\usepackage{booktabs}            % таблицы как в книгах
% Заповеди из документации к booktabs:
% 1. Будь проще! Глазам должно быть комфортно
% 2. Не используйте вертикальные линни
% 3. Не используйте двойные линии. Как правило, достаточно трёх горизонтальных линий
% 4. Единицы измерения - в шапку таблицы
% 5. Не сокращайте .1 вместо 0.1
% 6. Повторяющееся значение повторяйте, а не говорите "то же"
% 7. Есть сомнения? Выравнивай по левому краю!

%  вычисляемые колонки по tabularx
\newcolumntype{C}{>{\centering\arraybackslash}X}
\newcolumntype{L}{>{\raggedright\arraybackslash}X}
\newcolumntype{Y}{>{\arraybackslash}X}
\newcolumntype{Z}{>{\centering\arraybackslash}X}


%%%%%%%%%% Графика и рисование %%%%%%%%%%
\usepackage{tikz, pgfplots}      % язык для рисования графики из latex'a

%%%%%%%%%% Гиперссылки %%%%%%%%%%
\usepackage{xcolor}              % разные цвета

\usepackage{booktabs}
\usepackage{threeparttable}

\usepackage{hyperref}
\hypersetup{
	unicode=true,           % позволяет использовать юникодные символы
	colorlinks=true,       	% true - цветные ссылки, false - ссылки в рамках
	urlcolor =blue,         % цвет ссылки на url
	linkcolor=black,        % внутренние ссылки
	citecolor=black,        % на библиографию
	breaklinks              % если ссылка не умещается в одну строку, разбивать ли ее на две части?
}


%%%%%%%%%% Другие приятные пакеты %%%%%%%%%
\usepackage{multicol}       % несколько колонок
\usepackage{verbatim}       % для многострочных комментариев
\usepackage{cmap} % для кодировки шрифтов в pdf

\usepackage{enumitem} % дополнительные плюшки для списков
%  например \begin{enumerate}[resume] позволяет продолжить нумерацию в новом списке
	
\usepackage{todonotes} % для вставки в документ заметок о том, что  осталось сделать
% \todo{Здесь надо коэффициенты исправить}
% \missingfigure{Здесь будет Последний день Помпеи}
% \listoftodos --- печатает все поставленные \todo'шки

\usepackage{tikz}
\usepackage{pgfplots}
\usetikzlibrary{arrows.meta, positioning}



%%%%%%%%%%%%%% ГОСТОВСКИЕ ПРИБАМБАСЫ %%%%%%%%%%%%%%%

%%% размер листа бумаги
\usepackage[paper=a4paper,top=15mm, bottom=15mm,left=35mm,right=10mm,includehead]{geometry}


\usepackage{setspace}
\setstretch{1.33}     % Межстрочный интервал
\setlength{\parindent}{1.5em} % Красная строка.


%\flushbottom       % Эта команда заставляет LaTeX чуть растягивать строки, чтобы получить идеально прямоугольную страницу
\righthyphenmin=2  % Разрешение переноса двух и более символов
\widowpenalty=10000  % Наказание за вдовствующую строку (одна строка абзаца на этой странице, остальное --- на следующей)
%\clubpenalty=10000  % Наказание за сиротствующую строку (омерзительно висящая одинокая строка в начале страницы)
\tolerance=1000     % Ещё какое-то наказание.

\usepackage{zref-totpages}

% Нумерация страниц сверху по центру
\usepackage{fancyhdr}
\pagestyle{fancy}
\fancyhead{ } % clear all fields
\fancyfoot{ } % clear all fields
\fancyhead[C]{\thepage}
% Настройка fancyhdr для размещения номеров страниц внизу
\pagestyle{fancy}
\fancyhf{} % Очищаем все поля
\fancyfoot[C]{\thepage} % Номер страницы внизу по центру
\renewcommand{\headrulewidth}{0pt} % Убираем линию вверху страницы
\renewcommand{\footrulewidth}{0pt} % Убираем линию внизу страницы
% Чтобы не прорисовывалась черта!
\renewcommand{\headrulewidth}{0pt}


% Нумерация страниц с надписью "Глава"
\usepackage{etoolbox}
\patchcmd{\chapter}{\thispagestyle{plain}}{\thispagestyle{fancy}}{}{}


%%% Заголовки
\usepackage[indentfirst]{titlesec}{\raggedleft}
% Заголовки по левому краю
% опция identfirst устанавливает отступ в первом абзаце



% В Linux этот пакет сделан косячно. Исправляет это следующий непонятный кусок кода.
\makeatletter
\patchcmd{\ttlh@hang}{\parindent\z@}{\parindent\z@\leavevmode}{}{}
\patchcmd{\ttlh@hang}{\noindent}{}{}{}
\makeatother


% Редактирования Глав и названий
\titleformat{\chapter}
{\normalfont\large\bfseries}
{\thechapter }{0.5 em}{}

% Редактирование ненумеруемых глав chapter* (Введение и тп)
\titleformat{name=\chapter,numberless}
{\centering\normalfont\bfseries\large}{}{0.25em}{\normalfont}

% Убирает чеканутые отступы вверху страницы
\titlespacing{\chapter}{0pt}{-\baselineskip}{\baselineskip}

% Более низкие уровни
\titleformat{\section}{\bfseries}{\thesection}{0.5 em}{}
\titleformat{\subsection}{\bfseries}{\thesubsection}{0.5 em}{}

\titlespacing*{\section}{0 pt}{\baselineskip}{\baselineskip}
\titlespacing*{\subsection}{0 pt}{\baselineskip}{\baselineskip}


% Содержание. Команды ниже изменяют отступы и рисуют точечки!
\usepackage{titletoc}

\titlecontents{chapter}
[1em] %
{\normalsize}
{\contentslabel{1 em}}
{\hspace{-1 em}}
{\normalsize\titlerule*[10pt]{.}\contentspage}

\titlecontents{section}
[3 em] %
{\normalsize}
{\contentslabel{1.75 em}}
{\hspace{-1.75 em}}
{\normalsize\titlerule*[10pt]{.}\contentspage}

\titlecontents{subsection}
[6 em] %
{\normalsize}
{\contentslabel{3 em}}
{\hspace{-3 em}}
{\normalsize\titlerule*[10pt]{.}\contentspage}


% Правильные подписи под таблицей и рисунком
% Документация к пакету на русском языке!
\usepackage[tableposition=top, singlelinecheck=false]{caption}
\usepackage{subcaption}


\counterwithout*{footnote}{chapter}

\DeclareCaptionStyle{base}%
[justification=centering,indention=0pt]{}
\DeclareCaptionLabelFormat{gostfigure}{Figure #2}
\DeclareCaptionLabelFormat{gosttable}{Table #2}

\DeclareCaptionLabelSeparator{gost}{~---~}
\captionsetup{labelsep=gost}

\DeclareCaptionStyle{fig01}%
[margin=5mm,justification=centering]%
{margin={3em,3em}}
\captionsetup*[figure]{style=fig01,labelsep=gost,labelformat=gostfigure,format=hang}

\DeclareCaptionStyle{tab01}%
[margin=5mm,justification=centering]%
{margin={3em,3em}}
\captionsetup*[table]{style=tab01,labelsep=gost,labelformat=gosttable,format=hang}


% межстрочный отступ в таблице
\renewcommand{\arraystretch}{1.2}



% многостраничные таблицы под РОССИЙСКИЙ СТАНДАРТ
% ВНИМАНИЕ! Обязательно за CAPTION !
\usepackage{fr-longtable}


\usepackage{totcount}

\newtotcounter{citnum} %From the package documentation
\def\oldbibitem{} \let\oldbibitem=\bibitem
\def\bibitem{\stepcounter{citnum}\oldbibitem}



%Более гибкие спсики
\usepackage{enumitem}


%%% ГОСТОВСКИЕ СПИСКИ

% Первый тип списков. Большая буква.
\newlist{Enumerate}{enumerate}{1}

\setlist[Enumerate,1]{labelsep=0.5em,leftmargin=1.25em,labelwidth=1.25em,
	parsep=0em,itemsep=0em,topsep=0ex, before={\parskip=-1em},label=\arabic{Enumeratei}.}


% Второй тип списков. Маленькая буква.
\setlist[enumerate]{label=\arabic{enumi}),parsep=0em,itemsep=0em,topsep=0.75ex, before={\parskip=-1em}}


% Третий тип списков. Два уровня.
\newlist{twoenumerate}{enumerate}{2}
\setlist[twoenumerate,1]{itemsep=0mm,parsep=0em,topsep=0.75ex,, before={\parskip=-1em},label=\asbuk{twoenumeratei})}
\setlist[twoenumerate,2]{leftmargin=1.3em,itemsep=0mm,parsep=0em,topsep=0ex, before={\parskip=-1em},label=\arabic{twoenumerateii})}


% Четвёртый тип списков. Список с тире.
\setlist[itemize]{label=--,parsep=0em,itemsep=0em,topsep=0ex, before={\parskip=-1em},after={\parskip=-1em}}


%%% WARNING WARNING WARNIN!
%%% Если в списке предложения, то должна по госту стоять точка после цифры => команда Enumerate! Если идет перечень маленьких фактов, не обособляемых предложений то после цифры идет скобка ")" => команда enumerate! Если перечень при этом ещё и двууровневый, то twoenumerate.


\usetikzlibrary{intersections,calc}

%%%%%%%%%% Список литературы %%%%%%%%%%

%\usepackage[%
%backend=biber, %подключение пакета biber (тоже нужен)
%bibstyle=gost-numeric, %подключение одного из четырех главных стилей biblatex-gost
%sorting=ntvy, %тип сортировки в библиографии
%]{biblatex}
\usepackage[backend=biber,style=gost-numeric, maxbibnames=9,maxcitenames=2,uniquelist=false, babel=other]{biblatex}

% Справка по 4 главным стилям для ленивых:
% gost-inline  ссылки внутри теста в круглых скобках
% gost-footnote подстрочные ссылки
% gost-numeric затекстовые ссылки
% gost-authoryear тоже затекстовые ссылки, но немного другие

% Подробнее смотри страницу 4 документации. Она на русском
\DefineBibliographyStrings{english}{%
	pages = {P\adddot},
	number = {№},
}

\DeclareSourcemap{
	\maps[datatype=bibtex]{
		\map{
			\step[fieldsource=langid, match=english, final]
			\step[fieldset=presort, fieldvalue={a}]
		}
		\map{
			\step[fieldsource=langid, notmatch=english, final]
			\step[fieldset=presort, fieldvalue={z}]
		}
	}
}


% Ещё немного настроек
\DeclareFieldFormat{postnote}{#1} %убирает с. и p.
\renewcommand*{\mkgostheading}[1]{#1} % только лишь убираем курсив с авторов
% Переопределение названия оглавления
\renewcommand{\contentsname}{Contents}


\begin{document} % Начала документа
	
	\thispagestyle{empty} % Чтобы избежать нумерации титульника
	
	\begingroup
	
	\begin{center}
		
		% Первая строка: FEDERAL STATE AUTONOMOUS EDUCATIONAL
		\fontsize{13.5}{15}\selectfont
		\textbf{FEDERAL STATE AUTONOMOUS EDUCATIONAL}
		%\vspace{-\baselineskip}
		
		% Остальной текст: INSTITUTION FOR HIGHER EDUCATION и далее
		\fontsize{13}{15}\selectfont
		\setstretch{1.5} % Устанавливаем межстрочный интервал 1.5
		INSTITUTION FOR HIGHER EDUCATION \\
		NATIONAL RESEARCH UNIVERSITY \\
		HIGHER SCHOOL OF ECONOMICS \\
		Faculty of Social Sciences
		
		\includegraphics[width=0.1\textwidth]{hse_logo.png}
		
		\vspace{1em}
		
		\fontsize{13}{15}\selectfont
		\textbf{Sergeev Vladislav Alexandrovich}
		
		\vspace{1em}
		
		\fontsize{13}{15}\selectfont
		\textbf{Master Thesis}
		
		\vspace{1em}
		
		\fontsize{14}{16}\selectfont
		\textbf{The role of demographic factors in the development of the National Payment System in Russia}
		
		\vspace{1em}
		
		field of study 38.04.04 Public Administration\\
		Master’s program ‘Population and Development’

		\vspace{4em}

	\end{center}
	
	
	\begin{minipage}[t]{0.45\textwidth}
		\raggedright
		Reviewer \\
		Candidate of Sciences (PhD) \\
		\vspace{1em}
		Merekina Elena Vladimirovna
	\end{minipage}
	\hfill
	\begin{minipage}[t]{0.45\textwidth}
		\raggedleft
		Scientific Supervisor \\
		Candidate of Sciences (PhD) \\
		\vspace{1em}
		Larionov Alexander Vitalievich
	\end{minipage}
	
	
	\vfill
	
	\begin{center}
		\normalsize Moscow, 2025
	\end{center}
	
	\endgroup
	
	%%%%%%%%%%%%%%%%%%% Introduction %%%%%%%%%%%%%%%%%%%%%%%%%%%%%%%%%%%%%%
	
	\tableofcontents  % Команда, которая создаёт оглавление
	
	\chapter*{Introduction}
	\addcontentsline{toc}{chapter}{Introduction}
	
	National payment systems are a key element of the modern economy, ensuring the uninterruptible conduct of financial transactions and contributing to the development of trade, investment and economic growth. In the context of digital transformation, they are becoming not only a calculation tool, but also a factor in increasing financial accessibility and inclusivity. The regulation of payment systems carried out by central banks is aimed at ensuring their stability, security and efficiency, which is especially important in the context of growing cyber threats, changes in consumer behavior and global economic trends. National payment systems, as the infrastructural framework of the economy, reflect the level of technological development of the country, the degree of integration of financial services into the daily life of the population and the ability to adapt to the challenges of the times.
	
	
	In Russia, the development of the national payment system has gained strategic importance in the context of achieving technological sovereignty and ensuring stability of the financial sector. Over the past decade, this system has made significant progress. It started with the creation of the MIR payment card and culminated in the introduction of the Faster Payments System. This system has become a key driver for the transition to a cashless economy in Russia. In 2024, cashless transactions accounted for more than 85.8\% \footnote{https://www.cbr.ru/PSystem/} of retail transactions. However, it is important to note that the Russian market continues to exhibit heterogeneity in terms of digital adoption, with varying dynamics across regions, age groups and income levels. This highlights the need for a tailored approach to further development of the national payment system. The regulatory policy of the Central Bank of Russia, including piloting the introduction of the digital ruble and supporting financial technology innovation, aims to reduce these imbalances. Overcoming these imbalances, however, requires taking into consideration fundamental socio-demographic trends.
	
	
	Russia faces several significant demographic challenges, which pose both long-term difficulties and opportunities for its financial sector. One of the most significant is the aging population, which is set to continue. By 2030, according to OECD\footnote{https://mintrud.gov.ru/ministry/programms/12} estimates, the proportion of people aged 60 years and over will reach 25\%, and by 2050 it will be 30\%. This group has historically been less inclined to use digital payment methods. Only 42\% of people 55 or older are actively using online banking, according to NAFI's report from 2024\footnote{https://nafi.ru/analytics/dolya-polzovateley-mobilnogo-banka-rastet-no-rossiyane-stanovyatsya-menee-bditelnymi/}. Another significant challenge is the decline in the birth rate, which has led to a decrease in the number of young people. These age groups are crucial for credit products and financial services. Additionally, there are disparities between regions due to migration from smaller towns and rural areas. According to the Central Bank of Russia, only 48\% of residents in these areas have access to internet banking services in 2024\footnote{Банк России: Основные направления повышения доступности финансовых услуг в Российской Федерации на период 2025-2027 годов}. Additionally, migration – both internally (movement to larger cities) and externally (labour migration) – is changing the composition of the workforce, creating a demand for cross-border transactions and communications in multiple languages.
	
	
	These trends create conflicting pressures on national payment systems. On the one hand, urbanization and increased digital literacy among young people are driving innovation. This is evidenced by an increase in the share of payments made through QR codes among people aged 18-35 from 12\% in 2021 to 27\% by 2033\footnote{https://www.cbr.ru/press/event/?id=23262}. However, on the other hand, ageing populations and regional inequalities are hindering the consolidation of payment infrastructure. This can be seen in regions such as the Far East and North Caucasus, where cash payments continue to account for 45\% of transactions (versus 20\% in Moscow\footnote{https://cbr.ru/press/regevent/?id=28556}) due to a larger proportion of older people and a slower adoption of digital payment methods.

	
	
	The aim of the research is to evaluate the influence of demographic variables on the development of National Payment System of Russia.
	
	
	The following tasks were set in the course of the study:
	\begin{enumerate}
		\item systematize theoretical approaches to the analysis of the relationship between demographic changes and the development of payment systems;
		\item identify key demographic trends in Russia (aging, urbanization, regional differentiation) and their relationship to the dynamics of payment preferences;
		\item conduct a quantitative analysis of demographic factors influence on national payment system metrics (cashless payment share in total retail turnover) through econometric modeling;
		\item formulate practical recommendations for regulatory bodies to mitigate imbalances stemming from demographic changes.
	\end{enumerate}
	
	
	The object of the study is the national payment system of Russia.
	
	
	The subject of the study is the influence of demographic factors on the development of the National Payment System of Russia.
	
	The following research questions have been formulated:
	
	\begin{enumerate}
		\item How do population aging and regional variations impact the pace of digitalization of payment systems and the degree of financial inclusion?
		\item What regulatory initiatives can assist in enhancing the adaptability of payment services to demographic shifts, considering the diverse payment patterns of various age groups?
	\end{enumerate}
	
	
	
	The following hypotheses were put forward:
	\begin{itemize}
		\item Hypothesis of the impact of population aging on payment preferences (\textbf{H1}): the increasing share of the older generation in the age structure negatively affects the share of cashless payments in retail turnover;
		\item Hypothesis of regional differences in the development of the payment system (\textbf{H2}): an increase in the birth rate in urban areas has a positive effect on the share of cashless payments, while in rural areas it has a negative effect.
	\end{itemize}
	
	
	The first section of this study explores the operational mechanisms of Russia's national payment system and the theoretical frameworks that explain the impact of demographic changes on the payment systems.
	
	The second section examines empirical studies to understand the intricate relationship between demographic variables and the development of national payment infrastructures. In the next part, an econometric model is being developed that analyzes the relationship between demographic trends and the development of national payment systems, which will be characterized by a new generalized index developed during the study.
	
	
	Drawing on the findings from the empirical analysis, this final section interprets the results of the model and offers methodological recommendations to enhance the resilience and efficiency of national payment systems in light of demographic changes. A comprehensive strategy will be developed for the improvement of the national payment infrastructure, tailored to meet the specific socio-demographic needs.
	
	
	The structure of the final thesis includes an introduction, 3 chapters, and a conclusion. The total number of pages is \textrm{\ztotpages} pages. The total number of literature sources is \total{citnum}.
	
	
	\chapter{Theoretical foundations and operational principles of the National Payment System of Russia in the context of demographic changes}
	
	\section{The principles of functioning of the National Payment System of Russia}
	
	The National Payment System (NPS) of Russia is a complex network of institutions, technologies and processes that facilitate cashless transactions and ensure smooth processing of payments within the country. It consists of several essential components that work together to maintain stability and efficiency of financial operations. The NPS plays a crucial role in economic growth by enhancing access to financial services for individuals and businesses, which not only improves financial well-being but also contributes to higher standards of living through better financial management and transaction capabilities. Furthermore, the NPS helps safeguard the value of the ruble and promotes integration into global financial markets, strengthening Russia's position in the global economy.
	
	
	The legislative framework of the NPS is provided by Federal Law No. 161-FZ, which defines the legal, economic and organizational foundations for the creation and management of payment systems. This law regulates the activities of payment service providers, sets requirements for their qualifications and rules of interaction.
	
	
	According to 161-FZ, the National Payment System (NPS) includes the following key components:
	
	\begin{enumerate}
		\item \textit{Payment systems} are a set of rules and procedures that govern the process of making payments between parties. They can be interbank (such as the Bank of Russia's system) or retail (like money transfer services). Well-known payment systems in Russia include MIR, Faster Payment System (FPS), and international systems like Visa and MasterCard;
		\item \textit{Payment system participants} are organizations involved in the settlement process, including banks, non-bank financial institutions, and others;
		\item \textit{Infrastructure} refers to the technical tools and technologies that support the NPS, including software for transaction processing and security systems for data protection;
		\item \textit{Regulation} - includes legislative and regulatory acts that define the rules for the operation of NPS. The most significant of these is 161-FZ, which establishes requirements for participants and operators of payment systems, and also regulates issues related to security and consumer protection.
	\end{enumerate}


	The Bank of Russia plays a dual role: as a participant in the system through its cash settlement centers and as a mega-regulator that controls all elements of the NPS. In addition, the Bank of Russia has its own payment system designed for interbank settlements, government payments and operations related to the implementation of monetary policy. This ensures the reliability and security of settlements between all financial market participants.

	
	In accordance with Federal Law No. 161-FZ, the Bank of Russia oversees the activities of payment systems, ensuring their compliance with established requirements and standards. <<A payment system is a set of organizations that interact according to the rules of the payment system in order to transfer funds, including a payment system operator, payment infrastructure service operators, and payment system participants, of which at least three organizations are money transfer operators>>\footnote{Articlle 3 of Federal Law No. 161-FZ}. According to Article 20 of  Federal Law No. 161-FZ payment systems work according to certain rules:
	
	\begin{itemize}
		\item Order of interaction between the operator, participants and infrastructure organizations, including Information exchange algorithm, dispute resolution procedures, regulations for emergency situations;
		\item Monitoring compliance with the rules with mechanisms of regular audit, sanctions for violations (fines, suspension of participation), monitoring of operational activities;
		\item Criteria for access to the system: financial standards for participants, requirements for technological compatibility, prohibition of discriminatory condition;
		\item Payment system operator is obliged to provide the rules free of charge for review by potential participants, coordinate changes to the rules with the Bank of Russia, and ensure the uninterrupted operation of the infrastructure.
	\end{itemize}
	
	
	Article 21 of Federal Law No. 161-FZ defines the legal basis for organizations participation in Russian payment systems, creating a multi-level system of interaction. The main principle is to divide participants into direct and indirect ones, which allows achieving a balance between the availability of payment services and control over systemic risks. According to paragraph 1 of Article 21, direct participants may be:
	\begin{itemize}
		\item Money transfer operators (including electronic ones);
		\item Organizers of trading platforms (according to 325-FZ);
		\item Insurance companies (for CTP settlements);
		\item Federal Treasury and Russian Post;
		\item For international organizations and foreign banks, participation is allowed only in cross-border operations provided integration with the SPFS.
	\end{itemize}
	
	
	The interaction between the participants of the payment system can be clearly shown by the example of the process of transferring a payment through the payment system (Figure \ref{fig:operation}). The payment system transaction process involves 10 main steps\footnote{Масино, М. Н., and А. В. Ларионов. "Пул ликвидности как способ управления риском ликвидности в платежных системах." Финансы и кредит 24, no. 1 (769) (2018): 209-226.}. A client applies to a bank with an application, which forms a corresponding order. The bank then creates a payment order and sends it to the transaction center. The center receives the payment order and forwards it to the clearing center. The latter calculates the position of each participant and returns the information to the center. The operation center then transmits information about each participant's position to the settlement center. This center processes the order, sending information back to the operation center. The operation center receives information about order execution and transmits it to both the sender's and recipient's banks. Banks then notify customers about funds being debited or credited. Funds are debited and credited simultaneously.
	
	
	In some cases, the payment system operator functions as a one-stop shop for payment infrastructure services. In other instances, it divides responsibilities among various providers. Regardless of the arrangement, all payment infrastructure providers play a vital role in facilitating transactions within the system by transferring funds and communicating with participants. The mechanisms for these transfers and communications are determined by the payment operator, who oversees the entire system.
	
	\begin{figure}[H]
		\centering
		\includegraphics[width=0.7\textwidth]{operation.png}
		\caption{The process of transferring a payment through a payment system\\Source: Compilated by author based on legislative framework}
		\label{fig:operation}
	\end{figure}
	
	
	The principle of accessibility and universality is a aspect of the NPS of Russia, which ensures broad and equal access to payment services for citizens and organizations throughout the country. Within this framework, several elements are in place to ensure the availability of these services. One such element is the Faster Payment System (FPS), which was developed by the Bank of Russia and allows citizens and businesses to make transfers between accounts at different banks 24/7. Since its launch in 2019, the SBP has significantly expanded the availability of payments services, allowing users to make instant transfers using their mobile numbers or QR codes.
	
	
	Ensuring the smooth functioning of the NPS is another of the Bank of Russia's priority areas of activity, which is reflected in current regulatory documents and strategic plans of the regulator. Given the high dependence of the economy on digital payments and external challenges, the stability of the NPS is of critical importance. In accordance with 161-FZ, as well as Bank of Russia Regulation No. 738-P dated October 27, 2020, regulates the procedure for ensuring the smooth functioning of the Bank of Russia's payment system, including the FPS. In 2024-2027, the Bank of Russia is implementing a set of measures aimed at improving the availability and reliability of payment services, which is reflected in the <<Main Directions of Development of the NPS for 2025-2027>>.
	
	
	The Bank of Russia actively interacts with participants in the payment market, organizing consultations, discussions and public hearings on the regulation and development of the payment system, which allows taking into account the opinions of various parties and making decisions aimed at improving the functioning of the NPS. Regular assessment of the compliance of payment systems with international standards is carried out, which confirms the desire to maintain a high level of transparency and quality of services. The principle of transparency in the Russian national payment system is implemented through openness of information, accessibility of data for all market participants and active interaction with stakeholders, which helps to increase the trust and efficiency of the payment infrastructure. To ensure the stability of the payment infrastructure, the Bank of Russia is actively developing and implementing technological solutions to improve the reliability and resilience of the system. These solutions include creating backup data centers, implementing modern information security measures, and regularly stress-testing payment systems. The principle of reliability and stability lies at the heart of the efficient operation of the NPS of Russia. To achieve this, a comprehensive approach is required, including the creation and enforcement of regulations, the establishment of standards, and the continuous improvement of technological infrastructure. Through these efforts, the stability and security of payment services are ensured, which in turn enhances user confidence and contributes to the development of the countrys financial system.

	
	The Bank of Russia applies a risk-based approach when setting requirements for payment systems that are of particular importance for financial stability or consumer confidence in cahsless payments. Payment systems are recognized as systemically It is important that banks comply with the requirements of the NPS and international standards, and they are subject to regular assessments of their compliance. The results of these assessments show a high level of compliance, which is published on the website of the Bank of Russia. 
	
	
	
	
	
	
	The principle of efficiency and innovation in the NPS of Russia is aimed at ensuring fast, convenient and economically feasible provision of payment services, the introduction of modern technologies that contribute to the development of the financial market.
	
	
	During the period from 2021 to 2023, the number and volume of payments to NPS increased 1.5–2 times, the share of cashless payments in retail turnover increased from 70.3\% at the beginning of 2021 to 83.4\% at the ending of 2024. The indicators indicate an increase in the efficiency of the payment system, improved accessibility and quality of payment services for citizens and businesses.
	
	
	The introduction of innovative technologies is an aspect of NPS development. FPS, developed by the Bank of Russia, allows transfers between accounts in different banks around the clock and seven days a week, which significantly increases the convenience and speed of transactions. Technologies such as mobile payments, biometric identification, and the use of artificial intelligence are actively developing to improve the security and efficiency of payment transactions.
	
	
	The <<Main directions for the development of the national payment system for 2025-2027>>, approved by the Bank of Russia, provides for continued work on improving the payment infrastructure, developing regulation, stimulating product competition in the payment services market and introducing innovative solutions. The document identifies key areas that require special attention:
	\begin{enumerate}
		\item Scaling of the digital ruble;
		\item Biometric acquiring and a universal QR code;
		\item Support and development of the MIR system;
		\item Cross-border settlements;
		\item Open banking.
	\end{enumerate}

	The strategy aims to improve access to payment services for all residents, especially in remote and rural areas. As part of this strategy, it is planned to stimulate the spread of technologies such as SoftPOS and QR acquiring, which opens up new horizons for the self-employed and microenterprises.
	In addition, the strategy pays special attention to consumer protection, reducing fees, combating fraud in the field of payments and ensuring tariff transparency. It is also planned to strengthen the competitive environment, prevent monopolization and promote the development of alternative options for international payment platforms.
	
	In the context of the development of the national payment system, the results of a recent sociological study\footnote{Банк России: ОТНОШЕНИЕ НАСЕЛЕНИЯ РОССИЙСКОЙ ФЕДЕРАЦИИ К РАЗЛИЧНЫМ СРЕДСТВАМ ПЛАТЕЖА. Результаты социологического исследования за 2023 год} conducted by the Bank of Russia are also worth noting. The study found that the choice between cash and non-cash payments is largely influenced by demographic factors such as age, place of residence, income level, and digital literacy. These factors contribute to a diverse range of payment preferences among citizens. As a result, the national payment system faces the challenge of adapting to meet the varying needs of its users.
	
	
	While non-cash payments have become increasingly popular, cash continues to play an important role for 24\% of Russian citizens, particularly among the elderly and those living in smaller towns where access to banking services may be more limited. For example, in rural areas, 49\% of purchases at small stores are made with cash, and the average transaction value has increased to 1588 rubles. This trend reflects not only established habits of older generations, but also structural issues: 56\% of respondents face issues with non-functioning ATM machines, and 44\% complain about a lack of banknotes in the required denominations. Cash remains a symbol of reliability for many people, especially in remote areas. 36\% of citizens hold their savings in physical rubles out of fear of cyber risks or technical issues.


	However, young people and city residents actively use cashless payments in their daily lives. Of those with bank cards, 75\% prefer using plastic cards, and 48\% use mobile banking for transfers. Non-cash payments are more common in large cities when paying for online purchases (91\%), and utilities (80\%), due to the availability of digital services and confidence in their security.
	Technologically advanced groups, such as IT specialists and students, are more likely to use virtual cards and fast payment systems (42\%). However, even among these groups, there are those who have experienced problems. For example, 25\% of respondents have reported unauthorized debiting of funds and 15\% have fallen victim to cyberbullying. These incidents highlight the need for stronger protection of financial transactions.
	
	
	Differences in financial literacy and access to education programs play a role in these trends. In cities like Moscow and St. Petersburg, where cashless payments are more common, the use of virtual cards is higher. Meanwhile, in rural areas like Siberia and the Far East, where banking infrastructure is less developed and internet connection quality is poor, cash remains a more popular payment method.
	
	
	In order for the NPS to develop harmoniously, it is essential to take into account not only technological advancements, but also the demographic composition of the population. This means that for the effective development of the NPS, it is crucial to consider the diverse demographics of Russia.
	An integrated approach that combines technological solutions, educational initiatives, and investments in infrastructure can help create an inclusive financial environment for all. This approach will benefit both young people who are seeking digital innovation, as well as the elderly who value the convenience of cash payments.

	
	\section{Theoretical aspects of the relationship between demography and national payment systems}
	
	
	National payment systems, which form the basis of the financial infrastructure, are closely linked to demographic trends that affect demand for specific financial products. Demographic shifts, including changes in population demographics such as age, migration patterns and family structures, have a direct impact on consumer behaviour and preferences. By understanding these correlations, payment systems can be customized to address the diverse needs of users and anticipate future changes in demand for financial services.
	
	
	Theoretical models provide a valuable framework for analyzing these connections. This knowledge is crucial for payment system regulators, enabling them to create efficient and inclusive financial solutions that meet societal needs.
	In general, there are several approaches to examining the relationship between the development of the national payment system and demographic factors. These approaches can be categorized into the following broad groups: lifecycle hypothesis, microeconomic and financial theories, technology adoption theories, and transaction cost theory.
	
	
	
	\centerline{\textit{\underline{Life cycle hypothesis}}}


	The national payment systems, which constitute the foundation of the financial infrastructure, are closely intertwined with demographic trends that influence the demand for specific financial services. The life cycle hypothesis\footnote{Modigliani, F., \& Brumberg, R. (1954). Utility analysis and the consumption function: An interpretation of cross-section data. Franco Modigliani, 1(1), 388-436.}, a classical economic theory proposed by the economist Franco Modigliani in the mid-twentieth century, illuminates this intricate relationship. Originally conceived to explore consumer behavior, this hypothesis has proved valuable in comprehending the development of payment systems, especially in the context of an aging population and regional disparities.


	The life cycle hypothesis posits that individuals strive to optimize their financial choices throughout their lives, striking a balance between consumption, savings, and debt\footnote{Modigliani, F. (1966). The life cycle hypothesis of saving, the demand for wealth and the supply of capital. Social research, 160-217.}. Young individuals, in the process of accumulating human capital, often turn to borrowing to finance their education, home purchases, or entrepreneurial ventures. Conversely, the mature generation, having reached the pinnacle of their earnings, tend to prioritize saving, while older citizens gradually deplete their accumulated assets.


	According to theory\footnote{Ando, A., \& Modigliani, F. (1963). The “Life Cycle” Hypothesis of Saving: Aggregate Implications and Tests. The American Economic Review, 53(1), 55–84. http://www.jstor.org/stable/1817129}, there are several stages in people's lives:

	\begin{itemize}
		\item youth -- when people invest in education;
		\item working age -- when people are actively working and saving money;
		\item retirement age -- when people spend their savings.
	\end{itemize}

	Each stage is associated with specific preferences when it comes to financial instruments. Young people often opt for digital payments and mobile applications, whereas older individuals may favour traditional payment methods such as cash or bank transfers. In different periods of life, people can change their financial management strategies, which affects the amount of funds placed in digital banks.


	The life cycle hypothesis also helps to explain regional imbalances in urbanization\footnote{Modigliani, F. (1970). The life cycle hypothesis of saving and intercountry differences in the saving ratio (pp. 197-225). WA Eltis, M. FG. Scott and JN Wolfe, eds., Induction, trade, and growth: Essays in honour of Sir Roy Harrod (Clarendon Press, Oxford).}. Young people migration to large cities results in the concentration of innovative users in megacities, leaving rural areas and smaller towns as the <<demographic reservoir>> for the older generation. This leads to a negative cycle: low demand for digital services prevents their implementation, while the lack of necessary infrastructure contributes to regional inequality.


	However, like any theory, the Modigliani theory has its limitations, which must be taken into account.
	First, it focuses on long-term trends, but it cannot explain the drastic changes caused by external factors\footnote{Taleb, N.N. (2007) The Black Swan: The Impact of the Highly Improbable. Random House, New York}. For example, the COVID-19 pandemic has accelerated the transition to cashless payments even among the elderly, which contradicts the initial assumptions of life cycle hypothesis. Secondly, the theory does not always take into account cultural and institutional features\footnote{Tversky, A., \& Kahneman, D. (1992). Advances in prospect theory: Cumulative representation of uncertainty. Journal of Risk and uncertainty, 5, 297-323.}. Nevertheless, despite these nuances, the life cycle hypothesis remains an important tool for strategic planning, especially in the context of demographic transition. 
	
	
	\centerline{\textit{\underline{Technology adoption theory}}}
	
	
	Technology Adoption Theory is an important tool for understanding the process of how people adopt and utilize new technologies. This theory becomes particularly relevant when considering the context of national payment systems, as financial technologies continue to rapidly evolve and become integrated into the daily lives of citizens. Two significant approaches within this field are the Theory of Diffusion of Innovation and <<Unified Theory of Acceptance and Use of Technology>> (UTAUT). These theories combine various concepts and models in order to explain the factors that influence technology adoption.
	
	
	UTAUT identifies four key factors influencing technology adoption\footnote{Venkatesh, V., Morris, M. G., Davis, G. B., \& Davis, F. D. (2003). User acceptance of information technology: Toward a unified view. MIS quarterly, 425-478.}: 
	\begin{enumerate}
		\item performance expectancy -- the perceived benefits of technology;
		\item effort expectancy -- ease of mastering the technology;
		\item social influence -- social pressure;
		\item facilitating conditions -- infrastructure accessibility.
	\end{enumerate}
	
	These factors not only make it possible to predict the success of the introduction of new technology, but also take into account demographic differences that may affect the perception of these aspects. For example, young people who have grown up in the digital age may perceive new payment systems as more convenient and easier to use compared to the older generation, who may be less familiar with digital tools. This highlights the importance of taking demographic factors into account when developing and implementing new payment systems.
	
	Demographic shifts, such as population aging or increased migration patterns, significantly influence the adoption of novel technologies. Older individuals may encounter obstacles when using mobile payments or digital banking services due to unfamiliarity with or apprehension about new technologies. Younger individuals, on the other hand, may be more open to exploring new financial instruments like cryptocurrencies and digital wallets. Given these demographic differences, developers of national payment systems need to tailor their offerings to meet the specific needs of different demographic groups in order to promote wider adoption of these technologies.


	Another theory that describes the adoption of new technologies is Rogers Diffusion of Innovations\footnote{Rogers, E. M., Singhal, A., \& Quinlan, M. M. (2014). Diffusion of innovations. In An integrated approach to communication theory and research (pp. 432-448). Routledge.}. A key aspect of this theory is how innovations spread within society. The theory emphasizes that adoption of a new technology doesn't happen instantaneously, but rather goes through several stages: from awareness of the technology to its eventual acceptance and use.


	The theory identified five characteristics that determine the speed of technology proliferation\footnote{Oldenburg, Brian, and Karen Glanz. "Diffusion of innovations." Health behavior and health education: Theory, research, and practice 4 (2008): 313-333}:


	\begin{enumerate}
		\item relative advantage -- the perceived benefits of technology;
		\item compatibility -- ease of mastering the technology;
		\item complexity;
		\item trialability;
		\item observability.
	\end{enumerate}



	According to the model, the population is divided into categories based on their willingness to innovate: innovators, early adopters, early majority, late majority, and laggards. In the context of national payment systems, this is manifested in the fact that young people and residents of megacities are more often referred to as <<innovators>>, while the elderly and rural residents are considered the <<late majority>> or <<laggards>>. In turn, for the successful implementation of new technologies, it is necessary to take into account not only the functional characteristics of the new technology, but also how it is perceived by various groups of the population. For example, if a new mobile payment system is perceived as complex or unreliable, it may slow down its spread among certain demographic groups.


	Another important aspect to consider is the influence of the social environment on technology adoption. Social influence plays a significant role in how individuals make decisions about adopting new technologies, as discussed in the technology adoption model. People often turn to the opinions and behaviors of their peers and family members when deciding whether to adopt new financial tools. This emphasizes the need for actively promoting and educating users about new technologies through various social channels and platforms. For instance, if young individuals start actively using a new payment system and share positive experiences with others, it can significantly accelerate the adoption process among older generations. By promoting these positive experiences, we can encourage others to try out the new technology and make informed decisions based on their own experiences.


	Thus, theories explaining how people adopt new technologies and how they spread are valuable tools for understanding how demographic factors influence the implementation of national payment systems. Knowing these relationships helps developers and regulators more effectively adapt their strategies to the changing needs of society. In an era of rapid technological progress, it is important not only to offer new solutions, but also to ensure that they are accessible and understandable to all segments of the population.


	\centerline{\textit{\underline{Human Capital Theory}}}

	The theory of human capital, as developed by Nobel Laureate Gary Becker, is crucial for understanding the relationship between demographic change and the development of national payment systems. This theory posits that knowledge, abilities, and public health constitute the basis of economic progress, as investments in education and training have a direct impact on productivity and a society capacity to adapt to technological advancements.


	According to Becker\footnote{Becker, G. S. (1962). Investment in human capital: A theoretical analysis. Journal of political economy, 70(5, Part 2), 9-49.}, investments in human capital can be considered similar to investments in physical capital. He identifies several key aspects: firstly, individuals make decisions about how much to invest in their education and health based on the expected return on these investments; secondly, the level of education and qualifications directly affects the earnings and economic activity of individuals. Becker also emphasizes that differences in the level of human capital between population groups can lead to inequality in income and opportunities.

	
	Demographic factors play an important role in understanding the theory of human capital as they affect investments in this area. The structure of a population, including age groups, levels of education and gender as well as migration patterns all influence how human capital is developed and used. For example, an aging population requires more resources for health care and education. On the other hand, young people are more likely to adopt new technologies and innovation. Galor and Weil\footnote{Galor, O., \& Weil, D. N. (2000). Population, technology, and growth: From Malthusian stagnation to the demographic transition and beyond. American economic review, 90(4), 806-828.} show that the level of education depends on demographic factors such as age structure and income. The article discusses how changes in birth rates and income affect the demand for human capital. Also, Galor and Weile's model explain how migration and urbanisation lead to concentration of human resources in certain areas.
	
	
	Furthermore, demographic changes can significantly impact the development of national payment systems\footnote{EY report 2024: How Gen Z’s preference for digital is changing the payments landscape. } With increasing levels of education and technological literacy among populations, there is a growing demand for efficient and innovative payment methods. Younger generations are more likely to prefer digital transactions compared to traditional cash, prompting financial institutions to adjust their services accordingly. This trend not only improves convenience but also promotes financial inclusion by making digital payment systems accessible to underserved populations that may lack access to traditional banking services.


	Moreover, as demographic patterns evolve, so too do the demands and preferences of customers. For example, an increasing elderly population may necessitate payment systems that are intuitive and accessible, highlighting the significance of user experience in financial technology. Conversely, younger consumers may prioritize speed and safety in their transactions, promoting innovations such as contactless payments and blockchain technology. Additionally, demographic shifts can influence the regulatory framework governing national payment systems. Policy makers must consider how transformations in population dynamics impact economic stability and customer protection in digital finance. With societies becoming more diverse, regulatory frameworks must also adapt to accommodate various attitudes towards money management and technological adoption.
	
	
	In summary, the interaction between human capital theory, demographic shifts, and national payment systems is complex and significant. By investing in education and healthcare, which are key components of human capital, societies not only increase individual productivity but also shape the financial landscape. This influence extends to how payment systems are designed and utilized.


	\centerline{\textit{\underline{Behavioral Economics Theory}}}
	
	
	Behavioral economics, which emerged at the intersection of psychology and economics, challenges traditional neoclassical models that assume complete rationality in economic agents. Its key idea is that human decision-making often deviates from optimal choices due to cognitive biases, emotions, social norms, and lack of information. The seminal works of Daniel Kahneman and Amos Tversky\footnote{Tversky, A., \& Kahneman, D. (1992). Advances in prospect theory: Cumulative representation of uncertainty. Journal of Risk and uncertainty, 5, 297-323.}, such as Prospect Theory, have laid the foundation for understanding how people evaluate risks and rewards in an uncertain environment. According to this theory, individuals tend to be <<loss averse>> -- an emotional response to losses is greater than the equivalent gain. In addition, people often use heuristics - simplified mental shortcuts such as availability (estimating probability based on ease of recall) or anchoring (reliance on initial information). These cognitive biases systematically influence financial behavior, including the choice of payment instruments, savings management, and risk management.
	
	
	An important aspect of behavioral economics is the concept of <<bounded rationality>>, introduced by Herbert Simon\footnote{Simon, H. A. (1955). A behavioral model of rational choice. The quarterly journal of economics, 99-118.}. She emphasizes that individuals make decisions in conditions of limited computing power, time, and information, which leads to satisfying instead of optimizing. For example, when choosing between several payment methods, a person may choose the first available option rather than analyzing all possible alternatives. This behavior is especially typical for groups with low levels of financial literacy or under stress. Richard Thaler\footnote{Thaler, R. H. (2015). Misbehaving: The making of behavioral economics. WW Norton \& Company.}, developing these ideas, introduced the concept of <<nudging>> -- a gentle influence on peoples choices through changing the decision-making architecture, for example, by setting a default option for automatic replenishment of a digital wallet.
	
	
	An example of using nudge for a national payment system is soft interventions aimed at encouraging consumers to switch from cash to safer and more convenient electronic payments. For example, a study conducted\footnote{Aydogan, S., \& Van Hove, L. (2015). Nudging consumers towards card payments: A field experiment. In International Cash Conference 2014 (pp. 589-630). Deutsche Bundesbank.} in a university cafeteria used posters with calls for card payments that appealed to a sense of loyalty and belonging to the university. This led to a 6\% increase in the share of cashless payments, although the effect was temporary.
	
	
	Another example is nudges aimed at increasing the use of mobile payments (Apple Pay, Google Pay, etc.) in the United States. The researchers\footnote{Story, P., Smullen, D., Acquisti, A., Cranor, L. F., Sadeh, N., \& Schaub, F. (2020). From intent to action: Nudging users towards secure mobile payments. In Sixteenth Symposium on Usable Privacy and Security (SOUPS 2020) (pp. 379-415).} used informational messages that corrected users misconceptions about the security of mobile payments and helped them formulate plans for regular use of such services. This has helped to increase the adoption of more secure payment methods.
	
	
	Also in the tax sphere, nudges have been successfully applied\footnote{Calvo-Gonzalez, O., Cruz, A., \& Hernandez, M. (2018). The Ongoing Impact of ‘Nudging’People to Pay Their Taxes. World Bank Blogs, 2.} to improve the timeliness of tax payments, using messages about fines and social approval, which is indirectly related to behavior in payment systems and can be adapted to national payment systems.
	
	
	Demographic changes also have a significant impact on peoples economic behavior. Different age groups, educational levels, and cultural conditions create unique patterns of behavior that can be explained in terms of behavioral economics.
	Thus, Greence\footnote{Greene, C., Perry, J., \& Stavins, J. (2024). Consumer Payment Behavior by Income and Demographics.} concludes that demographic factors such as age, education, income, race, and gender play a key role in shaping payment behavior and the development of the national payment system. Young and highly educated people are actively using digital solutions such as mobile apps and cryptocurrencies, while the older generation and low-income segments of the population prefer traditional payment methods such as cash and debit cards. Gender and ethnic differences, such as the higher propensity of women and members of minorities to use BNPL (Buy Now, Pay Later), emphasize the need to take into account a variety of needs when developing inclusive payment instruments.
	
	
	To summarize, behavioral economics provides valuable tools for designing national payment systems that consider demographic diversity. By understanding cognitive biases, social norms, and age-related characteristics, we can create inclusive solutions that reduce resistance to innovation.  In the context of demographic changes, such as aging populations, this approach ensures the sustainability and adaptability of financial infrastructure.
	
	
	
	\centerline{\textit{\underline{Transaction Cost Theory}}}
	
	
	The theory of transaction costs formulated by Ronald Coase\footnote{Coase, R. H. (1937). ”The Nature of the Firm”. Economica. 4 (16): 386–405.}, occupies a central place in understanding economic processes related to the exchange of resources. Transaction costs include all costs incurred in the preparation, conclusion, and implementation of transactions: information retrieval, negotiation, contract execution, control over their execution, and conflict resolution. Coase showed that the existence of firms is conditioned by the desire to minimize these costs through internal coordination of actions rather than market interactions. Later, Oliver Williamson\footnote{Williamson, O. E. (1979). Transaction-cost economics: the governance of contractual relations. The journal of Law and Economics, 22(2), 233-261.} expanded on this theory, focusing on the role of institutions, information asymmetry, and opportunistic behavior. He emphasized that the structure of transaction management -- market, hierarchy, or hybrid form -- depends on the specificity of assets, frequency of transactions, and uncertainty of conditions. These ideas formed the basis for analyzing the effectiveness of financial systems, including national payment systems. National payment systems, as an infrastructure element of the economy, directly affect the amount of transaction costs. Their main function is to provide secure, fast, and affordable settlements between market participants. For example, the introduction of electronic payments reduces the costs associated with cash processing, such as storage, transportation, and authentication. As digitalization progresses, payment systems reduce the cost of information retrieval (real-time access to balances) and transaction monitoring (automated controls). However, the development of these systems requires significant investment in technological infrastructure, protocol standardization, and regulation, creating new types of costs, such as cybersecurity and user adaptation costs.
	
	
	Demographic factors play a key role in determining the structure and dynamics of transaction costs, influencing the supply and demand for financial services. The age structure of the population, the level of education, urbanization, and migration flows shape payment behavior patterns that, in turn, determine requirements for national payment systems. Younger generations, who have grown up in the digital age, demonstrate a higher willingness to use innovative tools such as mobile applications and cryptocurrencies, which reduces the cost of implementing new technologies. Their preferences drive the development of instant payments and open APIs and decentralized solutions. In contrast, older generations tend to be more conservative in their choice of payment methods and prefer cash or traditional bank transfers, requiring financial institutions to maintain a redundant infrastructure to support these users, increasing transaction costs.
	
	
	Demographic aging of the population creates an additional burden on national payment systems. The growing proportion of older people requires the development of inclusive solutions: large fonts in interfaces, voice control, simplified authentication procedures. These changes, while increasing accessibility, lead to increased development and testing costs. At the same time, reducing the share of youth as the main driver of innovation may slow down the adoption of breakthrough technologies (blockchain, artificail intellegence), which in the long run will lead to increased transaction costs due to infrastructure obsolescence. The theory of the <<demographic dividend>> described by Bloom\footnote{Bloom, D. E., \& Williamson, J. G. (1998). Demographic transitions and economic miracles in emerging Asia. The World Bank Economic Review, 12(3), 419-455.} explains how changes in the age structure affect economic growth through productivity and savings, which indirectly affects the efficiency of payment systems.
	
	
	The relationship between demography and transaction costs is also evident in the regulatory sphere. Policy makers need to balance between stimulating innovation (reducing costs for businesses) and protecting vulnerable groups (increasing costs through compliance). For example, the introduction of strict Know Your Customer\footnote{Mullins, R. R., Ahearne, M., Lam, S. K., Hall, Z. R., \& Boichuk, J. P. (2014). Know your customer: How salesperson perceptions of customer relationship quality form and influence account profitability. Journal of Marketing, 78(6), 38-58.} requirements increases banks costs for customer verification, but reduces fraud risks, which is especially important in the context of the growth of digital transactions among the elderly. 
	
	
	Thus, the theory of transaction costs provides a powerful analytical tool for understanding the development of national payment systems in the context of demographic changes. Demographic factors, influencing behavioral patterns, technological adoption, and regulatory priorities, shape cost dynamics at the micro and macro levels. Minimizing these costs requires a flexible approach that combines technological innovation, educational initiatives, and the adaptation of institutions to a changing age landscape. 
	
	
	\centerline{\textit{\underline{Comparison of theoretical approaches}}}
	
	Given the diverse range of theoretical approaches available, it is necessary to select one in order to undertake a more in-depth exploration of the subject through the chosen methodology. To make this decision, we will employ the method of comparative analysis, based on a specific formula:
	
	\begin{equation}
		U_j = \sum_{i=1}^{n} u_{ij}, 
	\end{equation}
	
	
	where $n$ -- amount of criteria, $U_j$ -- total score for alternatives by criteria $j$, $u_{ij}$ -- score for $j$ alternatives by criteria $i$.
	
	
	A framework for assessing options:
	
	
	\begin{itemize}
		\item 1 -- hardly applicable;
		\item 2 -- generally applicable;
		\item 3 -- applicable.
	\end{itemize}
	
	
	The following characteristics were identified as criteria:
	
	
	\begin{itemize}
		\item Relevance;
		\item Considers Demographic Factors;
		\item Considers Geographic Component;
		\item Considers Dynamics.
	\end{itemize}
	
	
	
	\begin{longtable}
		{|>{\centering\small\arraybackslash}p{2.5cm}
			|>{\centering\small\arraybackslash}p{1.8cm}
			|>{\centering\small\arraybackslash}p{3cm}
			|>{\centering\small\arraybackslash}p{2cm}
			|>{\centering\small\arraybackslash}p{3cm}
			|>{\centering\small\arraybackslash}p{1.5cm}|
		}
		\caption{The application of the method of comparative alternatives in order to study theories in the context of impact demographic factors on the development of the national payment systems.}\label{tab:conclusion_theory}\\
		
		\hline
		\textbf{Theory} & \textbf{Relevance} & \textbf{Considers Demographic Factors} & \textbf{Considers Dynamics} & \textbf{Considers Geographic Component} & \textbf{Total}
		\\\hline
		\endfirsthead
		
		
		\multicolumn{6}{r}{Continuation of the table \ref{tab:conclusion_theory}}\\\hline
		\endlasthead
		
		\multicolumn{6}{@{} l}{\small\centering{Source: compiled by the author based on the analysis of theoretical framework}}
		\endfoot
		
		Life Cycle Hypothesis & 3 & 3 & 3 & 3  & 12\\\hline
		Technology adoption theory  & 2 & 2 & 3 & 2 & 9\\\hline
		Human Capital Theory  & 2 & 3 & 2 & 2 & 9\\\hline
		Behavioral Economics Theory  & 3 & 3 & 1 & 1 & 8\\\hline
		Transaction Cost Theory  & 2 & 1 & 1 & 2 & 6\\\hline
		
	\end{longtable}
	
	
	Based on the evaluation results (Table \ref{tab:conclusion_theory}), it appears that the life cycle hypothesis is the most suitable approach, as it comprehensively reflects the impact of demographic factors on the development of national payment systems, considering both dynamic and geographical variations. Therefore, our future work will be based on the theory proposed by Franco Modigliani, whose studies have demonstrated their applicability and relevance in this context.
	
	\section{Application of the  Modigliani Life Cycle Hypothesis to analyze the impact of demographic factors on the development of the national payment system}
	
	The life cycle hypothesis, proposed by economist Franco Modigliani in 1957, offers a fundamental theory for understanding consumer behavior over time. This hypothesis suggests that people tend to maintain a consistent level of consumption regardless of income fluctuations throughout their lives. It assumes that individuals carefully plan their finances by allocating resources between periods with high and low incomes in order to ensure a stable living standard. In 1985, Modigliani was awarded the Nobel Prize in Economics for his work on this theory.
	
	
	The core idea behind the life cycle hypothesis is that people plan their consumption and savings to achieve a steady living standard over time. Specifically, the theory proposes: 
	
	\begin{itemize}
		\item Individuals consider not only their short-term consumption and savings but also plan for the long-term. They distribute resources among different income periods in order to maintain a consistent living standard;
		
		\item Short-term changes in income have a relatively minor impact on long-term spending patterns, as most peoples income tends to be relatively stable over time;
		
		\item Individuals build up assets and spread their consumption more evenly throughout time, rather than solely relying on current income;
		
		\item People often borrow during times when their income is low, such as while they are studying or at the start of their career, and save when their income is higher, like during the peak of their success.
	\end{itemize}
	
	The Figure \ref{fig:lch} clearly illustrates the three main stages of the life cycle hypothesis\footnote{Ketkaew, Chavis, Martine Van Wouwe, Preecha Vichitthamaros, and Duanpen Teerawanviwat. "The effect of expected income on wealth accumulation and retirement contribution of Thai wageworkers." SAGE Open 9, no. 4 (2019): 2158244019898247.}:
	
	\begin{enumerate}
		\item \textit{Youth phase (15-35 years old)}: During this period, income is usually low and a person may be actively borrowing, meaning they have negative savings. Consumptions are mainly supported by loans and future expected income;
		\item \textit{Mature phase (35-65 years old)}: This is the time when income reaches its maximum and a person actively accumulates savings. It is during this period when most of the assets used in retirement are formed;
		\item \textit{Retirement phase (65+ years old)}: With the onset of retirement, income decreases sharply and consumption is supported by gradually spending previously accumulated funds.
	\end{enumerate}
	
	
	
	\begin{figure}[H]
		\centering
		\includegraphics[width=0.7\textwidth]{life_cycle_graph.png}
		\caption{Life Cycle Hypothesis Illustration\\Source: <<The Effect of Expected Income on Wealth Accumulation and Retirement Contribution of Thai Wageworkers>> by Ketkaew}
		\label{fig:lch}
	\end{figure}

	
	
	The life cycle hypothesis is often considered using a model of savings and consumption. In this model, it is assumed that a persons total consumption during his lifetime is equal to the sum of his income.
	
	
	The formula describing this principle can be represented as follows: the sum of consumption for all periods of life is equal to the sum of income for all these periods. In more detail, the formula looks like this:
	
	\begin{equation}
		C  = \frac{W + R*Y}{T},
		\label{eq:lch}
	\end{equation}
 where $C$ is consumption, $T$ is life expectancy, W is initial wealth, $R$ is the number of working years, and $Y$ is income during these years.
	
	
	Demographic factors can be taken into account in the model by changing the parameters. For example, an increase in life expectancy ($T$) leads to the need to accumulate more savings during work in order to maintain consumption levels over a longer period of retirement age. This, in turn, can affect the structure of savings and, as a result, the development of financial instruments, which, in turn, has an impact on payment systems.
	
	
	A decrease in the number of younger generations (a reduction in fertility rate) leads to a decrease in the proportion of people in the working-age group, which directly affects the life cycle equation \ref{eq:lch}. A decrease in this group, with constant values for $W$, $Y$, and a growing value of $T$  leads to decreased consumption. To counteract this effect, households can either increase their savings during working years or reduce spending during retirement. Both scenarios increase demand for pension products and stimulate the development of automated payment platforms for managing long-term financial cycles. In addition, an aging population may have some difficulties using new technologies. This highlights the need to develop intuitive interfaces and accessible solutions for all age groups. Given that older people may not be very familiar with digital technologies, it is important to create conditions so that they can easily use payment systems.
	
	
	In the context of the analysis of demographic factors influencing the development of national payment systems, a number of observations can be made:
	\begin{enumerate}
		\item \textbf{Life expectancy}: As this indicator increases, the need to accumulate savings increases to ensure financial stability in retirement, which, in turn, affects the structure of savings and the development of financial instruments;
		\item \textbf{Decreasing fertility rate}:  A decrease in the number of younger generations leads to a reduction in labor resources, which reduces consumption while maintaining initial indicators of wealth and income. This may lead to an increase in demand for pension products, thereby stimulating the development of automated payment platforms;
		\item \textbf{Population aging} is a global trend that requires close attention and adaptation to new realities. The elderly, as the most vulnerable group, face difficulties in mastering modern technologies. This highlights the need to create intuitive and accessible interfaces that will facilitate the integration of older people into the digital space.
	\end{enumerate}
	
	\chapter{An empirical study of the impact of demographic factors on the development of the National Payment System of Russia}
	
	
	\section{Analysis of empirical approaches to assess the impact of demographic factors on the development of national payment systems}
	
	
	This section will review empirical research on the impact of demographic factors on the development of national payment systems. Variables will then be identified for subsequent econometric research.
	
	
	One of the traditional approaches to examining the influence of demographic variables on the development of national payment systems involves the utilization of survey data. Stavins\footnote{Stavins, J. (2016). The effect of demographics on payment behavior: panel data with sample selection (No. 16-5). Working Papers.} article explores the role of demographic and socioeconomic variables in shaping consumer payment habits in the United States, utilizing data from the <<Consumer Payment Choice Survey>> (CPCS) for the years 2009–2013. The writer analyzes how factors such as age, educational attainment, income, racial background, gender, and other attributes influence the adoption and use of various payment methods, including cash, credit cards, debit cards, online banking, and checks.
	The primary objective of the research is to identify trends that persist even when accounting for the attributes of each payment method (convenience, safety, and cost) and other factors. The study employs a panel model incorporating a selection adjustment proposed by Wooldridge\footnote{Wooldridge, J. M. 1995. “Selection Corrections for Panel Data Models Under Conditional Mean
		Independence Assumptions.” Journal of Econometrics 68 (1): 115–132.}.
	At the initial stage, a probabilistic regression model is used to estimate the probability of acceptance of a particular payment instrument $j$, where the probability of using the payment instrument is predicted for each year $t$:
	
	\begin{equation}
		P(s_{ijt} = 1 |  x_{it}) = \Phi \left( \delta_{t0} + x_{it} \delta_t + \vartheta_{it} \right),
	\end{equation}
	
	
	where $s_{ijt} = 1$ -- if the consumer $i$ accepted the instrument $j$ in the year $t$, and 0 otherwise; $x_{it}$ is a vector of independent variables (demographics, income, region, instrument characteristics).
	
	
	Then, the degree of its use is estimated using the usual least squares regression with random effects, which allowed the authors to take into account the fact that a persons propensity to use a particular instrument may vary in depending on his mastery of this tool:
	
	\begin{equation}
		y_{ijt} = c + x_{it} \beta + \gamma_t \hat{\lambda}_{ijt} + \alpha_i + u_{it},
	\end{equation}
	
	where $y_{ijt}$ is a proportion of transactions using the $j$ instrument; $\alpha_i$ is a random individual effects; $\gamma_t$ is a coefficient reflecting the selection adjustment; $\hat{\lambda}_{ijt}$ is a Inverse Mills Ratio. The selection of random effects over fixed effects is justified by the results of the Hausman test, which confirms the absence of a correlation between individual-specific effects and the independent variables.
	
	
	The author concludes that young people aged 18-25 are more likely to use online banking and debit cards, while older people over 65 continue to use checks. Women also spend 5\% less in cash, but they use debit cards and check more frequently, which may be due to their household budget management strategies.
	
	
	A further study\footnote{Greene, C., Perry, J., \& Stavins, J. (2024). Consumer Payment Behavior by Income and Demographics.
	} using the same CPCS database for 2023 also shows similar results. The authors demonstrate a stable relationship between socio-economic factors and preferences for both traditional and innovative payment methods. They apply the two-stage Heckman model to analyze the extensive and intensive margins of payment instrument use. The interesting result is that university graduates are 2.5 times more likely to use credit cards than those who have not completed high school. However, the least educated and lower-income groups continue to prefer cash and debit cards, consistent with the theory of financial vulnerability.
	
	
	The article by Camilleri and Agius\footnote{Camilleri, S. J., \& Agius, C. (2021). Choosing between innovative and traditional payment systems: an empirical analysis of European trends. Journal of Innovation Management, 9(4), 29-57.
	} examines the factors that influence the choice between traditional and innovative payment systems in Europe, with a particular focus on demographic characteristics. The study combines econometric analysis at the macro level (across 28 EU countries) with a survey of individual preferences in Malta. This allows us to identify both general trends and specific characteristics of a particular region. Demographic variables, such as age, education, gender, and employment, are central to this work, as they highlight their role in adopting new technologies. Age stratification has proven to be a significant factor. The econometric model revealed an ambiguous relationship, with the proportion of people aged 15-64 being negatively correlated with the use of traditional methods, consistent with the hypothesis that the economically active population has greater technological literacy. However, there was also a positive correlation between the proportion of 20-39 year olds and the use of these traditional systems, which was unexpected. The authors suggest that this group, though technologically oriented, may still maintain habits due to convenience, such as using debit cards, and this requires further investigation. The results of a survey in Malta with 90 respondents confirmed the significance of age. Older respondents over the age of 60 showed a strong preference for cash and a complete lack of interest in mobile banking, while younger respondents aged 16-29 were more likely to use online banking and mobile payments.
	
	
	Another fascinating study\footnote{Тишин, А. (2020). Влияние демографии на развитие финансового сектора Российской Федерации. Аналитическая записка Департамента исследований и прогнозирования, Банк России.} is the analysis of the impact of demographic factors on the development of the financial sector in Russia. This study is based on Rosstats demographic forecasts, which cover three scenarios: low, medium and high. In addition, the work uses data from household surveys conducted by the Russian Ministry of Finance in 2013 and 2015. The research methodology includes logistic regression to assess the probability of owning financial assets and liabilities, as well as predictive calculations based on the age structure of the population. It is important to note that in the modeling, assumptions were made that the younger generation, as a rule, seeks to minimize the use of cash and adopts the habit of using cards. The results indicate that, for instance, the proportion of the population requiring loans will decrease by 3-3.5 percentage points by 2036, while the use of credit cards will increase by 4-4.5 percent. Meanwhile, the amount of deposits and mortgages will slightly decrease. A significant aspect of this study was an analysis of age groups contributions: reductions among those aged 25-44 had a more significant negative impact on credit activity, while older groups maintained more conservative preferences such as saving money instead of spending it.
	
	
	In the context of studying the influence of demographic factors on the development of the national payment system, the Arkhipova paper\footnote{Arkhipova, N. (2022). Impact of Demographic Trends on Retail Banking. Procedia Computer Science, 214, 831-836.
	} is of particular interest. Her research focuses on analyzing the relationship between socio-demographic trends and retail banking dynamics in the regions of Russia. The author identifies key demographic and social variables that influence public demand for banking products, such as lending and savings. The research is based on econometric analysis using panel data from 78 Russian regions over the period 2010 - 2020. Regression models, including autoregressive components, are used to assess the impact of demographic, social, and economic factors on lending volumes. The formalized model takes into account various variables and their interactions, allowing for a more accurate assessment of the complex relationship between demographic trends and banking activity:
	
	\begin{equation}
		BI_t = \sum_{i,t} \alpha_i FC_{i,t} + \sum_{j,t} \beta_j D_{j,t} + \sum_{k,t} \gamma_k FI_{t} + \sum_{l,t,n} \mu_l BI_{t-n} + \epsilon_t,
	\end{equation}
	
	
	where $BI$ is indicators of banking product penetration; $FC$ is a control variables (income level, etc.); $D$ is a demographic characteristics of Russian regions; $FI$ is an indicators of financial innovation.
	
	Arkhipova identifies several key demographic trends relevant to the financial sector:
	
	\begin{itemize}
		\item the increasing number of older adults is changing the way people save and borrow money;
		\item high proportion of the urban population is associated with a greater propensity to use innovative financial products, including cashless transactions;
		\item growing number of single-parent and same-sex families requires the adaptation of credit products, which affects the design of payment instruments.
		
	\end{itemize}
	
	
	As part of the research on impact of demographic factors on the development of national payment system, a paper by Chawla and Joshi\footnote{Chawla, D., \& Joshi, H. (2018). The moderating effect of demographic variables on mobile banking adoption: An empirical investigation. Global Business Review, 19(3\_suppl), S90-S113.} is of significant interest. This study examines the moderating impact of demographic variables on the adoption of mobile banking. The authors integrated the Technology Acceptance Model and Innovation Diffusion Theory to develop a model that incorporates variables such as trust, convenience, efficiency, compatibility with lifestyle, and ease of use. Their focus was on identifying the significance of demographic traits in determining the magnitude of these factors impact on users attitudes toward mobile banking services. The study was conducted on data from 367 participants in India, predominantly male (74.4\%), with a high educational background (88.3\%). Multiple linear regression and Fishers transformation were employed for analysis. The authors concluded that female participants were more likely to be influenced by ease of use, suggesting that demographic factors significantly influenced attitudes towards mobile banking services. Participants over the age of 30 demonstrated a stronger link between trust and attitudes, which can be attributed to their cautious approach towards digital risks. Furthermore, married participants tended to prioritize convenience, likely owing to their responsibility for managing family finances.
	
	
	Using data from a survey conducted in China in 2017, the Vatsa study\footnote{Vatsa, P., Ma, W., \& Zheng, H. (2024). Mobile payment adoption in China: Do demographic and socioeconomic factors matter?. Managerial and Decision Economics, 45(3), 1428-1434.} examined the relationship between various factors and the adoption of mobile payments. The results showed that factors such as education, perceived health status, economic status, political affiliation, employment status, income, social activity, car ownership, Internet access, and geographical location positively correlate with the use of mobile payment systems. However, older people and men were less likely to use these systems compared to their younger counterparts and women. Interestingly, the study found no correlation between subjective well-being and the use of mobile payments.
	
	
	Another approach to assessing the adoption of cashless payment technology is an article by Crouzet \footnote{Crouzet, N., Ghosh, P., Gupta, A., \& Mezzanotti, F. (2024). Demographics and technology diffusion: Evidence from mobile payments. Available at SSRN.
	} who explore the impact of the age structure of the population on the rate of technology diffusion, focusing on mobile payments in India. The authors use transaction data from 200,000 Indian bank customers and data on the introduction of QR terminals by a fintech company. Empirical analysis is combined with a theoretical model where the age of consumers determines preferences in choosing technologies. The authors model formalizes the relationship between demographics and business decisions. The demand for technology among young consumers is described:
	
	\begin{equation}		
		s_{Y}(p(j), a(j)) = \frac{a(j)}{J} (\frac{p(j)}{P_{Y}})^{-\frac{1}{v-1}},
	\end{equation}
	
	
	where $a(j)$ is the level of technology adoption by businesses $j$; $P_{Y}$ is a price index for the young population; $v$ ia an elasticity of substitution.
	
	
	The key conclusion of the article is that age is a significant factor: 38\% of the variation in the share of mobile payments is explained by the age of users. Young consumers (18-30 years old) use mobile payments twice as often as older groups (60+). This is consistent with Rogers theory of innovation diffusion, where young people act as <<early adopters>>.
	
	
	It is also important to identify the control variables that influence the development of national payment systems. For example, Carvalho's article analyzes the impact of demographic trends on real interest rates in countries with insufficient capital mobility. The authors have developed a multidimensional general equilibrium model that takes into account various demographic characteristics and constraints on international capital flows. The main conclusion of the study is that the aging of the population, a decrease in the birth rate and an increase in life expectancy account for about a third of the global decline in real interest rates between 1990 and 2019. Another important factor to consider is the Covid-19 pandemic. In this regard, the authors\footnote{Graziano, E. A., Musella, F., \& Petroccione, G. (2024). Cashless payment: behavior changes and gender dynamics during the COVID-19 pandemic. EuroMed Journal of Business, 20(5), 54-74.
	} examine the impact of the pandemic on the adoption of cashless payments in Italy, focusing on demographic factors, financial literacy, and gender differences. The study is based on data collected through an online survey conducted among 836 respondents between November 2021 and February 2022. Panel regression analysis was used to analyze the data. Fear of COVID-19 was found to have a direct positive impact on the use of cashless payment methods. However, the level of financial literacy was found to be less influential. Gender analysis revealed no significant differences in the acceptance of cashless payments between men and women. Additionally, it was noted that media coverage of the pandemic indirectly influenced attitudes towards cashless payments through the fear of infection.
	
	
	
	\begin{longtable}
		{|>{\centering\small\arraybackslash}p{6cm}
			|>{\centering\small\arraybackslash}p{9.3cm}|
		}
		\caption{Conclusions obtained as a result of a literary review of empirical works}\label{tab:conclusion}\\
		
		\hline
		\textbf{Paper} & \textbf{Brief conclusions}
		\\\hline
		\endfirsthead
		
		
		\multicolumn{2}{r}{Continuation of the table \ref{tab:conclusion}}\\\hline
		\endlasthead
		
		\multicolumn{2}{@{} l}{\small\centering{Source: compiled by the author based on the analysis of empirical works}}
		\endfoot
		
		\multicolumn{2}{|c|}{\small\textbf{Analysis of demographic factors at the respondent level}}\\\hline
		
		Stavins (2016); Greene, Perry, Stavins (2024) &  Random effects model, with a correction for sampling bias, and the two-stage Heckman model, show that young individuals aged 18-25 are more likely to choose online banking and debit cards. In contrast, older individuals aged over 65 tend to use checks\\\hline
		
		Camilleri, Agius (2021) & Age stratification: the economically active population uses traditional methods less often, but the 20-39 age group still uses them because of habit\\\hline
		
		
		\multicolumn{2}{|c|}{\small\textbf{Analysis of aggregated data by regions}}\\\hline
		
		 Arkhipova (2022) & An analysis of data collected in 78 regions of Russia shows that with an increase in the proportion of elderly people in society, the structure of savings is changing. In particular, urban residents are more actively using innovative payment solutions.\\\hline
		
		Tishin (2020) & Examination of effects population shifts on Russia and specific financial metrics, based on the demographic projections provided by Rosstat until 2036\\\hline
		
		
		\multicolumn{2}{|c|}{\small\textbf{Analysis of control variables}}\\\hline
		
		Graziano et al. (2024) & The impact of COVID-19 on cashless payments in Italy: Due to the fear of COVID-19, there has been an increase in the use of cashless payment methods, without any gender differences\\\hline
		
		\multicolumn{2}{|c|}{\small\textbf{Technology adoption analysis}}\\\hline
		
		Chawla, Joshi (2018) & Demographic factors influence the use of mobile banking in India: women like it more because it is more convenient. People over the age of 30 trust more\\\hline
		
		Crouzet et al. (2024) & Models of innovation dissemination proposed by Rogers, as well as the analysis of demand elasticity, led to the conclusion that young people are among the <<early adopters>>\\\hline
		
	\end{longtable}
	
	
	In conclusion, to track the progress of the national payment system, the Bank of Russia can rely on demographic data such as age and gender differences, as well as control variables, including the key interest rate. By analyzing changes in these indicators, the Bank of Russia is able to identify potential areas for improvement and take appropriate steps. The review of empirical studies provides a deeper understanding of the effectiveness of various indicators and methods, which are summarized in the table \ref{tab:conclusion}.
	
	
	
	\section{Construction of an index for the development of the national payment system in Russia}
	
	
	The level of development of national payment systems in Russia is often characterized by a high degree of regional heterogeneity, making it difficult to implement a unified strategy for regulating and supporting them. To address this issue, it is necessary to create a comprehensive index that can quantify the state of payment infrastructure across different regions, identify imbalances, and serve as a basis for Central Bank policy. In this chapter, we will develop a methodology to calculate an integral index for the development of the national payment system (NPSI) in Russia. This index will allow us to measure objectively the level of payment infrastructure development across regions and establish a connection between the development of payment systems and demographic factors.
	
	
	The construction of integral indexes is a common practice in economic and social research. Such indexes, which combine several different indicators into a single indicator, allow for a comprehensive and comparable assessment of the phenomenon under study. Among the most famous examples are the Human Development Index (HDI)\footnote{Ul Haq, M. (1995). Reflections on human development. oxford university Press.}, Digital Economy and Society Index (DESI) \footnote{Bánhidi, Z., Dobos, I., \& Nemeslaki, A. (2020). What the overall Digital Economy and Society Index reveals: A statistical analysis of the DESI EU28 dimensions. Regional Statistics, 10(2).}, the index, as well as Urban Health Index (UHI)\footnote{World Health Organization. (2014). The urban health index: A handbook for its calculation and use. In Kobe, Japan.} from the World Health Organization.
	
	
	Sympathize with the UHI methodologies used to build the indexes in this study, they are versatile and flexible. This methodology provides for the standardization of individual indicators with subsequent aggregation by means of a geometric mean, which minimizes the impact of extreme values and makes the index resistant to data variability. This approach makes it possible to effectively assess and compare the development of complex multifactorial phenomena, which include the national payment system.
	
	
	The UHI methodology was selected to evaluate the development of Russia's national payment system due to several factors. First, a payment system similar to urban infrastructure and healthcare has a wide range of components with different units of measurement. Second, this methodology takes into account spatial and regional differences in Russia diverse socio-economic and demographic landscape. Third, standardization and averaging minimize potential errors caused by extreme regional variations.
	
	To calculate the integral index, the following sets of Bank of Russia indicators were selected for the period from 2013 to 2023 for all regions\footnote{Except of Crymia, Sevastopol, DPR, LPR, Zaporizhia Oblast and Kherson Oblast of the fact that there is lack of data for these regions} of Russia\footnote{https://www.cbr.ru/statistics/nps/psrf/}:
	
	\begin{enumerate}
		\item Institutional provision of payment services in the territorial context -- It shows the degree of development of the infrastructure of financial institutions and the availability of banking services in a particular region;
		\item Number of accounts opened by institutions of the banking system, by territory  -- it reflects the level of financial accessibility and involvement of the region population in the banking sector;
		\item Number of payments made through credit institutions (by payment instruments) in the territorial context -- it allows to assess the intensity of the populatio use of payment services and the prevalence of cashless payments;
		\item Volume of payments made through credit institutions (for payment instruments) in the territorial context -- complements the previous indicator, characterizing not only the number of transactions, but also their financial significance;
		\item Number of electronic payment orders  -- it shows the level of digitalization of the payment behavior of the population and the degree of penetration of electronic payment channels;
		\item Volume of electronic payment orders  -- assessment of the degree of integration of electronic payment technologies into the daily economic life of the region.
	\end{enumerate}
	
	
	In general, the resulting dataset contains 40 different variables that are related to each other. Due to their large number and the fact that they clearly correlate, a primary correlation analysis was conducted using Spearman's rank correlation. The final set of indicators and their descriptive statistics are presented in Table \ref{tab:stats}. Their correlation matrix is shown in the figure \ref{fig:corr}.
	
	The next step in building an index is to standardize the data (calculated separately for each year). The standardization method is applied to convert the data into a uniform, comparable format:
	
	\begin{equation}
		S_i = \frac{I_i - \min(I_i)}{\max(I_i) - \min(I_i)},
	\end{equation}
	
	
	where \( S_i \) -- standardized indicator value;  \( I_i \) -- the initial value of the indicator for the ith region; \( \min(I_i)\), \( \max(I_i)\) are the minimum and maximum values of the corresponding indicator among all regions of Russia.
	
	
	\begin{longtable}
		{|>{\centering\scriptsize}p{10em}
			|>{\centering\scriptsize}p{14em}
			|>{\centering\scriptsize}p{3em}
			|>{\centering\scriptsize\arraybackslash}p{2.3em}|
		}
		
		\caption{Descriptive statistics of the data used to build the index of development of the national payment system of Russia}\label{tab:stats}
		\\
		
		\hline
		\textbf{Variable Name} & \textbf{Source} & \textbf{Mean} & \textbf{Standart Deviation}
		\\\hline
		\endfirsthead
		
		\multicolumn{4}{r}{Continuation of the table \ref{tab:stats}}\\\hline
		\endhead
		
		\multicolumn{4}{r}{End of the table \ref{tab:stats}}\\\hline
		\endlasthead
		
		\multicolumn{4}{@{} l}{\centering{Source: compiled by the author based on the data analyzed}}
		\endfoot
		
		Total number of banking system institutions & Institutional provision of payment services in the territorial context, Bank of Russia & 432.85 &  639.38\\\hline
		
		The number of banks per million people & Institutional provision of payment services in the territorial context, Bank of Russia & 235.45 & 85.07\\\hline
		
		The number of accounts opened by banking institutions per resident & Number of accounts opened by institutions of the banking system, by territory, Bank of Russia & 4.33 & 2.02\\\hline
		
		Total amount of payment requirements with internet & Number of payments made through credit institutions (by payment instruments) in the territorial context, Bank of Russia & 40.47 & 127.35\\\hline
		
		Volume of bank orders in billion rubbles & Volume of payments made through credit institutions (for payment instruments) in the territorial context, Bank of Russia & 29.71 & 173.83\\\hline
		
		Volume of all orders in electronic form with internet & Volume of electronic payment orders, Bank of Russia & 1808.66 & 12677.03\\\hline
		
		
	\end{longtable}
	
	
	\begin{figure}[H]
		\centering
		\noindent\includegraphics[width=0.5\linewidth]{corr.png}
		\caption{Correlation matrix\\Source: compiled by the author based on the data analyze}
		\label{fig:corr}
	\end{figure}
	
	
	The last step is to aggregate standardized indicators into an integral index. A geometric mean is used to combine standardized values:
	
	\begin{equation}
		Index = \left( \prod_{i=1}^{n} S_i \right)^{\frac{1}{n}}
	\end{equation}
	
	Using a geometric mean minimizes the impact of sharp regional disparities and provides a more balanced assessment. Logarithmization was also applied to balance outliers in the data before calculating the geometric mean.
	
	
	The index of development of the national payment system in Russia, obtained during the study, shows interesting results (Table \ref{tab:index_final}). The index takes values from 0 to 1. The closer the value is to 1, the higher the level of development of the payment system in the region.
	
	
	\begin{table}[H]
		\caption{The top 5 and bottom 5 regions of Russia according to the national payment system development index for October 2023}\label{tab:index_final}
		\centering
		\begin{tabular}%{\linewidth}
			{|c|c|c|c|}\hline
			\textbf{Regions-<<leaders>>} & \textbf{Index Value} & \textbf{Regions-<<losers>>}& \textbf{Index Value}\\\hline
			
			Moscow & 0.94 &  Republic of Ingushetia & 0.09\\\hline
			
			St. Petersburg & 0.69 &  Republic of Dagestan & 0.20\\\hline
			
			Novosibirsk Oblast & 0.63 &  Chechen Republic & 0.21\\\hline
			
			Sverldlovsk Oblast & 0.62 & Republic of Kalmykia & 0.26\\\hline
			
			Voronezh Oblast & 0.61 & Jewish Autonomous Oblast & 0.28\\\hline
			
		\end{tabular}
		Source: compiled by the author based on data from the Bank of Russia
	\end{table}
	
	
	It is worth noting that Moscow and St. Petersburg occupy leading positions in this index, which is understandable, since they are the financial centers of the country with the largest population. In addition, technologically advanced and urbanized regions such as Novosibirsk and Sverdlovsk regions are on the list of leaders. The underdeveloped southern regions complete the ranking according to this index. For a full-fledged comparison of the index, competence diagrams were constructed for 3 types of regions: top, middle, and losers (Figure \ref{fig:compare}). Based on the diagrams, only one conclusion can be drawn -- in Russia there is a high differentiation between the top regions (Moscow, St. Petersburg) and outsiders (Dagestan, Chechnya).
	
	
	\begin{figure}[H]
		\centering
		\includegraphics[width=0.3\textwidth]{losers.png}
		\hfill % Заполнитель пространства между изображениями
		\includegraphics[width=0.3\textwidth]{top.png}
		\hfill
		\includegraphics[width=0.3\textwidth]{middle.png}

		\caption{Competence diagrams\\Source: compiled by the author based on the data analisys}
		\label{fig:compare}
	\end{figure}


	It is also possible to compile a generalized index that will reflect the situation in Russia as a whole. However, given the significant differences between regions, the distribution of the index will have a wide range of values, which will undoubtedly affect the integral indicator. Therefore, instead of the regional average, it was decided to use the median. The graph of this indicator(National Payment System Development Index -- NPSD Index), was built from 2013 to 2023 (Figure \ref{fig:index_whole}). As can be seen from the graph, there was a decrease in the index in 2014-2015, which can be explained by sanctions against Russia. It was during this period that the National Card Payment System and the MIR payment system were created. Until 2020, the index showed significant growth due to the development of these systems and the introduction of new technologies and processes into the National Payment System of Russia, for example, the fast payment system in 2019. However, growth has slowed during the pandemic. Since 2022, the index has begun to decline, which is associated with the general economic crisis in Russia. During this period, the key interest rate was high, and the ruble was unstable, which aggravated the situation. The dynamics of the index was also affected by internal shocks, which also had a negative impact.
	
	\begin{figure}[H]
		\centering
		\noindent\includegraphics[width=0.7\linewidth]{index_whole.png}
		\caption{Aggregated National Payment System Development (NPSD) Index for Russia\\Source: compiled by the author based on the data analyze}
		\label{fig:index_whole}
	\end{figure}
	
	
	Summing up, the following conclusions are drawn:
	
	\begin{enumerate}
		\item It is feasible to develop an index that would objectively reflect the development of the National Payment System of Russia both at the regional level and nationwide, based on established methodologies for index development (UHI) -- which is related to the similarity of tasks: the need to aggregate heterogeneous indicators, taking into account regional specifics;
		\item The results of the calculations revealed significant differences between the regions. Moscow (0.94) and St. Petersburg (0.69) were the leaders, which is explained by their status as financial centers with a high concentration of infrastructure and population. The Republic of Ingushetia (0.09) and Dagestan (0.20), on the other hand, are among the regions with the lowest levels of development of payment systems;
		\item The calculated index demonstrates not only the current state of the payment system, but also its vulnerability to macroeconomic factors. Its application allows the Bank of Russia to identify problem regions, evaluate the effectiveness of regulatory measures and develop targeted strategies to reduce imbalances.
	\end{enumerate}
	
	
	\section{Developing an econometric model to estimate the impact of demographic variables on the share of cashless transactions in total retail turnover in Russia}
	
	
	To develop an econometric model that can analyze the impact of demographic variables on cashless payment share in total retail turnover, there will identify variables that influence cashless payments utilization based on economic principles, empirical studies, and regulatory guidelines. Additionally, there will conduct trend analysis to accurately specify the model and validate the simulation outcomes.


	The data on cashless payments of Sberbank was used as a target -- <<the share of expenses for goods and services paid by cashless method. It is calculated based on the amount of expenses and does not include transactions that cannot be unambiguously attributed to total retail turnover (for example, transfers from one person to another)>>.The model includes annual data from 2017 to 2023 for all regions of Russia.  This time period is due to the fact that the Sberbank began publishing statistics in 2017\footnote{https://sberindex.ru/ru/dashboards/dolya-beznala}. The statistics on the average in Russia for all regions can be seen on the Figure \ref{fig:cashless_russia}.
	
	
	\begin{figure}[H]
		\centering
		\noindent\includegraphics[width=0.61\linewidth]{cashless_russia.png}
		\caption{The share of cashless payments in Russia from 2018 to 2023\\Source: compiled by the author based on the data analyze}
		\label{fig:cashless_russia}
	\end{figure}
	
	
	To analyze the impact of demographic factors on the share of cashless payments, based on a review of theoretical and empirical work, the following demographic indicators were selected (Table \ref{tab:data_model}) (the data is taken from the <<Unified Interdepartmental Information and Statistical System>> website\footnote{https://fedstat.ru/} -- UIISS -- and New Economic School Russian database on fertility and mortality -- NES RusFMD\footnote{https://www.nes.ru/demogr-fermort-data?lang=en}):
	
	\begin{enumerate}
		\item \textit{Life expectancy at birth by Russians regions} -- <<the number of years that, on average, one person from a generation born in a given year would have to live, provided that throughout the life of this generation, the death rate at each age remains the same as in the year for which the indicator was calculated>>\footnote{https://fedstat.ru/indicator/31293}. An increase in this indicator leads to an increase in the proportion of older citizens whose financial preferences differ significantly from those of young people. According to theoretical and empirical studies, older people tend to save more money and use more conservative payment methods, which can slow down innovation. However, the effects of aging will not appear immediately: today increase in life expectancy will affect the structure of effective demand in a few years, when the current 40-50 year olds enter retirement age. Therefore, lag variables for this factor were added to the model;
		
		\item \textit{Total fertility rate by Russians regions} -- <<The total fertility rate shows how much, on average, one woman would give birth throughout the entire reproductive period (i.e., from 15 to 50 years), while maintaining the birth rate of the year for which the indicator is calculated at each age. Its value does not depend on the age composition of the population and characterizes the average birth rate in a given calendar period>>\footnote{https://fedstat.ru/indicator/31517}. This indicator reflects the potential for future economic activity is the low birth rate. This is typical of many developed countries and leads to a decrease in the proportion of young people, who are the main drivers of technological innovation. The younger generation that grew up in a digital environment are <<early adopters>> of new payment services. However, the demographic decline caused by the decrease in birthrate in the 1990s has limited the influx of these users, and this time lag allows us to connect past fluctuations in birth rates to the current pace of digitalization;
		
		\item \textit{All-cause mortality rate, cases per 1,000 population} -- <<Age coefficients measure the mortality rate for individual age groups. They are calculated as the ratio of the absolute number of deaths in a given age group during the reporting period to its average number>>\footnote{https://fedstat.ru/indicator/43516}. The mortality rate is a marker of socio-economic well-being. High mortality rates (as in the 1990s) disrupt the continuity of generations, reducing the transfer of financial experience and slowing the spread of digital tools. Reducing mortality contributes to the preservation of human capital, but it takes time to transform consumer habits;
		
		\item \textit{Age structure factors} -- <<The ratio of the mid-year population to the total population estimates across regions of Russia by  five year groups>>\footnote{https://www.nes.ru/files/research/demogr/mort-database/description/ENG/01-DemographicDatabaseDescription-ENG-20-01-14.pdf}.It is expected that different generations will form their financial habits based on the achievements of scientific and technological progress.
		
		\item \textit{Control variables} sush as COVID-19 indicator and customer price index (CPI). The inclusion of control variables in the model is necessary to isolate the influence of external shocks and macroeconomic dynamics on the studied relationship between demography and the development of the national payment system. The COVID-19 pandemic has become an exogenous event that dramatically changed behavioral patterns. The consumer price index reflects inflationary pressures that adjust household financial strategies: high inflation can reduce confidence in cash, stimulating demand for digital instruments as protection against depreciation.
		
	\end{enumerate}
	
	
	\begin{longtable}
		{|>{\centering\scriptsize}p{10em}
			|>{\centering\scriptsize}p{14em}
			|>{\centering\scriptsize}p{3em}
			|>{\centering\scriptsize\arraybackslash}p{2.3em}|
		}
		
		\caption{Descriptive statistics of data used to build a model of the influence of demographic factors on the share of cashless payments in retail turnover in Russia}\label{tab:data_model}
		\\
		
		\hline
		\textbf{Name} & \textbf{Description} & \textbf{Source} & \textbf{Average}
		\\\hline
		\endfirsthead
		
		\multicolumn{4}{r}{Continuation of the table \ref{tab:data_model}}\\\hline
		\endhead
		
		\multicolumn{4}{r}{End of the table \ref{tab:data_model}}\\\hline
		\endlasthead
		
		\multicolumn{4}{@{} l}{\centering{Source: compiled by the author based on the exploratory data analysis}}
		\endfoot
		
		death\_rate\_village\_lag\_20 & 20-year death rate lag for rural areas, per mill & UIISS &  20.36\\\hline
		
		death\_rate\_village\_lag\_15 & 15-year death rate lag for rural areas, per mill & UIISS &  18.91\\\hline
		
		death\_rate\_village\_lag\_10 & 10-year death rate lag for rural areas, per mill & UIISS & 16.96\\\hline
		
		death\_rate\_village\_lag\_5 & 5-year death rate lag for rural areas, per mill & UIISS &  15.21\\\hline
		
		death\_rate\_village\_lag\_3 & 5-year death rate lag for rural areas, per mill & UIISS &  14.79\\\hline
		
		death\_rate\_city\_lag\_20 & 20-year death rate lag for urban areas, per mill & UIISS &  15.23\\\hline
		
		death\_rate\_city\_lag\_15 & 15-year death rate lag for urban areas, per mill & UIISS &  14.76\\\hline
		
		death\_rate\_city\_lag\_10 & 10-year death rate lag for urban areas, per mill & UIISS &  13.31\\\hline
		
		death\_rate\_city\_lag\_5 & 5-year death rate lag for urban areas, per mill & UIISS &  12.55\\\hline
		
		death\_rate\_city\_lag\_3 & 3-year death rate lag for urban areas, per mill & UIISS &  12.34\\\hline
		
		death\_rate\_all\_lag\_20 & 20-year death rate lag in total, per mill & UIISS &  16.91\\\hline
		
		death\_rate\_all\_lag\_15 & 15-year death rate lag in total, per mill & UIISS &  16.01\\\hline
		
		death\_rate\_all\_lag\_10 & 10-year death rate lag in total, per mill & UIISS &  14.39\\\hline
		
		death\_rate\_all\_lag\_5 & 5-year death rate lag in total, per mill & UIISS &  13.32\\\hline
		
		death\_rate\_all\_lag\_3 & 3-year death rate lag in total, per mill & UIISS &  13.04\\\hline
		
		fertility\_rate\_city\_lag\_20 & 20-year total fertility rate lag for urban areas & UIISS & 1.16\\\hline
		
		fertility\_rate\_city\_lag\_15 & 15-year total fertility rate lag for urban areas & UIISS & 1.29\\\hline
		
		fertility\_rate\_city\_lag\_10 & 10-year total fertility rate lag for urban areas & UIISS & 1.49\\\hline
		
		fertility\_rate\_city\_lag\_5 & 5-year total fertility rate lag for urban areas & UIISS & 1.63\\\hline
		
		fertility\_rate\_city\_lag\_3 & 3-year total fertility rate lag for urban areas & UIISS & 1.60\\\hline
		
		fertility\_rate\_all\_lag\_20 & 20-year total fertility rate lag in total & UIISS & 1.27\\\hline
		
		fertility\_rate\_all\_lag\_15 & 15-year total fertility rate lag in total & UIISS & 1.39\\\hline
		
		fertility\_rate\_all\_lag\_10 & 10-year total fertility rate lag in total & UIISS & 1.63\\\hline
		
		fertility\_rate\_all\_lag\_5 & 5-year total fertility rate lag in total & UIISS & 1.78\\\hline
		
		fertility\_rate\_all\_lag\_3 & 3-year total fertility rate lag in total & UIISS & 1.70\\\hline
		
		fertility\_rate\_all\_lag\_20 & 20-year total fertility rate lag for rural areas & UIISS & 1.57\\\hline
		
		fertility\_rate\_all\_lag\_15 & 15-year total fertility rate lag for rural areas & UIISS & 1.66\\\hline
		
		fertility\_rate\_all\_lag\_10 & 10-year total fertility rate lag for rural areas & UIISS & 2.04\\\hline
		
		fertility\_rate\_all\_lag\_5 & 5-year total fertility rate lag for rural areas & UIISS & 2.23\\\hline
		
		fertility\_rate\_all\_lag\_3 & 3-year total fertility rate lag for rural areas & UIISS & 2.04\\\hline
		
		life\_expectancy\_male\_sex\_village\_lag\_20 & 20-year life expectancy lag for male in rural areas & UIISS & 57.65\\\hline
		
		life\_expectancy\_male\_sex\_village\_lag\_15 & 15-year life expectancy lag for male in rural areas & UIISS & 57.20\\\hline
		
		life\_expectancy\_male\_sex\_village\_lag\_10 & 10-year life expectancy lag for male in rural areas & UIISS & 60.33\\\hline
		
		life\_expectancy\_male\_sex\_village\_lag\_5 & 5-year life expectancy lag for male in rural areas & UIISS & 63.48\\\hline
		
		life\_expectancy\_male\_sex\_village\_lag\_3 & 3-year life expectancy lag for male in rural areas & UIISS & 64.65\\\hline
		
		life\_expectancy\_female\_sex\_village\_lag\_20 & 20-year life expectancy lag for female in rural areas & UIISS & 70.61\\\hline
		
		life\_expectancy\_female\_sex\_village\_lag\_15 & 15-year life expectancy lag for female in rural areas & UIISS & 70.58\\\hline
		
		life\_expectancy\_female\_sex\_village\_lag\_10 & 10-year life expectancy lag for female in rural areas & UIISS & 72.63\\\hline
		
		life\_expectancy\_female\_sex\_village\_lag\_5 & 5-year life expectancy lag for female in rural areas & UIISS & 74.69\\\hline
		
		life\_expectancy\_female\_sex\_village\_lag\_3 & 3-year life expectancy lag for female in rural areas & UIISS & 75.38\\\hline
		
		life\_expectancy\_male\_sex\_city\_lag\_20 & 20-year life expectancy lag for male in urban areas & UIISS & 59.14\\\hline
		
		life\_expectancy\_male\_sex\_city\_lag\_15 & 15-year life expectancy lag for male in urban areas & UIISS & 59.35\\\hline
		
		life\_expectancy\_male\_sex\_city\_lag\_10 & 10-year life expectancy lag for male in urban areas & UIISS & 62.90\\\hline
		
		life\_expectancy\_male\_sex\_city\_lag\_5 & 5-year life expectancy lag for male in urban areas & UIISS & 65.49\\\hline
		
		life\_expectancy\_male\_sex\_city\_lag\_3 & 3-year life expectancy lag for male in urban areas & UIISS & 66.45\\\hline
		
		life\_expectancy\_female\_sex\_city\_lag\_20 & 20-year life expectancy lag for female in urban areas & UIISS & 71.62\\\hline
		
		life\_expectancy\_male\_sex\_city\_lag\_15 & 15-year life expectancy lag for female in urban areas & UIISS & 72.49\\\hline
		
		life\_expectancy\_male\_sex\_city\_lag\_10 & 10-year life expectancy lag for female in urban areas & UIISS & 74.83\\\hline
		
		life\_expectancy\_female\_sex\_city\_lag\_5 & 5-year life expectancy lag for female in urban areas & UIISS & 76.46\\\hline
		
		life\_expectancy\_female\_sex\_city\_lag\_3 & 3-year life expectancy lag for female in urban areas & UIISS & 77.04\\\hline
		
		PopD5a15 & The average annual proportion of the population aged 15 to 19 years & NES RusFMD & 5.13\\\hline
		PopD5a20 & The average annual proportion of the population aged 20 to 24 years & NES RusFMD & 4.86\\\hline
		PopD5a25  & The average annual proportion of the population aged 25 to 29 years & NES RusFMD & 6.28\\\hline
		PopD5a30  & The average annual proportion of the population aged 30 to 34 years & NES RusFMD & 8.36\\\hline
		PopD5a35  & The average annual proportion of the population aged 35 to 39 years & NES RusFMD & 8.18\\\hline
		PopD5a40  & The average annual proportion of the population aged 40 to 44 years & NES RusFMD & 7.43\\\hline
		PopD5a45  & The average annual proportion of the population aged 45 to 49 years & NES RusFMD & 6.72\\\hline
		PopD5a50  & The average annual proportion of the population aged 50 to 54 years & NES RusFMD & 6.27\\\hline
		PopD5a55  & The average annual proportion of the population aged 55 to 59 years & NES RusFMD & 7.05\\\hline
		PopD5a60  & The average annual proportion of the population aged 60 to 64 years & NES RusFMD & 6.93\\\hline
		PopD5a65  & The average annual proportion of the population aged 65 to 69 years & NES RusFMD & 5.60\\\hline
		PopD5a70  & The average annual proportion of the population aged 70 to 74 years & NES RusFMD &  3.39\\\hline
		PopD5a75  & The average annual proportion of the population aged 75 to 79 years & NES RusFMD & 2.12\\\hline
		PopD5a80  & The average annual proportion of the population aged 80 to 84 years & NES RusFMD & 1.91\\\hline
		PopD5a85  & The average annual proportion of the population aged 85 and over & NES RusFMD & 1.16\\\hline
		
		cpi & Customer Price Index & RosStat & 105.95\\\hline
		
		covid & COVID-10 Dummy variables & RosStat & 0.5025\\\hline
	\end{longtable}


	Lag transformations of 3, 5, 10, 15 and 20 years were applied for each demographic variable, which are due to a significant latency period. Demographic changes are also shaping long-term trends in demand for financial services. Ignoring lags leads to underestimation of long-term effects, such as the aging of the population and its pressure on the national payment system. Lag transformations were applied to each demographic variable due to a significant delay period. Demographic changes also shape long-term trends in demand for financial services. Ignoring delays leads to an underestimation of long-term effects such as the aging population and its impact on the national payment system. From an econometric perspective, demographic variables often correlate with past values. For instance, the birth rate in year $t$ is influenced by the birth rate of the previous year $(t-1)$ due to cultural and economic factors. A lag-free model, however, violates the assumption of uncorrelated errors ($Cov (\epsilon_{t}, \epsilon_{t-1} ) \neq 0$). Values for rural and urban areas have also been added for mortality and fertility.
	
	
	Based on the assumptions and selected factors, the specification of the econometric model for assessing the impact of demographic factors on the share of cashless payments in retail turnover is as follows:
	
	
	\begin{equation}
		Y_{it} = \beta_0 + \beta_1 DR_{i,t-l_{j}} + \beta_2 FR_{i,t-l_{j}} + \beta_3 LE_{i,t-l_{j}} + \beta_4 AS_{i,t_{j}} + \beta_5 CV_{i,t_{j}} + \mu_{i} + \varepsilon_{it},
		\label{eq:model}
	\end{equation}
	
	where $Y_{it}$ -- percentage of cashless payments in the region $i$ at time $t$;
	\\
	$DR_{it-l_{j}}$ -- death rate factor $j$ with lag $l$ in the region $i$ at time $t$;
	\\
	$FR_{it-l_{j}}$ -- total fertility rate factor $j$ with lag $l$ in the region $i$ at time $t$;
	\\
	$LE_{it-l_{j}}$ -- life expectancy factor $j$ with lag $l$ in the region $i$ at time $t$;
	\\
	$AS_{it-l_{j}}$ -- age structure factor $j$ in the region $i$ at time $t$;
	\\
	$CV_{i,t_{j}}$ -- control variable $j$ in the region $i$ at time $t$;
	\\
	$\mu_{i}$ -- individual effects;
	\\
	$\varepsilon_{i t}$ -- model error with $E(\varepsilon_{i t} | DR_{it-l_{j}}, ..., CV_{i,t_{j}}) = 0$;
	\\
	$\beta_0$, $\beta_1$, $\beta_2$, $\beta_3$, $\beta_4$, $\beta_5$ -- estimated coefficients of the model.
	
	
	This model specification is called a <<fixed effects model>> (FE), which implies that individual effects correlate with independent variables. It is important to note that the model takes into account the relationship between regressor values that relate to the same object but at different times.  Before modeling, the data was tested for multicollinearity using the Variance Inflation Factor (VIF) Test. Thus, due to the high VIF rates, some demographic factors were excluded from the model. The coefficients obtained with their significance, obtained using the <<within estimator>>, can be seen in Table \ref{tab:results} (The model construction code can be seen in Appendix \ref{appendix:fe_model}).
	
	
	\begin{table}[!htbp]
		\centering
		\small
		\caption{Panel Regression Results: Cashless Payments Determinants}
		\label{tab:results}
		
		\begin{tabular}{@{}lclc@{}}
			\toprule
			\textbf{Dep. Variable:}          & value        & \textbf{R-squared:}          & 0.9684      \\
			\textbf{Estimator:}              & PanelOLS     & \textbf{R-squared (Between):} & -6.8384     \\
			\textbf{No. Observations:}       & 549          & \textbf{R-squared (Within):}  & 0.9684      \\
			\textbf{Date:}                   & Sat, May 17 2025 & \textbf{R-squared (Overall):} & -6.0082     \\
			\textbf{Time:}                   & 16:29:57     & \textbf{Log-likelihood}      & -1024.2     \\
			\textbf{Cov. Estimator:}         & Clustered    & \textbf{F-statistic:}        & 784.03      \\
			\textbf{Entities:}               & 71           & \textbf{P-value}             & 0.0000      \\
			\textbf{Avg Obs:}                & 7.7324       & \textbf{Distribution:}       & F(18,460)   \\
			\textbf{Min Obs:}                & 1       & \textbf{F-statistic (robust):} & 2.537e+04  \\
			\textbf{Max Obs:}                & 96       & \textbf{P-value}             & 0.0000      \\
			\textbf{Time periods:}           & 6            & \textbf{Distribution:}       & F(18,460)   \\
			\bottomrule
		\end{tabular}
		
		\vspace{1em}
		
		\begin{tabular}{@{}lcccc@{}}
			\toprule
			& \textbf{Parameter} & \textbf{Std.Err.} & \textbf{t-stat} & \textbf{P-value} \\
			\midrule
			\textbf{Demographic variables} \\
			PopD5a30 & -0.5959 & 1.3591 & -0.4384 & 0.6613 \\
			PopD5a35 & 8.3961*** & 1.8122 & 4.6331 & 0.0000\\
			PopD5a45 & 2.9995* & 1.5407 & 1.9469 & 0.0522 \\
			PopD5a50 & -0.6355 & 0.9094 & -0.6988 & 0.4850 \\
			PopD5a55 & -1.5335 & 1.4530 & -1.0554 & 0.2918 \\
			PopD5a75 & -5.8966*** & 1.2400 & -4.7552 & 0.0000 \\
			death\_rate\_village\_lag\_3 & -0.0307 & 0.0232 & -1.3256 & 0.1856\\
			death\_rate\_city\_lag\_15 & 0.0087 & 0.0125 & 0.7020 & 0.483 \\
			fertility\_rate\_city\_lag\_3 & 4.5592** & 2.1926 & 2.0794 & 0.0381  \\
			fertility\_rate\_village\_lag\_20 & -0.0104 & 0.1188 & -0.0876 & 0.9303  \\
			fertility\_rate\_village\_lag\_10 & -0.0418 & 0.1025 & -0.4079 & 0.6835  \\
			fertility\_rate\_village\_lag\_3 & -3.3424*** & 1.2643 & -2.6438 & 0.0085 \\
			life\_expectancy\_female\_sex\_village\_lag\_5 & 0.7863*** & 0.2367 & 3.3217 & 0.0010 \\
			life\_expectancy\_male\_sex\_city\_lag\_10 & 0.0033 & 0.0098 & 0.3331 & 0.7392 \\
			life\_expectancy\_male\_sex\_village\_lag\_20 & -0.0078 & 0.0125 & -0.6218 & 0.5344 \\
			\textbf{Control variables} \\
			life\_expectancy\_female\_sex\_city\_lag\_5 & 0.8192* & 0.4607 & 1.7781 & 0.0760 \\
			cpi & 0.1353 & 0.0922 & 1.4679 & 0.1428 \\
			grp & 3.895e-07*** & 1.358e-07 & 2.8670 & 0.0043  \\
			\bottomrule
		\end{tabular}
		
		\begin{tablenotes}
			\small
			\item Notes: *** p<0.01, ** p<0.05, * p<0.1.
			\item F-test for Poolability: 71.668 (P-value: 0.0000, Distribution: F(70,460))
			\item Included effects: Entity
			\item Source: compiled by the author based on the econometric modeling with Statsmodels\footnote{https://www.statsmodels.org/stable/index.html}
		\end{tablenotes}
	\end{table}
	

	Before proceeding to the interpretation of the results, it is necessary to check the robustness of the model. The Hausman test is used to verify the robustness of the choice of the fixed effects model relative to the random effects model (RE). The null hypothesis assumes that random effects are correct ($Cov(x_{it}, \mu_{i})=0$), and the alternative hypothesis states that there is a correlation between regressors and individual effects, which makes the FE model more suitable. The test statistics are defined as:
	
	\begin{equation}
		H = (\hat{\beta}_{\text{FE}} - \hat{\beta}_{\text{RE}})^\top 
		\left[ \text{Var}(\hat{\beta}_{\text{FE}}) - \text{Var}(\hat{\beta}_{\text{RE}}) \right]^{-1} 
		(\hat{\beta}_{\text{FE}} - \hat{\beta}_{\text{RE}}),
	\end{equation}
	
	where $\hat{\beta}_{\text{FE}}$ and $\hat{\beta}_{\text{RE}}$ -- coefficient estimates for FE and RE models.	
	
	In this case, the Hausman test showed significance ($p-value\approx0.000$), which allows to reject the hypothesis about the correctness of the random effects model. This indicates that individual invariant characteristics ($\mu_{i}$) correlate with explanatory variables, and using the FE model minimizes the risk of bias caused by an omitted variable.
	
	
	Analyzing Table \ref{tab:results}, there are some findings (Figure \ref{fig:marginal_effects}):
	
	\begin{itemize}
		\item The fixed-effects model shows a high explanatory power (R-squared = 0.9684), indicating that 96\% of the variation in the share of cashless payments in Russian regions can be explained by the demographic variables included with lags and macroeconomic factors;
		\item The $PopD5a35$ coefficient of $8.3961$ meants that an increase in the population aged 35 to 39 by 1 percentage point increases the share of non-cash payments by $8.3$ percentage points. This may be due to the active use of digital technologies by this age group;
		\item The $PopD5a75$ coefficient of $-5.8966$ means that an increase in the population aged 75 to 79 by 1 percentage point decreases the share of non-cash payments by $5.8$ percentage points. Older people are more likely to remain committed to cash payments;
		\item Coefficient \textit{fertility\_rate\_city\_lag\_3} ($4.5592$) indicates that an increase in the urban birth rate 3 years ago by 1 unit increases the current share of non-cash payments by ~4.6\%. This reflects the contribution of young people entering economic activity and using digital payments;
		\item Coefficient \textit{fertility\_rate\_village\_lag\_3} ($-3.342$) indicates that the high birth rate in rural areas 3 years ago reduces the share of non-cash payments. This is probably due to the migration of young people to cities, which leaves the older generation in the villages who prefer cash;
		\item The coefficient \textit{life\_expectancy\_female\_sex\_village\_lag\_5} (0.7863) demonstrates that the increase in life expectancy of women in rural areas 5 years ago by 1 year increases the share of non-cash payments by 0.79\%. This may be due to their long-term economic activity and adaptation to technology;
		\item Control variable \textit{grp} ($3.895e-07$) significant: The growth of the gross regional product by 1 unit increases the share of non-cash payments. This highlights the role of economic development in digitalization;
		\item Variables \textit{death\_rate\_city\_lag\_15} ($p=0.483$), \textit{fertility\_rate\_village\_lag\_10/20} ($p > 0.1$) и \textit{life\_expectancy\_male\_sex\_city\_lag\_10} ($p=0.739$) statistically insignificant. This indicates that there is no long-term impact of urban mortality and rural fertility with large lags on the share of non-cash payments.
		
	\end{itemize}
	
	
	\begin{figure}[H]
		\centering
		\noindent\includegraphics[width=0.9\linewidth]{margin_effects.png}
		\caption{Marginal effects with its confidence intervals of factors with p\_value > 0.05 for FE model\\Source: compiled by the author based on econometric modeling}
		\label{fig:marginal_effects}
	\end{figure}
	
	
	Summarizing the results of econometric modeling, there are several conclusions:
	
	\begin{enumerate}
		\item Demographic factors with certain lags significantly affect the share of cashless payments. Fertility and female life expectancy are particularly important;
		\item Age structure factors for young and older generations significantly affect the share of cashless payments and results are consistent with global trends, where young and middle-aged groups are the drivers of digitalization, and the elderly are conservative users;
		\item Cities, unlike villages, show a stronger connection between demographics and digitalization of payments;
		\item Macroeconomic factors reinforce the trends. Inflation and gross region product are accelerating the transition to cashless payments, which is consistent with global trends.
	\end{enumerate}
	
	
	\newpage
	
	\chapter{Improving the development strategy of the national payment system of Russia}
	
	\section{Development of the spatial development strategy of the National Payment System of Russia using the constructed index}
	
	
	In the strategic planning document <<Main Directions of Development of the National Payment System>>, the Bank of Russia can use an index developed in Chapter 2.2 to monitor the development of the National Payment System at the regional level. This index can be applied in Chapter 5 (<<INDICATORS AND OBSERVED INDICATORS OF THE MAIN DIRECTIONS OF NPS DEVELOPMENT>>), as one of the indicators for assessing the implementation of NPS MDD. The model developed in Chapter 2.3 can also be used to predict long-term and medium-term trends in strategic planning. 
	
	
	To calculate the development index of the National Payment System at the regional level, we used a traditional approach based on the Urban Health Index. This method involves several steps:
	
	\begin{enumerate}
		\item Data collection and preparation: for this purpose, it is necessary to use information published by the Bank of Russia, which includes data by region. These data are publicly available quarterly on the official website of the Bank of Russia;
		\item Factors selection: to create an index, it is necessary to determine a limited number of indicators that would reflect the level of development of the national payment system in the regions. The following indicators are proposed to be considered as factors that reflect the development of the National Payment System of Russia:
		\begin{itemize}
			\item Institutional provision of payment services in the regional context;
			\item The number of accounts opened by institutions of the banking system, also in the regional context;
			\item The average daily, maximum, and minimum number and volume of money transfers from credit institutions made through the Bank of Russia payment system, also regionally;
			\item The number of payments made through credit institutions (for various payment instruments), by region;
			\item The volume of payments made through credit institutions (using the same payment instruments), also regionally;
			\item The number of remote access accounts opened with credit institutions, by region;
			\item The volume of payments, orders for which were drawn up and transmitted electronically by customers of credit institutions and by the credit institution itself, also in the regional context., The number of payments made by customers of credit institutions using payment orders, by various receipt methods and in the regional context;
		\end{itemize}
		
		\item Before calculating the index, it is recommended to perform a correlation analysis using the Spearman's rank correlation in order to identify factors that are not correlated;
		
		\item Calculation of the index based on standardization and geometric mean. As mentioned in the second chapter, the differences between the regions are too great. To smooth them out, there was apply some statistical transformations before calculating the index. For example, before standardization, each variable can be logarithmic, which will help reduce variance;
		
		\item Application of the rating scale to obtain the category of development of the National Payment System of Russia.
	\end{enumerate}
	
	A percentile-based approach will be used to calculate the rating scale by Koenker and Hallock\footnote{Koenker, Roger, and Kevin F. Hallock. “Quantile Regression.” The Journal of Economic Perspectives 15, no. 4 (2001): 143–56. http://www.jstor.org/stable/2696522.}. Unlike parametric methods, deciles are more robust to outliers and skewed distributions, making them a suitable choice for Russia regionally diverse scale (0-1). This is in line with the empirical findings, which identified age and urbanization as key factors influencing payment behavior heterogeneity. The rating calculation algorithm is shown below:
	
	
	\begin{enumerate}
		\item \textit{Ranking}: Rank regions by index in ascending order:
		
		
		$R_{i} = rank (x_{i}), i=1, 2, ..., 89$, where $x_{i}$ -- index for region $i$;
		
		
		\item \textit{Percentile Calculation}: Compute percentile ranks using the Hazen formula\footnote{Hyndman, Rob J., and Yanan Fan. “Sample Quantiles in Statistical Packages.” The American Statistician 50, no. 4 (1996): 361–65. https://doi.org/10.2307/2684934.
		} to mitigate bias in small samples:
		
		
		$P_{i} = \frac{R_{i} - 0.5}{89} * 100$
		
		
		This adjusts for discrete ranking effects, ensuring smoother transitions between deciles;
		
		
		\item \textit{Decile Assignment}: Assign regions to deciles based on percentile thresholds:
		
		
		$\text { Decile }_k=\left\lfloor\frac{P_i}{10}\right\rfloor+1, \quad k=1,2, \ldots, 10$	
		
		\item \textit{Handling Ties}: Apply average ranking for tied index values:
		
		$R_{tied} = \frac{\sum_{j=1}^{m}R_{j}}{m}$, where $m$ -- the number of tied regions.
	\end{enumerate}
	
	
	Using the algorithm described above, we have compiled ratings for all regions of Russia (Table \ref{tab:rating}). The breakdown into ratings by index was made taking into account the principle of granularity, that is, each rating should not contain too many regions. A method based on the use of traffic lights was also used, which is an effective way to visually present information. It makes it possible to quickly identify areas where there are risks, where the situation is stable and where progress is being made.
	
	
	\begin{table}
		\caption{Rating scale}\label{tab:rating}
		\centering
		\begin{tabular}%{\linewidth}
			{|c|c|}\hline
			\textbf{Index} & \textbf{Rating} \\\hline
			0 - 0.331 & 1\\\hline
			0.334 - 0.386 & 2\\\hline
			0.388 - 0.419 & 3\\\hline
			0.424 - 0.439 & 4\\\hline
			0.445 - 0.468 & 5\\\hline
			0.469 - 0.482 & 6\\\hline
			0.483 - 0.495 & 7\\\hline
			0.504 - 0.533 & 8\\\hline
			0.534 - 0.578 & 9\\\hline
			0.580 - 1 & 10\\\hline
		\end{tabular}\\
		Source: compiled by the author based on the rating approach
	\end{table}
	
	To implement the traffic light approach, a percentile scale based on the decile calculation was used. The regions were ranked by the value of the Nation Paymet System Index and divided into 10 equal groups (deciles), where the 1st decile corresponds to the lowest values and the 10th to the highest. The deciles were further grouped into three categories:
	
	\begin{enumerate}
		\item Green Zone (8-10 deciles): leading regions where the level of digitalization of payments meets or exceeds national standards;
		\item Yellow zone (4-7 deciles): regions with average indicators that demonstrate potential for growth, but require targeted support measures;
		\item Red zone (1-3 deciles ): regions with a critically low level of NPC development, where the share of non-cash transactions does not exceed 35\%, and the infrastructure is limited.
	\end{enumerate}
	
	
	The choice of thresholds is justified by the analysis of the index distribution, which revealed a significant gap between the top regions and outsiders. For example, Moscow is 11.5 times ahead of the Republic of Ingushetia, which confirms the need for a differentiated approach.
	
	\begin{table}
		\caption{Traffic Light Rule}\label{tab:tl_rule}
		\centering
		\begin{tabular}%{\linewidth}
			{|c|c|c|}\hline
			\textbf{Zone} & \textbf{Index} & \textbf{Decile} \\\hline
			
			Green & 0.504 - 1 & 8 - 10\\\hline
			
			Yellow & 0.424 - 0.495 & 4 - 7\\\hline
			
			Red & 0 - 0.419 & 1 - 3\\\hline
			
		\end{tabular}\\
		Source: compiled by the author based on the rating approach
	\end{table}
	
	
	Based on calculations (Table \ref{tab:tl_rule}), the following groups are identified (\ref{fig:map}):
	
	\begin{itemize}
		\item \textit{Green zone (8-10 deciles)}:
		This zone covers 23 regions, including Moscow, St. Petersburg, Voronezh, Novosibirsk and Sverdlovsk regions, as well as the Republic of Tatarstan and others. These regions are different:
		\begin{itemize}
			\item High density of bank branches -- an average of 4.3 accounts per resident, which is significantly more than 1.2 in the red zone;
			\item Active technology adoption — the share of electronic payments reaches 85\%;
			\item The concentration of the young population (the average age is 36 years), which is consistent with the conclusions of Chapter 2.3 on the impact of demography on digitalization.
		\end{itemize}
		
		\item \textit{Yellow zone (4-7 deciles)}:
		This zone includes 31 regions, including Sakhalin, Astrakhan and Kaluga regions:
		\begin{itemize}
			\item People in these regions have an average of 2.1 accounts per person;
			\item The majority of residents pay in cash — from 45\% to 55\%;
			\item Young people are moving to big cities, and this creates problems for the local payment system.
		\end{itemize}
		
			\item \textit{Red zone (1-3 deciles)}:
			It includes 24 regions, such as the Republic of Ingushetia, Dagestan, Chechnya, Kalmykia, and the Jewish Autonomous Region. Their features:
			\begin{itemize}
				\item Minimal infrastructure (less than 0.5 bank branches per 10,000 inhabitants);
				\item Dominance of cash payments (up to 70\% of transactions);
				\item\ A high proportion of the rural population (60-80\%) and senior citizens (25\% over 60 years old).
			\end{itemize}
	\end{itemize}
	
	\begin{figure}[H]
		\centering
		\noindent\includegraphics[width=0.99\linewidth]{map.png}
		\caption{Map of Russian regions with traffic light rule reference\\Source: compiled by the author}
		\label{fig:map}
	\end{figure}
	

	
	Thus, according to the constructed map, several zones of regional instability can be identified. These zones include the southern regions of Russia, the northwestern territories (excluding St. Petersburg), the southern Siberian regions, and Chukotka. This areas in the red category indicate a need for increased focus on establishing bank branches in rural areas and organizing financial literacy programs for older adults. Furthermore, it is essential to evaluate the effectiveness of ongoing reforms, such as the introduction of digital rubles in certain regions like Moscow and Tatarstan. These initiatives should be closely monitored to assess their impact on the National Payment System Development index.
	
	
	The use of the traffic light method has confirmed the high level of differentiation in the development of the National Payment System in Russia, as identified in chapters 1 and 2. These results emphasize the need for targeted policies that consider the demographic, infrastructure, and economic characteristics of each region. 
	
	
	In order to better understand the differences in the development of Russia regions and to propose effective support measures, it is necessary to refer to the Spatial Development Strategy of the Russian Federation\footnote{Распоряжение Правительства Российской Федерации от 28.12.2024 г. № 4146-р}. This strategy, approved by the government of the Russian federation, aims to create a sustainable settlement system and a territorial organization of the economy that will take into account national goals and objectives, such as reducing inter-regional differences in standard and quality of life, accelerating economic growth, and ensuring national security, as well as technological development.
	
	
	As part of the strategy aimed at developing the National Payment System, special attention is paid to territories with a low level of socio-economic development and low population density. According to our research, these regions, which are included in the red zone, require priority attention. In particular, the strategy proposes to develop strong settlements, improve infrastructure, and create conditions for attracting investment and improving quality of life. Additionally, the strategy identifies promising centers for economic growth - large urban agglomerations and scientific and educational centers - that serve as the foundation for innovative development and digitalization. These regions, such as Moscow, St Petersburg, Tatarstan, and others, are part of the green zone, have well-developed infrastructure with a high density of bank branches, and actively introduce digital payment technologies.
	
	
	The strategy also emphasizes the importance of coordination between federal, regional, and municipal authorities when implementing development programs. This enables an individualized approach to supporting regions with varying levels of digital payment infrastructure.
	
	
	It is recommended that event planning take into account demographic, economic, and infrastructural factors in each region. These considerations are fully consistent with the findings of the analysis. Thus, combining NPS ratings with spatial development strategies opens up opportunities to form a targeted policy aiming to reduce regional digital divides, increase access to payment services, and foster sustainable economic growth across Russia.
	
	
	
	\section{Implementing the model of demographic factors impact on the share of cashless transactions in strategic planning for the National Payment System of Russia}
	
	
	The developed econometric model, which confirms the significance of demographic factors in the dynamics of cashless payments, is a valuable tool for strategic planning. Integrating this model into the documents of the Bank of Russia, such as <<Main directions for the development of the national payment system for 2025-2027>> and <<Financial Market Development Strategy>>, will enhance the evidence base for regulatory decisions and tailor support measures to address demographic challenges. the framework of the <<Main directions of NPS development until 2027>>, the model can be used for:
	
	\begin{enumerate}
		\item Forecasting the share of non-cash transactions in the medium term, taking into account demographic trends. For example, a decrease in the birth rate in rural areas (coefficient -3.34) signals the need to strengthen digitalization measures in these regions;
		\item Evaluating the effectiveness of current initiatives, such as the introduction of the digital ruble or the development of QR acquiring. The model allows us to quantify how changes in the age structure of the population (for example, an increase in the proportion of people over 75) affect the achievement of KPIs;
		\item Adaptation of regional programs. For example, in the <<red>> zones (the Republic of Ingushetia, Dagestan), the focus should shift to infrastructure projects and educational campaigns for the elderly, while in the <<green>> zones (Moscow, St. Petersburg) - to innovations (biometrics, Open Banking).
	\end{enumerate}
	
	
	According to Rosstat's average forecast option\footnote{https://rosstat.gov.ru/folder/313/document/72529}, by 2035, for children and adolescents aged 15 and younger, it will decrease from 18.7\% to 14.2\%, and for those aged 16-64, it will increase from 65.8\% to 65.1\%, and the life expectancy will increase to 79,1 years. This creates two key challenges for the NPC:
	
	\begin{enumerate}
		\item A decrease in the share of the <<digital>> population: the 35-39-year-old generation, which actively uses cashless payments (coefficient +8.39), will gradually be replaced by older age groups who prefer cash;
		\item Growing regional inequality: Urbanization will increase the outflow of young people from villages, where the fertility rate negatively correlates with cashlessness (-3.34).
	\end{enumerate}
	
	
	The model allows to simulate these scenarios. Thus, demographic coefficients can be introduced in the allocation of financing for infrastructure projects; age-oriented products can be developed, such as simplified mobile banking interfaces for the elderly; regional differentiation of strategies can be taken into account.
	
	
	
	
	
	
	
	
	\newpage
	%%%%%%% Conclusion %%%%%%%
	\chapter*{Conclusion}
	\addcontentsline{toc}{chapter}{Conclusion}
	
	The study presented here addresses the question of how demographic factors affect the development of Russia's national payment system. It shows that the trend in system development is an increase in non-cash payments as a percentage of overall retail turnover in Russia. However, there are also significant regional variations in both the share of non-cash transactions and the development of payment systems. The study confirms that demographic changes such as population aging, variations in mortality, and fertility have an impact on the Russian payment system.
	
	
	The study consisted of several main stages. In the first stage, a regulatory analysis of current legislation in the field of payment systems was conducted, as well as an analysis of theories describing how demographic factors influence the development of national payment systems. An econometric analysis and other empirical studies were also carried out. Methods for modeling the impact of demographic factors on non-cash payment shares were systematized. In the second stage, an econometric analysis was conducted and an index for measuring the development of the national payment system at the regional level was created. In the final stage, specific recommendations were proposed for using the constructed econometric model and index to improve the national payment system.
	
	
	The results of the review of empirical works are similar to theoretical concepts. In their works, the authors explore demographic factors influencing the development of the national payment system. The share of non-cash payments in retail turnover is more often accepted as the development of the national payment system. As a result of the analysis, it was revealed that when building an econometric model of the influence of demographic factors on the share of non-cash payments, it is necessary to include indicators of life expectancy, fertility, mortality, age structure of the population and macroeconomic factors as control variables.
	
	Based on the factors described above, an econometric model of the influence of demographic factors on the share of non-cash payments in retail turnover by regions of Russia was built. The study used a fixed-effects model. The model has been tested for robustness. Thanks to the model, the hypotheses put forward at the beginning of the study were confirmed:
	
	
	\begin{itemize}
		\item \textbf{Hypothesis 1:} the increasing share of the older generation in the age structure negatively affects the share of non-cash payments in retail turnover;
		\item \textbf{Hypothesis 2}: an increase in the birth rate in urban areas has a positive effect on the share of non-cash payments, while in rural areas it has a negative effect.
	\end{itemize}

	
	Additionally, an index of the development of the national payment system of Russia at the regional level was built. After analyzing the regulatory framework, development factors were identified: institutional provision of payment services, the number of accounts opened by institutions of the banking system, the number and volume of payments, orders for which were drawn up and transmitted electronically by customers. This index is interpretable and compiled for all regions of Russia. Based on the index obtained, a rating of the regions on the development of the national payment system was built. Based on the rating obtained, a traffic light method was proposed that allows regions to be classified into <<red>>, <<yellow>> or <<green>> zones, depending on the degree of their development. Falling into one of the three zones allows you to classify regions and combine them into zones. The application of the developed approach will allow the Bank of Russia to improve the mechanisms of Russia spatial development strategy.
	
	
	The developed econometric model can help the Bank of Russia in building a strategy for the development of the National Payment System, namely, it is possible to use this model as one of the tools for strategic forecasting using demographic forecasting. The resulting forecast of the share of non-cash payments can be taken into account in strategic planning, in particular, when introducing new payment instruments such as the digital ruble. 
	
	
	Further research may be aimed at improving the construction of an econometric model, namely the use of microeconomic data and population surveys. Besides the data at the regional level, you can also use municipal data. In addition to the main demographic factors used in the model, alternative indicators can also be used, such as population migration, financial literacy of the population, and Internet coverage. Scenario-based approach can also be used with stress testing for strategic planning. 
	
	
	
	
	%%%%%%%%%%% References	%%%%%%%%%%%


	\newpage
	\addcontentsline{toc}{chapter}{References}
	\titleformat*{\section}{\bfseries\normalsize\fontsize{14}{2.5mm}\centering}
	\begin{thebibliography}{3}
		
		
		\bibitem{cental_bank_russia_natiional_fin_serv} Bank of the Russia. The main directions of increasing the availability of financial services in the Russian Federation for the period 2025-2027. // https://www.cbr.ru/Content/Document/File/170684/onpdfu  \_2025-2027.pdf
		
		\bibitem{cental_bank_russia_natiional_payment_dir} Bank of the Russia. The main directions of development of the national payment system for the period 2025-2027. // https://www.cbr.ru/Content/Document/File/170680/onrnps  \_2025-27.pdf
		
		\bibitem{labour_min} Decree of the Government of the Russian Federation on the approval of the Strategy of Actions in the Interests of Older Citizens in the Russian Federation until 2030 // https://mintrud.gov.ru/ministry/programms/12   
		
		\bibitem{decree2} Decree of the Government of the Russian Federation dated December 28, 2024 No. 4146-r\\
		http://publication.pravo.gov.ru/document/0001202501060001
		
		\bibitem{tishin} Тишин, А. (2020). Влияние демографии на развитие финансового сектора Российской Федерации. Аналитическая записка Департамента исследований и прогнозирования, Банк России.
		
		\bibitem{Ando} Ando, A., \& Modigliani, F. (1963). The “Life Cycle” Hypothesis of Saving: Aggregate Implications and Tests. The American Economic Review, 53(1), 55–84. http://www.jstor.org/stable/1817129
		
		\bibitem{arkhipova} Arkhipova, N. (2022). Impact of Demographic Trends on Retail Banking. Procedia Computer Science, 214, 831-836.
		
		\bibitem{Aygdogan} Aydogan, S., \& Van Hove, L. (2015). Nudging consumers towards card payments: A field experiment. In International Cash Conference 2014 (pp. 589-630). Deutsche Bundesbank
		
		\bibitem{becker} Becker, G. S. (1962). Investment in human capital: A theoretical analysis. Journal of political economy, 70(5, Part 2), 9-49.
		
		\bibitem{bloom} Bloom, D. E., \& Williamson, J. G. (1998). Demographic transitions and economic miracles in emerging Asia. The World Bank Economic Review, 12(3), 419-455.
		
		\bibitem{Camilleri} Camilleri, S. J., \& Agius, C. (2021). Choosing between innovative and traditional payment systems: an empirical analysis of European trends. Journal of Innovation Management, 9(4), 29-57.
		
		\bibitem{Calvo-Gonzalez} Calvo-Gonzalez, O., Cruz, A., \& Hernandez, M. (2018). The Ongoing Impact of ‘Nudging’ People to Pay Their Taxes. World Bank Blogs, 2
		
		\bibitem{chawla} Chawla, D., \& Joshi, H. (2018). The moderating effect of demographic variables on mobile banking adoption: An empirical investigation. Global Business Review, 19(3\_suppl), S90-S113.
		
		\bibitem{coase} Coase, R. H. (1937). ”The Nature of the Firm”. Economica. 4 (16): 386–405
		
		\bibitem{Finkelstein} Finkelstein, M., \& Fishman, R. (2019). Demographics and Innovation: Evidence from Patent Data
		
		\bibitem{galor} Galor, O., \& Weil, D. N. (2000). Population, technology, and growth: From Malthusian stagnation to the demographic transition and beyond. American economic review, 90(4), 806-828.
		
		\bibitem{Greene} Greene, C., Perry, J., \& Stavins, J. (2024). Consumer Payment Behavior by Income and Demographics.

		
		\bibitem{Graziano} Graziano, E. A., Musella, F., \& Petroccione, G. (2024). Cashless payment: behavior changes and gender dynamics during the COVID-19 pandemic. EuroMed Journal of Business, 20(5), 54-74.
		
		\bibitem{hyndman} Hyndman, Rob J., and Yanan Fan. “Sample Quantiles in Statistical Packages.” The American Statistician 50, no. 4 (1996): 361–65. https://doi.org/10.2307/2684934.
		
		\bibitem{kyc} Mullins, R. R., Ahearne, M., Lam, S. K., Hall, Z. R., \& Boichuk, J. P. (2014). Know your customer: How salesperson perceptions of customer relationship quality form and influence account profitability. Journal of Marketing, 78(6), 38-58.
		
		\bibitem{ketkaew} Ketkaew, Chavis, Martine Van Wouwe, Preecha Vichitthamaros, and Duanpen Teerawanviwat. "The effect of expected income on wealth accumulation and retirement contribution of Thai wageworkers." SAGE Open 9, no. 4 (2019): 2158244019898247.
		
		\bibitem{Kotlikoff} Kotlikoff, L. J. (1989). What determines savings?. MIT Press Books, 1.
		
		\bibitem{Malta} Malta, P., Ma, W., \& Zheng, H. (2024). Mobile payment adoption in China: Do demographic and socioeconomic factors matter?. Managerial and Decision Economics, 45(3), 1428-1434.
		
		\bibitem{Modigliani_brumberg} Modigliani, F., \& Brumberg, R. (1954). Utility analysis and the consumption function: An interpretation of cross-section data. Franco Modigliani, 1(1), 388-436.
		
		\bibitem{modigliani_1966} Modigliani, F. (1966). The life cycle hypothesis of saving, the demand for wealth and the supply of capital. Social research, 160-217.
		
		\bibitem{modigliani_1970} Modigliani, F. (1970). The life cycle hypothesis of saving and intercountry differences in the saving ratio (pp. 197-225). WA Eltis, M. FG. Scott and JN Wolfe, eds., Induction, trade, and growth: Essays in honour of Sir Roy Harrod (Clarendon Press, Oxford).
		
		\bibitem{nudge2} Story, P., Smullen, D., Acquisti, A., Cranor, L. F., Sadeh, N., \& Schaub, F. (2020). From intent to action: Nudging users towards secure mobile payments. In Sixteenth Symposium on Usable Privacy and Security (SOUPS 2020) (pp. 379-415.
		
		\bibitem{Oldenburg} Oldenburg, Brian, and Karen Glanz. "Diffusion of innovations." Health behavior and health education: Theory, research, and practice 4 (2008): 313-333.
		
		\bibitem{rogers} Rogers, E. M., Singhal, A., \& Quinlan, M. M. (2014). Diffusion of innovations. In An integrated approach to communication theory and research (pp. 432-448). Routledge.
		
		\bibitem{simon} Simon, H. A. (1955). A behavioral model of rational choice. The quarterly journal of economics, 99-118.
		
		\bibitem{taleb} Taleb, N.N. (2007) The Black Swan: The Impact of the Highly Improbable. Random House, New York
		
		\bibitem{taler} Thaler, R. H. (2015). Misbehaving: The making of behavioral Economics. WW Norton \& Company.
		
		\bibitem{tversky} Tversky, A., \& Kahneman, D. (1992). Advances in prospect theory: Cumulative representation of uncertainty. Journal of Risk and uncertainty, 5, 297-323.
		
		\bibitem{Vaportzis} Vaportzis, E., Giatsi Clausen, M., \& Gow, A. J. (2017). Older adults perceptions of technology and barriers to interacting with tablet computers: a focus group study. Frontiers in psychology, 8, 1687.
		
		\bibitem{Venkatesh} Venkatesh, V., Morris, M. G., Davis, G. B., \& Davis, F. D. (2003). User acceptance of information technology: Toward a unified view. MIS quarterly, 425-478.
		
		\bibitem{Williamson} Williamson, O. E. (1979). Transaction-cost economics: the governance of contractual relations. The journal of Law and Economics, 22(2), 233-261.
		
		\bibitem{wooldridge} Wooldridge, J. M. 1995. “Selection Corrections for Panel Data Models Under Conditional Mean Independence Assumptions.” Journal of Econometrics 68 (1): 115–132.
		
		\bibitem{stavins} Stavins, J. (2016). The effect of demographics on payment behavior: panel data with sample selection (No. 16-5). Working Papers.
		
		\bibitem{ey} EY Report: How Gen Z’s preference for digital is changing the payments landscape (2024) // https://www.ey.com/en  \_us/insights/payments/how-gen-z-is-changing-the-payments-landscape
		
		\bibitem{cental_bank_russia_natiional_statistics} Bank of Russia. National Payment System statistics (2025). Date of the application 01.04.2025\\
		https://www.cbr.ru/PSystem/ 
			
		
		
	\end{thebibliography}
	
	
	\appendix
	
	% Стиль заголовков для приложений
	\titleformat{\chapter}
	{\normalfont\bfseries\large}{\chaptertitlename~\thechapter}{0.25em}{\normalfont}
	
	% Стиль для содержания приложений
	\titlecontents{chapter}
	[0em] %
	{\normalsize}
	{\makebox[7em][l]{Appendix \thecontentslabel}}
	{Appendix}
	{\titlerule*[10pt]{.}\contentspage}
	
	
	\chapter{Index  Creation Python Code}
	
	\label{appendix:index}
	
	\begin{lstlisting}[language=Python, caption={Index Construction}]
		
		
		
		corr = df.select_dtypes("float").rename(columns={
			"bank orders": "volume bank orders",
			"The number of accounts opened by institutions of the banking system, per 1 resident, units.": "accounts per 1",
			"Number of bank institutions per 1 million inhabitants": "bank_isnt_per_1",
			"Total number of banking system institutions": "total_bank",
			"volume orders in electronic form with internet": "vol_internet",
			"payment requirements with internet": "pay_internet"
		}).corr("spearman")
		
		sns.heatmap(corr, annot=True, linewidths=.5,)
		
		
		import numpy as np
		
		for col in df.drop(labels=["region", "date"], axis=1).columns:
		df[f'{col}_normalized'] = (
		df.groupby(['date'])[col]
		.transform(
		lambda x: (lambda tx: (tx - tx.min()) / (tx.max() - tx.min() + 1e-8))( 
		np.where(x >= 1, np.log(x), 0)
		)
		)
		)
		
		
		def geometric_mean_no_zeros(row):
		positive_values = row[row > 0]
		if len(positive_values) == 0:
		return 0
		return gmean(positive_values)
		
		df['geometric_mean'] = df[df.columns[df.columns.str.contains("_normalized")]].apply(geometric_mean_no_zeros, axis=1)
		
		
		
		df = df.sort_values(by=["date", "region"])
		df.index = df.date
		
		import numpy as np
		import pandas as pd
		
		# Load NPSDI data (0-1 scale)
		npsdi = index_2023
		
		# Calculate deciles using pandas.qcut
		npsdi['decile'] = pd.qcut(npsdi['geometric_mean'], q=10, labels=False) + 1
		
		# Handle ties via average ranking
		npsdi['rank'] = npsdi['geometric_mean'].rank(method='average')
		npsdi['percentile'] = (npsdi['rank'] - 0.5) / len(npsdi) * 100
		npsdi['decile'] = np.floor(npsdi['percentile'] / 10) + 1
		
		
		N = len(factors)
		angles = np.linspace(0, 2 * np.pi, N, endpoint=False).tolist()
		angles += angles[:1]
		
		
		
		values =df_mosc.mean().values.tolist()
		values += values[:1]
		
		values_st =df_st.mean().values.tolist()
		values_st += values_st[:1]
		
		fig, ax = plt.subplots(figsize=(8, 8), subplot_kw={'polar': True})
		
		ax.plot(angles, values, color='blue', linewidth=2, label='Moscow')
		ax.plot(angles, values_st, color='green', linewidth=2, label='St. Petersburg')
		
		ax.set_theta_offset(np.pi / 2)
		ax.set_theta_direction(-1)
		ax.set_rlabel_position(0)
		
		plt.legend(loc='best')
		plt.xticks(angles[:-1], factors)
		plt.title('Top Regions', y=1.08)
		plt.ylim(0, 1 * 1.1)
		
		
		
		import matplotlib.pyplot as plt
		
		N = len(factors)
		angles = np.linspace(0, 2 * np.pi, N, endpoint=False).tolist()
		angles += angles[:1]
		values =df_mosc.mean().values.tolist()
		values += values[:1]
		
		values_st =df_st.mean().values.tolist()
		values_st += values_st[:1]
		
		fig, ax = plt.subplots(figsize=(8, 8), subplot_kw={'polar': True})
		
		ax.plot(angles, values, color='blue', linewidth=2, label='Chechen')
		ax.plot(angles, values_st, color='green', linewidth=2, label='Republic of Dagestan')
		
		ax.set_theta_offset(np.pi / 2)
		ax.set_theta_direction(-1)
		ax.set_rlabel_position(0)
		
		plt.legend(loc='best')
		plt.xticks(angles[:-1], factors)
		plt.title('Losers Regions', y=1.08)
		plt.ylim(0, 1 * 1.1)
		
	\end{lstlisting}
	
	
	\chapter{Fixed Effects Model training Python Code}
	
	\label{appendix:fe_model}
	
	\begin{lstlisting}[language=Python, caption={Fixed Effects Model}]
		#!/usr/bin/env python
		# coding: utf-8
		
		import warnings
		import pandas as pd
		import statsmodels.api as sm
		from statsmodels.stats.outliers_influence import variance_inflation_factor
		
		
		warnings.filterwarnings("ignore")
		pd.set_option("display.max_columns", None)
		
	
		target = pd.read_csv("../../sber_dolya_beznala.csv", sep=';')
		target["date"] = pd.to_datetime(target["date"])
		target["year"] = target["date"].dt.year
		target = target.sort_values(by='date')
		target = target[target["date"].dt.month == 1]
		target = target.drop(labels=["date"], axis=1)
		target = target.drop_duplicates(subset=["region", "year"])
		
		life_expectancy = pd.read_csv("../data/life_expectancy/life_expectancy_features.csv")
		life_expectancy = life_expectancy.rename(columns={"date": "year"})
		final_df = pd.merge(life_expectancy, target, on=["region", "year"], how="inner")
		
		
		cpi = pd.read_excel("../data/ipc_s_1992-2024.xlsx")
		cpi = cpi.rename(columns={"Unnamed: 0": "region"})
		
		cpi = cpi.melt(
			id_vars='region',  
			var_name='year', 
			value_name='cpi' 
		)
		
		cpi['year'] = cpi['year'].astype(int)
		cpi = cpi.sort_values(['region', 'year'])

		final_df = pd.merge(cpi, final_df, on=["region", "year"], how="inner")
		grp = pd.read_excel("../data/VRP_s_1998.xlsx", sheet_name="2")
		
		grp = grp.melt(
			id_vars='region', 
			var_name='year',
			value_name='grp' 
		)
		grp.head()
		grp['year'] = grp['year'].astype(int)
		
		grp = grp.sort_values(['region', 'year'])
		final_df = pd.merge(grp, final_df, on=["region", "year"], how="inner")

		
		
		fertility = pd.read_csv("../data/fertility_rate/fertility_rate_features.csv")
		fertility = fertility.rename(columns={"date": "year"})
		final_df = pd.merge(fertility, final_df, on=["region", "year"], how="inner")
		
		
		death_rate = pd.read_csv("../data/death_rate/death_rate_features.csv")
		death_rate = death_rate.rename(columns={"date": "year"})
		final_df = pd.merge(death_rate, final_df, on=["region", "year"], how="inner")
		
		nes_data = pd.read_csv("../data/hse_data/share_nes.csv")
		nes_data = nes_data.rename(columns={"Year": "year", "Reg": "region"})
		final_df = pd.merge(nes_data, final_df, on=["region", "year"], how="inner")
		
		
		df = final_df.set_index(['region', 'year']).copy().fillna(-1)
		cols_to_drop = ["fertility_rate_city", "fertility_rate_all", "fertility_rate_village", "life_expectancy_both_sex_village",
		"life_expectancy_both_sex_all", "life_expectancy_female_sex_village",
		'life_expectancy_both_sex_city',
		'life_expectancy_male_sex_city', 'life_expectancy_male_sex_village',
		'life_expectancy_female_sex_city', 'life_expectancy_female_sex_all', "death_rate_all", "death_rate_village", "death_rate_city",
		'life_expectancy_male_sex_all', "PopD5a0", "PopD5a1", "PopD5a5", "PopD5a10", "PopD5a80", "PopD5a85", "covid"]
		
		age_structure_merged = ["old", "middle", "young"]
		age_structure = [col for col in df.columns if "PopD5a" in col and col not in ["PopD5a0", "PopD5a1", "PopD5a5", "PopD5a10"]]
		control = ["cpi", "grp"]
		le = [col for col in df.columns if "life_expectancy" in col]
		dr = [col for col in df.columns if "death_rate_" in col]
		fr = [col for col in df.columns if "fertility_rate_" in col]
	
		def calculate_vif(data, target_column=None):
			if target_column:
				features = data.drop(labels=target_column, axis=1)
			else:
				features = data.copy()
			features_with_const = sm.add_constant(features)
			vif = pd.DataFrame()
			vif["Variable"] = features_with_const.columns
			vif["VIF"] = [variance_inflation_factor(features_with_const.values, i) 
			for i in range(features_with_const.shape[1])]
				vif = vif[vif["Variable"] != "const"]
			return vif
		
		
		def remove_high_vif(data, threshold=5, target_column=None):
				if target_column:
					features = data.drop(labels=[target_column], axis=1)
				else:
					features = data.copy()
		
				while True:
					vif_df = calculate_vif(features, target_column)
					max_vif = vif_df["VIF"].max()
					if max_vif <= threshold:
						break
					max_vif_row = vif_df[vif_df["VIF"] == max_vif]
					feature_to_remove = max_vif_row["Variable"].values[0]
					features = features.drop(columns=[feature_to_remove])
		
				return features
		
		cleaned_data = remove_high_vif(df.drop(labels=["value"] + cols_to_drop + control, axis=1, errors="ignore"), threshold=5, target_column=None)
		
	
		from linearmodels.panel import PanelOLS, RandomEffects
		from linearmodels.panel.results import compare
		import statsmodels.api as sm
		import statsmodels.formula.api as smf
		from statsmodels.stats.diagnostic import het_breuschpagan
		
		
		cleaned_by_vif_cols = cleaned_data.columns.tolist()
		fe_model = PanelOLS(
		df['value'],
		# df[control + age_structure + le + dr],
		df[cleaned_by_vif_cols + control],
		# df.drop(labels=["value"], axis=1, errors="ignore"),
		# df[fe_model.pvalues[fe_model.pvalues < 0.05].index.tolist()],
		entity_effects=True,
		time_effects=False,
		drop_absorbed=True
		).fit(cov_type='clustered', cluster_entity=True)
		print(fe_model)
		
		
		
		fe_model.pvalues[fe_model.pvalues < 0.05].index.tolist()
		
		fe_model.summary.as_latex()
		
		import numpy as np
		from scipy.stats import chi2
		
		b_fe = fe_model.params.values
		b_re = re_model.params.values
		
		diff = b_fe - b_re
		
		cov_diff = fe_model.cov.values - re_model.cov.values
		inv_cov_diff = np.linalg.inv(cov_diff)
		hausman_stat = diff.T @ inv_cov_diff @ diff
		
		chi2.sf(hausman_stat, len(b_fe))
		
		residuals = fe_model.resids
		exog = sm.add_constant(df[cleaned_by_vif_cols])
		bp_test = het_breuschpagan(residuals, exog)
		print(f"\nBreusch-Pagan Test: LM={bp_test[0]:.2f}, p={bp_test[1]:.3f}")
		
		
		effects = fe_model.estimated_effects.reset_index().drop(labels=["year"], axis=1).drop_duplicates(subset=["region"])
		
		
		
		effects.to_latex()
		
		
		
		params = fe_model.params
		conf_int = fe_model.conf_int()
		conf_int.columns = ['lower', 'upper']
		
		
		
	
		fe_params = params[fe_model.pvalues[fe_model.pvalues < 0.05].index.tolist()]
		fe_conf_int = conf_int.loc[fe_model.pvalues[fe_model.pvalues < 0.05].index.tolist()]
		
		
		
		import matplotlib.pyplot as plt
		import seaborn as sns
		
		sns.set_style('whitegrid')
	
		variables = fe_params.index
		estimates = fe_params.values
		lower = fe_conf_int['lower'].values
		upper = fe_conf_int['upper'].values
		
	
		fig, ax = plt.subplots(figsize=(10, 6))
		y_ticks = range(len(variables))
		
		ax.errorbar(estimates, y_ticks, xerr=[estimates - lower, upper - estimates],
		fmt='o', ecolor='gray', capsize=5, color='navy')
		ax.set_yticks(y_ticks)
		ax.set_yticklabels(variables)
		ax.set_xlabel("Margin effects")
		ax.axvline(0, color='black', linestyle='--', linewidth=1)
		
		plt.tight_layout()
		plt.show()
		
	\end{lstlisting}
	
	
	
	
\end{document}
	
	