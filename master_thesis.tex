\documentclass[14pt,a4paper, oneside]{extreport}

%%%%%%%%%% Програмный код %%%%%%%%%%
% \usepackage{minted}
% Включает подсветку команд в программах!
% Нужно, чтобы на компе стоял питон, надо поставить пакет Pygments, в котором он сделан, через pip.

% Для Windows: Жмём win+r, вводим cmd, жмём enter. Открывается консоль.
% Прописываем pip install Pygments
% Заходим в настройки texmaker и там прописываем в PdfLatex или XelaTeX:
% pdflatex -shell-escape -synctex=1 -interaction=nonstopmode %.tex

% Для Linux: Открываем консоль. Убеждаемся, что у вас установлен pip командой pip --version
% Если он не установлен, ставим его: sudo apt-get install python-pip
% Ставим пакет sudo pip install Pygments

% Для Mac: Всё то же самое, что на Linux, но через brew.

% После всего этого вы должны почувствовать себя тру-программистами!
% Документация по пакету хорошая. Сам читал, погуглите!

%%%%%%%%%% Математика %%%%%%%%%%
\usepackage{amsmath,amsfonts,amssymb,amsthm,mathtools}
% Показывать номера только у тех формул, на которые есть \eqref{} в тексте.
%\mathtoolsset{showonlyrefs=true}
%\usepackage{leqno} % Нумерация формул слева
%\usepackage{tipa} %Для формулки из логитов


\usepackage{hyphenat}


%%%%%%%%%% Шрифты %%%%%%%%
\usepackage[english]{babel} % выбор языка для документа
\usepackage[utf8]{inputenc} % задание utf8 кодировки исходного tex файла
\usepackage[X2,T2A]{fontenc}        % кодировка

\usepackage{fontspec}         % пакет для подгрузки шрифтов
\setmainfont{Times New Roman}       % задаёт основной шрифт документа

\usepackage{unicode-math}      % пакет для установки математического шрифта
\setmathfont{Asana-Math.otf}    % шрифт для математики

% Конкретный символ из конкретного шрифта
% \setmathfont[range=\int]{Neo Euler}


%%%%%%%%%% Работа с картинками %%%%%%%%%
\usepackage{graphicx}                  % Для вставки рисунков
\usepackage{graphics}
\graphicspath{{images/}{pictures/}}    % можно указать папки с картинками
\usepackage{wrapfig}                   % Обтекание рисунков и таблиц текстом


%%%%%%%%%% Работа с таблицами %%%%%%%%%%
\usepackage{tabularx}            % новые типы колонок
\usepackage{tabulary}            % и ещё новые типы колонок
\usepackage{array,delarray}      % Дополнительная работа с таблицами
\usepackage{longtable}           % Длинные таблицы
\usepackage{multirow}            % Слияние строк в таблице
\usepackage{float}               % возможность позиционировать объекты в нужном месте

\usepackage{booktabs}            % таблицы как в книгах
% Заповеди из документации к booktabs:
% 1. Будь проще! Глазам должно быть комфортно
% 2. Не используйте вертикальные линни
% 3. Не используйте двойные линии. Как правило, достаточно трёх горизонтальных линий
% 4. Единицы измерения - в шапку таблицы
% 5. Не сокращайте .1 вместо 0.1
% 6. Повторяющееся значение повторяйте, а не говорите "то же"
% 7. Есть сомнения? Выравнивай по левому краю!

%  вычисляемые колонки по tabularx
\newcolumntype{C}{>{\centering\arraybackslash}X}
\newcolumntype{L}{>{\raggedright\arraybackslash}X}
\newcolumntype{Y}{>{\arraybackslash}X}
\newcolumntype{Z}{>{\centering\arraybackslash}X}


%%%%%%%%%% Графика и рисование %%%%%%%%%%
\usepackage{tikz, pgfplots}      % язык для рисования графики из latex'a

%%%%%%%%%% Гиперссылки %%%%%%%%%%
\usepackage{xcolor}              % разные цвета

\usepackage{hyperref}
\hypersetup{
	unicode=true,           % позволяет использовать юникодные символы
	colorlinks=true,       	% true - цветные ссылки, false - ссылки в рамках
	urlcolor =blue,         % цвет ссылки на url
	linkcolor=black,        % внутренние ссылки
	citecolor=black,        % на библиографию
	breaklinks              % если ссылка не умещается в одну строку, разбивать ли ее на две части?
}


%%%%%%%%%% Другие приятные пакеты %%%%%%%%%
\usepackage{multicol}       % несколько колонок
\usepackage{verbatim}       % для многострочных комментариев
\usepackage{cmap} % для кодировки шрифтов в pdf

\usepackage{enumitem} % дополнительные плюшки для списков
%  например \begin{enumerate}[resume] позволяет продолжить нумерацию в новом списке
	
\usepackage{todonotes} % для вставки в документ заметок о том, что  осталось сделать
% \todo{Здесь надо коэффициенты исправить}
% \missingfigure{Здесь будет Последний день Помпеи}
% \listoftodos --- печатает все поставленные \todo'шки



%%%%%%%%%%%%%% ГОСТОВСКИЕ ПРИБАМБАСЫ %%%%%%%%%%%%%%%

%%% размер листа бумаги
\usepackage[paper=a4paper,top=15mm, bottom=15mm,left=35mm,right=10mm,includehead]{geometry}


\usepackage{setspace}
\setstretch{1.33}     % Межстрочный интервал
\setlength{\parindent}{1.5em} % Красная строка.


%\flushbottom       % Эта команда заставляет LaTeX чуть растягивать строки, чтобы получить идеально прямоугольную страницу
\righthyphenmin=2  % Разрешение переноса двух и более символов
\widowpenalty=10000  % Наказание за вдовствующую строку (одна строка абзаца на этой странице, остальное --- на следующей)
%\clubpenalty=10000  % Наказание за сиротствующую строку (омерзительно висящая одинокая строка в начале страницы)
\tolerance=1000     % Ещё какое-то наказание.

\usepackage{zref-totpages}

% Нумерация страниц сверху по центру
\usepackage{fancyhdr}
\pagestyle{fancy}
\fancyhead{ } % clear all fields
\fancyfoot{ } % clear all fields
\fancyhead[C]{\thepage}
% Настройка fancyhdr для размещения номеров страниц внизу
\pagestyle{fancy}
\fancyhf{} % Очищаем все поля
\fancyfoot[C]{\thepage} % Номер страницы внизу по центру
\renewcommand{\headrulewidth}{0pt} % Убираем линию вверху страницы
\renewcommand{\footrulewidth}{0pt} % Убираем линию внизу страницы
% Чтобы не прорисовывалась черта!
\renewcommand{\headrulewidth}{0pt}


% Нумерация страниц с надписью "Глава"
\usepackage{etoolbox}
\patchcmd{\chapter}{\thispagestyle{plain}}{\thispagestyle{fancy}}{}{}


%%% Заголовки
\usepackage[indentfirst]{titlesec}{\raggedleft}
% Заголовки по левому краю
% опция identfirst устанавливает отступ в первом абзаце



% В Linux этот пакет сделан косячно. Исправляет это следующий непонятный кусок кода.
\makeatletter
\patchcmd{\ttlh@hang}{\parindent\z@}{\parindent\z@\leavevmode}{}{}
\patchcmd{\ttlh@hang}{\noindent}{}{}{}
\makeatother


% Редактирования Глав и названий
\titleformat{\chapter}
{\normalfont\large\bfseries}
{\thechapter }{0.5 em}{}

% Редактирование ненумеруемых глав chapter* (Введение и тп)
\titleformat{name=\chapter,numberless}
{\centering\normalfont\bfseries\large}{}{0.25em}{\normalfont}

% Убирает чеканутые отступы вверху страницы
\titlespacing{\chapter}{0pt}{-\baselineskip}{\baselineskip}

% Более низкие уровни
\titleformat{\section}{\bfseries}{\thesection}{0.5 em}{}
\titleformat{\subsection}{\bfseries}{\thesubsection}{0.5 em}{}

\titlespacing*{\section}{0 pt}{\baselineskip}{\baselineskip}
\titlespacing*{\subsection}{0 pt}{\baselineskip}{\baselineskip}


% Содержание. Команды ниже изменяют отступы и рисуют точечки!
\usepackage{titletoc}

\titlecontents{chapter}
[1em] %
{\normalsize}
{\contentslabel{1 em}}
{\hspace{-1 em}}
{\normalsize\titlerule*[10pt]{.}\contentspage}

\titlecontents{section}
[3 em] %
{\normalsize}
{\contentslabel{1.75 em}}
{\hspace{-1.75 em}}
{\normalsize\titlerule*[10pt]{.}\contentspage}

\titlecontents{subsection}
[6 em] %
{\normalsize}
{\contentslabel{3 em}}
{\hspace{-3 em}}
{\normalsize\titlerule*[10pt]{.}\contentspage}


% Правильные подписи под таблицей и рисунком
% Документация к пакету на русском языке!
\usepackage[tableposition=top, singlelinecheck=false]{caption}
\usepackage{subcaption}


\counterwithout*{footnote}{chapter}

\DeclareCaptionStyle{base}%
[justification=centering,indention=0pt]{}
\DeclareCaptionLabelFormat{gostfigure}{Figure #2}
\DeclareCaptionLabelFormat{gosttable}{Table #2}

\DeclareCaptionLabelSeparator{gost}{~---~}
\captionsetup{labelsep=gost}

\DeclareCaptionStyle{fig01}%
[margin=5mm,justification=centering]%
{margin={3em,3em}}
\captionsetup*[figure]{style=fig01,labelsep=gost,labelformat=gostfigure,format=hang}

\DeclareCaptionStyle{tab01}%
[margin=5mm,justification=centering]%
{margin={3em,3em}}
\captionsetup*[table]{style=tab01,labelsep=gost,labelformat=gosttable,format=hang}


% межстрочный отступ в таблице
\renewcommand{\arraystretch}{1.2}



% многостраничные таблицы под РОССИЙСКИЙ СТАНДАРТ
% ВНИМАНИЕ! Обязательно за CAPTION !
\usepackage{fr-longtable}


\usepackage{totcount}

\newtotcounter{citnum} %From the package documentation
\def\oldbibitem{} \let\oldbibitem=\bibitem
\def\bibitem{\stepcounter{citnum}\oldbibitem}



%Более гибкие спсики
\usepackage{enumitem}


%%% ГОСТОВСКИЕ СПИСКИ

% Первый тип списков. Большая буква.
\newlist{Enumerate}{enumerate}{1}

\setlist[Enumerate,1]{labelsep=0.5em,leftmargin=1.25em,labelwidth=1.25em,
	parsep=0em,itemsep=0em,topsep=0ex, before={\parskip=-1em},label=\arabic{Enumeratei}.}


% Второй тип списков. Маленькая буква.
\setlist[enumerate]{label=\arabic{enumi}),parsep=0em,itemsep=0em,topsep=0.75ex, before={\parskip=-1em}}


% Третий тип списков. Два уровня.
\newlist{twoenumerate}{enumerate}{2}
\setlist[twoenumerate,1]{itemsep=0mm,parsep=0em,topsep=0.75ex,, before={\parskip=-1em},label=\asbuk{twoenumeratei})}
\setlist[twoenumerate,2]{leftmargin=1.3em,itemsep=0mm,parsep=0em,topsep=0ex, before={\parskip=-1em},label=\arabic{twoenumerateii})}


% Четвёртый тип списков. Список с тире.
\setlist[itemize]{label=--,parsep=0em,itemsep=0em,topsep=0ex, before={\parskip=-1em},after={\parskip=-1em}}


%%% WARNING WARNING WARNIN!
%%% Если в списке предложения, то должна по госту стоять точка после цифры => команда Enumerate! Если идет перечень маленьких фактов, не обособляемых предложений то после цифры идет скобка ")" => команда enumerate! Если перечень при этом ещё и двууровневый, то twoenumerate.




%%%%%%%%%% Список литературы %%%%%%%%%%

%\usepackage[%
%backend=biber, %подключение пакета biber (тоже нужен)
%bibstyle=gost-numeric, %подключение одного из четырех главных стилей biblatex-gost
%sorting=ntvy, %тип сортировки в библиографии
%]{biblatex}
\usepackage[backend=biber,style=gost-numeric, maxbibnames=9,maxcitenames=2,uniquelist=false, babel=other]{biblatex}

% Справка по 4 главным стилям для ленивых:
% gost-inline  ссылки внутри теста в круглых скобках
% gost-footnote подстрочные ссылки
% gost-numeric затекстовые ссылки
% gost-authoryear тоже затекстовые ссылки, но немного другие

% Подробнее смотри страницу 4 документации. Она на русском
\DefineBibliographyStrings{english}{%
	pages = {P\adddot},
	number = {№},
}

\DeclareSourcemap{
	\maps[datatype=bibtex]{
		\map{
			\step[fieldsource=langid, match=english, final]
			\step[fieldset=presort, fieldvalue={a}]
		}
		\map{
			\step[fieldsource=langid, notmatch=english, final]
			\step[fieldset=presort, fieldvalue={z}]
		}
	}
}


% Ещё немного настроек
\DeclareFieldFormat{postnote}{#1} %убирает с. и p.
\renewcommand*{\mkgostheading}[1]{#1} % только лишь убираем курсив с авторов
% Переопределение названия оглавления
\renewcommand{\contentsname}{Contents}


\begin{document} % Начала документа
	
	\thispagestyle{empty} % Чтобы избежать нумерации титульника
	
	\begingroup
	
	\begin{center}
		
		% Первая строка: FEDERAL STATE AUTONOMOUS EDUCATIONAL
		\fontsize{13.5}{15}\selectfont
		\textbf{FEDERAL STATE AUTONOMOUS EDUCATIONAL}
		%\vspace{-\baselineskip}
		
		% Остальной текст: INSTITUTION FOR HIGHER EDUCATION и далее
		\fontsize{13}{15}\selectfont
		\setstretch{1.5} % Устанавливаем межстрочный интервал 1.5
		INSTITUTION FOR HIGHER EDUCATION \\
		NATIONAL RESEARCH UNIVERSITY \\
		HIGHER SCHOOL OF ECONOMICS \\
		Faculty of Social Sciences
		
		\includegraphics[width=0.1\textwidth]{hse_logo.png}
		
		\vspace{1em}
		
		\fontsize{13}{15}\selectfont
		\textbf{Sergeev Vladislav Alexandrovich}
		
		\vspace{1em}
		
		\fontsize{13}{15}\selectfont
		\textbf{Master Thesis}
		
		\vspace{1em}
		
		\fontsize{14}{16}\selectfont
		\textbf{The role of demographic factors in the development of the National Payment System in Russia}
		
		\vspace{1em}
		
		field of study 38.04.04 Public Administration\\
		Master’s program ‘Population and Development’

		\vspace{4em}

	\end{center}
	
	
	\begin{minipage}[t]{0.45\textwidth}
		\raggedright
		Reviewer \\
		Candidate of Sciences (PhD) \\
		\vspace{1em}
		Larionov Alexander Vitalievich
	\end{minipage}
	\hfill
	\begin{minipage}[t]{0.45\textwidth}
		\raggedleft
		Scientific Supervisor \\
		Candidate of Sciences (PhD) \\
		\vspace{1em}
		Larionov Alexander Vitalievich
	\end{minipage}
	
	
	\vfill
	
	\begin{center}
		\normalsize Moscow, 2025
	\end{center}
	
	\endgroup
	
	%%%%%%%%%%%%%%%%%%% Introduction %%%%%%%%%%%%%%%%%%%%%%%%%%%%%%%%%%%%%%
	
	\tableofcontents  % Команда, которая создаёт оглавление
	
	\chapter*{Introduction}
	\addcontentsline{toc}{chapter}{Introduction}
	
	National payment systems are a key element of the modern economy, ensuring the uninterruptible conduct of financial transactions and contributing to the development of trade, investment and economic growth. In the context of digital transformation, they are becoming not only a calculation tool, but also a factor in increasing financial accessibility and inclusivity. The regulation of payment systems carried out by central banks is aimed at ensuring their stability, security and efficiency, which is especially important in the context of growing cyber threats, changes in consumer behavior and global economic trends. National payment systems, as the infrastructural framework of the economy, reflect the level of technological development of the country, the degree of integration of financial services into the daily life of the population and the ability to adapt to the challenges of the times.
	
	
	In Russia, the development of the national payment system has gained strategic importance in the context of achieving technological sovereignty and ensuring stability of the financial sector. Over the past decade, this system has made significant progress. It started with the creation of the MIR payment card and culminated in the introduction of the Faster Payments System. This system has become a key driver for the transition to a cashless economy in Russia. In 2024, non-cash transactions accounted for more than 85.8\% \footnote{https://www.cbr.ru/PSystem/} of retail transactions. However, it is important to note that the Russian market continues to exhibit heterogeneity in terms of digital adoption, with varying dynamics across regions, age groups and income levels. This highlights the need for a tailored approach to further development of the national payment system. The regulatory policy of the Central Bank of Russia, including piloting the introduction of the digital ruble and supporting financial technology innovation, aims to reduce these imbalances. Overcoming these imbalances, however, requires taking into consideration fundamental socio-demographic trends.
	
	
	Russia faces several significant demographic challenges, which pose both long-term difficulties and opportunities for its financial sector. One of the most significant is the aging population, which is set to continue. By 2030, according to OECD\footnote{https://mintrud.gov.ru/ministry/programms/12} estimates, the proportion of people aged 60 years and over will reach 25\%, and by 2050 it will be 30\%. This group has historically been less inclined to use digital payment methods. Only 42\% of people 55 or older are actively using online banking, according to NAFI's report from 2024\footnote{https://nafi.ru/analytics/dolya-polzovateley-mobilnogo-banka-rastet-no-rossiyane-stanovyatsya-menee-bditelnymi/}. Another significant challenge is the decline in the birth rate, which has led to a decrease in the number of young people. These age groups are crucial for credit products and financial services. Additionally, there are disparities between regions due to migration from smaller towns and rural areas. According to the Central Bank of Russia, only 48\% of residents in these areas have access to internet banking services in 2024\footnote{Основные направления повышения доступности финансовых услуг в Российской Федерации на период 2025-2027 годов}. Additionally, migration – both internally (movement to larger cities) and externally (labour migration) – is changing the composition of the workforce, creating a demand for cross-border transactions and communications in multiple languages.
	
	
	These trends create conflicting pressures on national payment systems. On the one hand, urbanization and increased digital literacy among young people are driving innovation. This is evidenced by an increase in the share of payments made through QR codes among people aged 18-35 from 12\% in 2021 to 27\% by 2033\footnote{https://www.cbr.ru/press/event/?id=23262}. However, on the other hand, ageing populations and regional inequalities are hindering the consolidation of payment infrastructure. This can be seen in regions such as the Far East and North Caucasus, where cash payments continue to account for 45\% of transactions (versus 20\% in Moscow\footnote{https://cbr.ru/press/regevent/?id=28556}) due to a larger proportion of older people and a slower adoption of digital payment methods.

	
	
	The aim of the research is to evaluate the influence of demographic variables on the evolution of the Russian national payment system and to formulate proposals for its adjustment to shifting socio-demographic circumstances.
	
	
	\vspace{2em}
	
	
	The following tasks were set in the course of the study:
	\begin{enumerate}
		\item systematize theoretical approaches to the analysis of the relationship between demographic changes and the evolution of payment systems, including foreign experience in adapting to population aging;
		\item identify key demographic trends in Russia (aging, urbanization, migration, regional differentiation) and their relationship to the dynamics of payment preferences;
		\item conduct a quantitative analysis of demographic factors' influence on national payment system metrics (non-cash payment share, digital instrument penetration, loan portfolio volume) through econometric modeling;
		\item formulate practical recommendations for regulatory bodies and payment industry players to mitigate imbalances stemming from demographic changes.
	\end{enumerate}
	
	
	The object of the study is the national payment system of Russia.
	
	
	The subject of the study is the influence of demographic factors on the development of the Russian national payment system.
	
	The following research questions have been formulated:
	
	\begin{enumerate}
		\item How do population aging and regional variations impact the pace of digitalization of payment systems and the degree of financial inclusion?
		\item What regulatory initiatives can assist in enhancing the adaptability of payment services to demographic shifts, considering the diverse payment patterns of various age groups?
	\end{enumerate}
	
	
	
	The following hypotheses were put forward:
	\begin{enumerate}
		\item Hypothesis of the impact of population aging on payment preferences (\textbf{H1}): the aging of the population leads to a decrease in the share of active users of digital payment instruments, which slows down the pace of digitalization of the national payment system;
		\item Hypothesis of regional differences in the development of the payment system (\textbf{H2}): Regions with a higher proportion of the elderly population and a low level of urbanization demonstrate lower rates of introduction of digital payment instruments compared to large cities.
	\end{enumerate}
	
	
	The first section of this study explores the operational mechanisms of Russia's national payment system and the theoretical frameworks that explain the impact of demographic changes on the payment systems.
	
	The second section examines empirical studies and international best practices to understand the intricate relationship between demographic variables and the development of national payment infrastructures, as well as how demographic factors influence the regulation of these systems. In the next part, an econometric model is being developed that analyzes the relationship between demographic trends and the development of national payment systems, which will be characterized by a new generalized index developed during the study.
	
	
	Drawing on the findings from the empirical analysis, this final section interprets the results of the model and offers methodological recommendations to enhance the resilience and efficiency of national payment systems in light of demographic changes. A comprehensive strategy will be developed for the improvement of the national payment infrastructure, tailored to meet the specific socio-demographic needs.
	
	
	The structure of the final thesis includes an introduction, 3 chapters, and a conclusion. The total number of pages is \textrm{\ztotpages} pages. The total number of literature sources is \total{citnum}.
	
	
	\chapter{Theoretical foundations and operational principles of the Russian National Payment System in the context of demographic changes}
	
	\section{The operational principles of the Russian national payment system}
	
	\section{Theoretical aspects of the relationship between demography and national payment systems}
	
	
	National payment systems, which form the basis of the financial infrastructure, are closely linked to demographic trends that affect demand for specific financial products. Demographic shifts, including changes in population demographics such as age, migration patterns and family structures, have a direct impact on consumer behaviour and preferences. By understanding these correlations, payment systems can be customized to address the diverse needs of users and anticipate future changes in demand for financial services.
	
	
	Theoretical models provide a valuable framework for analyzing these connections. This knowledge is crucial for payment system regulators, enabling them to create efficient and inclusive financial solutions that meet societal needs.
	In general, there are several approaches to examining the relationship between the development of the national payment system and demographic factors. These approaches can be categorized into the following broad groups: lifecycle hypothesis, microeconomic and financial theories, technology adoption theories, and transaction cost theory.
	
	
	
	\centerline{\textit{\underline{Life cycle hypothesis}}}


	The national payment systems, which constitute the foundation of the financial infrastructure, are closely intertwined with demographic trends that influence the demand for specific financial services. The life cycle hypothesis\footnote{Modigliani, F., \& Brumberg, R. (1954). Utility analysis and the consumption function: An interpretation of cross-section data. Franco Modigliani, 1(1), 388-436.}, a classical economic theory proposed by the economist Franco Modigliani in the mid-twentieth century, illuminates this intricate relationship. Originally conceived to explore consumer behavior, this hypothesis has proved valuable in comprehending the development of payment systems, especially in the context of an aging population and regional disparities.


	The life cycle hypothesis posits that individuals strive to optimize their financial choices throughout their lives, striking a balance between consumption, savings, and debt\footnote{Modigliani, F. (1966). The life cycle hypothesis of saving, the demand for wealth and the supply of capital. Social research, 160-217.}. Young individuals, in the process of accumulating human capital, often turn to borrowing to finance their education, home purchases, or entrepreneurial ventures. Conversely, the mature generation, having reached the pinnacle of their earnings, tend to prioritize saving, while older citizens gradually deplete their accumulated assets.


	According to theory\footnote{Ando, A., \& Modigliani, F. (1963). The “Life Cycle” Hypothesis of Saving: Aggregate Implications and Tests. The American Economic Review, 53(1), 55–84. http://www.jstor.org/stable/1817129}, there are several stages in people's lives:

	\begin{itemize}
		\item youth -- when people invest in education;
		\item working age -- when people are actively working and saving money;
		\item retirement age -- when people spend their savings.
	\end{itemize}

	Each stage is associated with specific preferences when it comes to financial instruments. Young people often opt for digital payments and mobile applications, whereas older individuals may favour traditional payment methods such as cash or bank transfers. In different periods of life, people can change their financial management strategies, which affects the amount of funds placed in digital banks.


	The life cycle hypothesis also helps to explain regional imbalances in urbanization\footnote{Modigliani, F. (1970). The life cycle hypothesis of saving and intercountry differences in the saving ratio (pp. 197-225). WA Eltis, M. FG. Scott and JN Wolfe, eds., Induction, trade, and growth: Essays in honour of Sir Roy Harrod (Clarendon Press, Oxford).}. Young people's migration to large cities results in the concentration of innovative users in megacities, leaving rural areas and smaller towns as the <<demographic reservoir>> for the older generation. This leads to a negative cycle: low demand for digital services prevents their implementation, while the lack of necessary infrastructure contributes to regional inequality.


	However, like any theory, the Modigliani theory has its limitations, which must be taken into account.
	First, it focuses on long-term trends, but it cannot explain the drastic changes caused by external factors\footnote{Taleb, N.N. (2007) The Black Swan: The Impact of the Highly Improbable. Random House, New York}. For example, the COVID-19 pandemic has accelerated the transition to cashless payments even among the elderly, which contradicts the initial assumptions of life cycle hypothesis. Secondly, the theory does not always take into account cultural and institutional features\footnote{Tversky, A., \& Kahneman, D. (1992). Advances in prospect theory: Cumulative representation of uncertainty. Journal of Risk and uncertainty, 5, 297-323.}. Nevertheless, despite these nuances, the life cycle hypothesis remains an important tool for strategic planning, especially in the context of demographic transition. 
	
	
	\centerline{\textit{\underline{Technology adoption theory}}}
	
	
	Technology Adoption Theory is an important tool for understanding the process of how people adopt and utilize new technologies. This theory becomes particularly relevant when considering the context of national payment systems, as financial technologies continue to rapidly evolve and become integrated into the daily lives of citizens. Two significant approaches within this field are the Theory of Diffusion of Innovation and <<Unified Theory of Acceptance and Use of Technology>> (UTAUT). These theories combine various concepts and models in order to explain the factors that influence technology adoption.
	
	
	UTAUT identifies four key factors influencing technology adoption\footnote{Venkatesh, V., Morris, M. G., Davis, G. B., \& Davis, F. D. (2003). User acceptance of information technology: Toward a unified view. MIS quarterly, 425-478.}: 
	\begin{enumerate}
		\item performance expectancy -- the perceived benefits of technology;
		\item effort expectancy -- ease of mastering the technology;
		\item social influence -- social pressure;
		\item facilitating conditions -- infrastructure accessibility.
	\end{enumerate}
	
	These factors not only make it possible to predict the success of the introduction of new technology, but also take into account demographic differences that may affect the perception of these aspects. For example, young people who have grown up in the digital age may perceive new payment systems as more convenient and easier to use compared to the older generation, who may be less familiar with digital tools. This highlights the importance of taking demographic factors into account when developing and implementing new payment systems.
	
	Demographic shifts, such as population aging or increased migration patterns, significantly influence the adoption of novel technologies. Older individuals may encounter obstacles when using mobile payments or digital banking services due to unfamiliarity with or apprehension about new technologies. Younger individuals, on the other hand, may be more open to exploring new financial instruments like cryptocurrencies and digital wallets. Given these demographic differences, developers of national payment systems need to tailor their offerings to meet the specific needs of different demographic groups in order to promote wider adoption of these technologies.


	Another theory that describes the adoption of new technologies is Rogers Diffusion of Innovations\footnote{Rogers, E. M., Singhal, A., \& Quinlan, M. M. (2014). Diffusion of innovations. In An integrated approach to communication theory and research (pp. 432-448). Routledge.}. A key aspect of this theory is how innovations spread within society. The theory emphasizes that adoption of a new technology doesn't happen instantaneously, but rather goes through several stages: from awareness of the technology to its eventual acceptance and use.
	
	
	The theory identified five characteristics that determine the speed of technology proliferation\footnote{Oldenburg, Brian, and Karen Glanz. "Diffusion of innovations." Health behavior and health education: Theory, research, and practice 4 (2008): 313-333}:
	
	
	\begin{enumerate}
		\item relative advantage -- the perceived benefits of technology;
		\item compatibility -- ease of mastering the technology;
		\item complexity;
		\item trialability;
		\item observability.
	\end{enumerate}
	
	
	
	According to the model, the population is divided into categories based on their willingness to innovate: innovators, early adopters, early majority, late majority, and laggards. In the context of national payment systems, this is manifested in the fact that young people and residents of megacities are more often referred to as <<innovators>>, while the elderly and rural residents are considered the <<late majority>> or <<laggards>>. In turn, for the successful implementation of new technologies, it is necessary to take into account not only the functional characteristics of the new technology, but also how it is perceived by various groups of the population. For example, if a new mobile payment system is perceived as complex or unreliable, it may slow down its spread among certain demographic groups.
	
	
	Another important aspect to consider is the influence of the social environment on technology adoption. Social influence plays a significant role in how individuals make decisions about adopting new technologies, as discussed in the technology adoption model. People often turn to the opinions and behaviors of their peers and family members when deciding whether to adopt new financial tools. This emphasizes the need for actively promoting and educating users about new technologies through various social channels and platforms. For instance, if young individuals start actively using a new payment system and share positive experiences with others, it can significantly accelerate the adoption process among older generations. By promoting these positive experiences, we can encourage others to try out the new technology and make informed decisions based on their own experiences.
	
	
	Thus, theories explaining how people adopt new technologies and how they spread are valuable tools for understanding how demographic factors influence the implementation of national payment systems. Knowing these relationships helps developers and regulators more effectively adapt their strategies to the changing needs of society. In an era of rapid technological progress, it is important not only to offer new solutions, but also to ensure that they are accessible and understandable to all segments of the population.
	
	
	\centerline{\textit{\underline{Human Capital Theory}}}
	
	The theory of human capital, as developed by Nobel Laureate Gary Becker, is crucial for understanding the relationship between demographic change and national payment systems evolution. This theory posits that knowledge, abilities, and public health constitute the basis of economic progress, as investments in education and training have a direct impact on productivity and a society's capacity to adapt to technological advancements.
	
	
	According to Becker\footnote{Becker, G. S. (1962). Investment in human capital: A theoretical analysis. Journal of political economy, 70(5, Part 2), 9-49.}, investments in human capital can be considered similar to investments in physical capital. He identifies several key aspects: firstly, individuals make decisions about how much to invest in their education and health based on the expected return on these investments; secondly, the level of education and qualifications directly affects the earnings and economic activity of individuals. Becker also emphasizes that differences in the level of human capital between population groups can lead to inequality in income and opportunities.
	
	
	Demographic factors play an important role in understanding the theory of human capital as they affect investments in this area. The structure of a population, including age groups, levels of education and gender as well as migration patterns all influence how human capital is developed and used. For example, an aging population requires more resources for health care and education. On the other hand, young people are more likely to adopt new technologies and innovation. Galor and Weil\footnote{Galor, O., \& Weil, D. N. (2000). Population, technology, and growth: From Malthusian stagnation to the demographic transition and beyond. American economic review, 90(4), 806-828.} show that the level of education depends on demographic factors such as age structure and income. The article discusses how changes in birth rates and income affect the demand for human capital. Also, Galor and Weile's model explain how migration and urbanisation lead to concentration of human resources in certain areas.
	
	
	Furthermore, demographic changes can significantly impact the development of national payment systems\footnote{EY report 2024: How Gen Z’s preference for digital is changing the payments landscape. } With increasing levels of education and technological literacy among populations, there is a growing demand for efficient and innovative payment methods. Younger generations are more likely to prefer digital transactions compared to traditional cash, prompting financial institutions to adjust their services accordingly. This trend not only improves convenience but also promotes financial inclusion by making digital payment systems accessible to underserved populations that may lack access to traditional banking services.


	Moreover, as demographic patterns evolve, so too do the demands and preferences of customers. For example, an increasing elderly population may necessitate payment systems that are intuitive and accessible, highlighting the significance of user experience in financial technology. Conversely, younger consumers may prioritize speed and safety in their transactions, promoting innovations such as contactless payments and blockchain technology. Additionally, demographic shifts can influence the regulatory framework governing national payment systems. Policy makers must consider how transformations in population dynamics impact economic stability and customer protection in digital finance. With societies becoming more diverse, regulatory frameworks must also adapt to accommodate various attitudes towards money management and technological adoption.
	
	
	In summary, the interaction between human capital theory, demographic shifts, and national payment systems is complex and significant. By investing in education and healthcare, which are key components of human capital, societies not only increase individual productivity but also shape the financial landscape. This influence extends to how payment systems are designed and utilized.


	\centerline{\textit{\underline{Behavioral Economics Theory}}}
	
	
	Behavioral economics, which emerged at the intersection of psychology and economics, challenges traditional neoclassical models that assume complete rationality in economic agents. Its key idea is that human decision-making often deviates from optimal choices due to cognitive biases, emotions, social norms, and lack of information. The seminal works of Daniel Kahneman and Amos Tversky\footnote{Tversky, A., \& Kahneman, D. (1992). Advances in prospect theory: Cumulative representation of uncertainty. Journal of Risk and uncertainty, 5, 297-323.}, such as Prospect Theory, have laid the foundation for understanding how people evaluate risks and rewards in an uncertain environment. According to this theory, individuals tend to be <<loss averse>> -- an emotional response to losses is greater than the equivalent gain. In addition, people often use heuristics - simplified mental shortcuts such as availability (estimating probability based on ease of recall) or anchoring (reliance on initial information). These cognitive biases systematically influence financial behavior, including the choice of payment instruments, savings management, and risk management.
	
	
	An important aspect of behavioral economics is the concept of <<bounded rationality>>, introduced by Herbert Simon\footnote{Simon, H. A. (1955). A behavioral model of rational choice. The quarterly journal of economics, 99-118.}. She emphasizes that individuals make decisions in conditions of limited computing power, time, and information, which leads to satisfying instead of optimizing. For example, when choosing between several payment methods, a person may choose the first available option rather than analyzing all possible alternatives. This behavior is especially typical for groups with low levels of financial literacy or under stress. Richard Thaler\footnote{Thaler, R. H. (2015). Misbehaving: The making of behavioral economics. WW Norton \& Company.}, developing these ideas, introduced the concept of <<nudging>> -- a gentle influence on people's choices through changing the decision-making architecture, for example, by setting a default option for automatic replenishment of a digital wallet.
	
	
	An example of using nudge for a national payment system is soft interventions aimed at encouraging consumers to switch from cash to safer and more convenient electronic payments. For example, a study conducted\footnote{Aydogan, S., \& Van Hove, L. (2015). Nudging consumers towards card payments: A field experiment. In International Cash Conference 2014 (pp. 589-630). Deutsche Bundesbank.} in a university cafeteria used posters with calls for card payments that appealed to a sense of loyalty and belonging to the university. This led to a 6\% increase in the share of non-cash payments, although the effect was temporary.
	
	
	Another example is nudges aimed at increasing the use of mobile payments (Apple Pay, Google Pay, etc.) in the United States. The researchers\footnote{Story, P., Smullen, D., Acquisti, A., Cranor, L. F., Sadeh, N., \& Schaub, F. (2020). From intent to action: Nudging users towards secure mobile payments. In Sixteenth Symposium on Usable Privacy and Security (SOUPS 2020) (pp. 379-415).} used informational messages that corrected users misconceptions about the security of mobile payments and helped them formulate plans for regular use of such services. This has helped to increase the adoption of more secure payment methods.
	
	
	Also in the tax sphere, nudges have been successfully applied\footnote{Calvo-Gonzalez, O., Cruz, A., \& Hernandez, M. (2018). The Ongoing Impact of ‘Nudging’People to Pay Their Taxes. World Bank Blogs, 2.} to improve the timeliness of tax payments, using messages about fines and social approval, which is indirectly related to behavior in payment systems and can be adapted to national payment systems.
	
	
	Demographic changes also have a significant impact on people's economic behavior. Different age groups, educational levels, and cultural conditions create unique patterns of behavior that can be explained in terms of behavioral economics.
	Thus, Greence\footnote{Greene, C., Perry, J., \& Stavins, J. (2024). Consumer Payment Behavior by Income and Demographics.} concludes that demographic factors such as age, education, income, race, and gender play a key role in shaping payment behavior and the development of the national payment system. Young and highly educated people are actively using digital solutions such as mobile apps and cryptocurrencies, while the older generation and low-income segments of the population prefer traditional payment methods such as cash and debit cards. Gender and ethnic differences, such as the higher propensity of women and members of minorities to use BNPL (Buy Now, Pay Later), emphasize the need to take into account a variety of needs when developing inclusive payment instruments.
	
	
	To summarize, behavioral economics provides valuable tools for designing national payment systems that consider demographic diversity. By understanding cognitive biases, social norms, and age-related characteristics, we can create inclusive solutions that reduce resistance to innovation.  In the context of demographic changes, such as aging populations, this approach ensures the sustainability and adaptability of financial infrastructure.
	
	
	
	\centerline{\textit{\underline{Transaction Cost Theory}}}
	
	
	The theory of transaction costs formulated by Ronald Coase\footnote{Coase, R. H. (1937). ”The Nature of the Firm”. Economica. 4 (16): 386–405.}, occupies a central place in understanding economic processes related to the exchange of resources. Transaction costs include all costs incurred in the preparation, conclusion, and implementation of transactions: information retrieval, negotiation, contract execution, control over their execution, and conflict resolution. Coase showed that the existence of firms is conditioned by the desire to minimize these costs through internal coordination of actions rather than market interactions. Later, Oliver Williamson\footnote{Williamson, O. E. (1979). Transaction-cost economics: the governance of contractual relations. The journal of Law and Economics, 22(2), 233-261.} expanded on this theory, focusing on the role of institutions, information asymmetry, and opportunistic behavior. He emphasized that the structure of transaction management -- market, hierarchy, or hybrid form -- depends on the specificity of assets, frequency of transactions, and uncertainty of conditions. These ideas formed the basis for analyzing the effectiveness of financial systems, including national payment systems. National payment systems, as an infrastructure element of the economy, directly affect the amount of transaction costs. Their main function is to provide secure, fast, and affordable settlements between market participants. For example, the introduction of electronic payments reduces the costs associated with cash processing, such as storage, transportation, and authentication. As digitalization progresses, payment systems reduce the cost of information retrieval (real-time access to balances) and transaction monitoring (automated controls). However, the development of these systems requires significant investment in technological infrastructure, protocol standardization, and regulation, creating new types of costs, such as cybersecurity and user adaptation costs.
	
	
	Demographic factors play a key role in determining the structure and dynamics of transaction costs, influencing the supply and demand for financial services. The age structure of the population, the level of education, urbanization, and migration flows shape payment behavior patterns that, in turn, determine requirements for national payment systems. Younger generations, who have grown up in the digital age, demonstrate a higher willingness to use innovative tools such as mobile applications and cryptocurrencies, which reduces the cost of implementing new technologies. Their preferences drive the development of instant payments and open APIs and decentralized solutions. In contrast, older generations tend to be more conservative in their choice of payment methods and prefer cash or traditional bank transfers, requiring financial institutions to maintain a redundant infrastructure to support these users, increasing transaction costs.
	
	
	Demographic aging of the population creates an additional burden on national payment systems. The growing proportion of older people requires the development of inclusive solutions: large fonts in interfaces, voice control, simplified authentication procedures. These changes, while increasing accessibility, lead to increased development and testing costs. At the same time, reducing the share of youth as the main driver of innovation may slow down the adoption of breakthrough technologies (blockchain, artificail intellegence), which in the long run will lead to increased transaction costs due to infrastructure obsolescence. The theory of the <<demographic dividend>> described by Bloom\footnote{Bloom, D. E., \& Williamson, J. G. (1998). Demographic transitions and economic miracles in emerging Asia. The World Bank Economic Review, 12(3), 419-455.} explains how changes in the age structure affect economic growth through productivity and savings, which indirectly affects the efficiency of payment systems.
	
	
	The relationship between demography and transaction costs is also evident in the regulatory sphere. Policy makers need to balance between stimulating innovation (reducing costs for businesses) and protecting vulnerable groups (increasing costs through compliance). For example, the introduction of strict Know Your Customer\footnote{Mullins, R. R., Ahearne, M., Lam, S. K., Hall, Z. R., \& Boichuk, J. P. (2014). Know your customer: How salesperson perceptions of customer relationship quality form and influence account profitability. Journal of Marketing, 78(6), 38-58.} requirements increases banks costs for customer verification, but reduces fraud risks, which is especially important in the context of the growth of digital transactions among the elderly. 
	
	
	Thus, the theory of transaction costs provides a powerful analytical tool for understanding the evolution of national payment systems in the context of demographic changes. Demographic factors, influencing behavioral patterns, technological adoption, and regulatory priorities, shape cost dynamics at the micro and macro levels. Minimizing these costs requires a flexible approach that combines technological innovation, educational initiatives, and the adaptation of institutions to a changing age landscape. 
	
	
	\centerline{\textit{\underline{Comparison of theoretical approaches}}}
	
	Given the diverse range of theoretical approaches available, it is necessary to select one in order to undertake a more in-depth exploration of the subject through the chosen methodology. To make this decision, we will employ the method of comparative analysis, based on a specific formula:
	
	\begin{equation}
		U_j = \sum_{i=1}^{n} u_{ij}, 
	\end{equation}
	
	
	where $n$ -- amount of criteria, $U_j$ -- total score for alternatives by criteria $j$, $u_{ij}$ -- score for $j$ alternatives by criteria $i$.
	
	
	A framework for assessing options:
	
	
	\begin{itemize}
		\item 1 -- hardly applicable;
		\item 2 -- generally applicable;
		\item 3 -- applicable.
	\end{itemize}
	
	
	The following characteristics were identified as criteria:
	
	
	\begin{itemize}
		\item Relevance;
		\item Considers Demographic Factors;
		\item Considers Geographic Component;
		\item Considers Dynamics.
	\end{itemize}
	
	
	
	\begin{longtable}
		{|>{\centering\small\arraybackslash}p{2.5cm}
			|>{\centering\small\arraybackslash}p{1.8cm}
			|>{\centering\small\arraybackslash}p{3cm}
			|>{\centering\small\arraybackslash}p{2cm}
			|>{\centering\small\arraybackslash}p{3cm}
			|>{\centering\small\arraybackslash}p{1.5cm}|
		}
		\caption{The application of the method of comparative alternatives in order to study theories in the context of impact demographic factors on the development of the national payment systems.}\label{tab:conclusion_theory}\\
		
		\hline
		\textbf{Theory} & \textbf{Relevance} & \textbf{Considers Demographic Factors} & \textbf{Considers Dynamics} & \textbf{Considers Geographic Component} & \textbf{Total}
		\\\hline
		\endfirsthead
		
		
		\multicolumn{6}{r}{Continuation of the table \ref{tab:conclusion_theory}}\\\hline
		\endlasthead
		
		\multicolumn{6}{@{} l}{\small\centering{Source: compiled by the author based on the analysis of theoretical framework}}
		\endfoot
		
		Life Cycle Hypothesis & 3 & 3 & 3 & 3  & 12\\\hline
		Technology adoption theory  & 2 & 2 & 3 & 2 & 9\\\hline
		Human Capital Theory  & 2 & 3 & 2 & 2 & 9\\\hline
		Behavioral Economics Theory  & 3 & 3 & 1 & 1 & 8\\\hline
		Transaction Cost Theory  & 2 & 1 & 1 & 2 & 6\\\hline
		
	\end{longtable}
	
	
	Based on the evaluation results (Table \ref{tab:conclusion_theory}), it appears that the life cycle hypothesis is the most suitable approach, as it comprehensively reflects the impact of demographic factors on the evolution of national payment systems, considering both dynamic and geographical variations. Therefore, our future work will be based on the theory proposed by Franco Modigliani, whose studies have demonstrated their applicability and relevance in this context.
	
	\section{Application of the  Modigliani Life Cycle Hypothesis to analyze the impact of demographic factors on the development of the national payment system}
	

	%%%%%%%%%%% References	%%%%%%%%%%%


	\newpage
	\addcontentsline{toc}{chapter}{References}
	\titleformat*{\section}{\bfseries\normalsize\fontsize{14}{2.5mm}\centering}
	\begin{thebibliography}{3}

		\bibitem{cental_bank_russia_natiional_statistics} Bank of Russia. National Payment System statistics (2025). Date of the application 01.04.2025\\
		https://www.cbr.ru/PSystem/

		\bibitem{cental_bank_russia_natiional_payment_dir} Центральный Банк Российской Федерации. Основные направления развития национальной платежной системы на период 2025-2027 годов. // https://www.cbr.ru/Content/Document/File/170680/onrnps\_2025-27.pdf

		\bibitem{cental_bank_russia_natiional_fin_serv} Центральный Банк Российской Федерации. Основные направления повышения доступности финансовых услуг в Российской Федерации на период 2025-2027 годов. // https://www.cbr.ru/Content/Document/File/170684/onpdfu\_2025-2027.pdf

		\bibitem{labour_min} Распоряжение Правительства РФ об утверждении Стратегии действий в интересах граждан старшего поколения в Российской Федерации до 2030 года //
		https://mintrud.gov.ru/ministry/programms/12

		\bibitem{modigliani_brumberg} Modigliani, F., \& Brumberg, R. (1954). Utility analysis and the consumption function: An interpretation of cross-section data. Franco Modigliani, 1(1), 388-436.

		\bibitem{modigliani} Ando, A., \& Modigliani, F. (1963). The “Life Cycle” Hypothesis of Saving: Aggregate Implications and Tests. The American Economic Review, 53(1), 55–84. http://www.jstor.org/stable/1817129

		\bibitem{modigliani_1966} Modigliani, F. (1966). The life cycle hypothesis of saving, the demand for wealth and the supply of capital. Social research, 160-217.
		
		\bibitem{modigliani_1970} Modigliani, F. (1970). The life cycle hypothesis of saving and intercountry differences in the saving ratio (pp. 197-225). WA Eltis, M. FG. Scott and JN Wolfe, eds., Induction, trade, and growth: Essays in honour of Sir Roy Harrod (Clarendon Press, Oxford).

		\bibitem{Kotlikoff} Kotlikoff, L. J. (1989). What determines savings?. MIT Press Books, 1.	
		
		\bibitem{taleb} Taleb, N.N. (2007) The Black Swan: The Impact of the Highly Improbable. Random House, New York
		
		\bibitem{tversky} Tversky, A., \& Kahneman, D. (1992). Advances in prospect theory: Cumulative representation of uncertainty. Journal of Risk and uncertainty, 5, 297-323.
		
		\bibitem{rogers} Rogers, E. M., Singhal, A., \& Quinlan, M. M. (2014). Diffusion of innovations. In An integrated approach to communication theory and research (pp. 432-448). Routledge.
		
		\bibitem{Venkatesh} Venkatesh, V., Morris, M. G., Davis, G. B., \& Davis, F. D. (2003). User acceptance of information technology: Toward a unified view. MIS quarterly, 425-478.
		
		\bibitem{Oldenburg} Oldenburg, Brian, and Karen Glanz. "Diffusion of innovations." Health behavior and health education: Theory, research, and practice 4 (2008): 313-333.
		
		\bibitem{becker} Becker, G. S. (1962). Investment in human capital: A theoretical analysis. Journal of political economy, 70(5, Part 2), 9-49.
		
		\bibitem{galor} Galor, O., \& Weil, D. N. (2000). Population, technology, and growth: From Malthusian stagnation to the demographic transition and beyond. American economic review, 90(4), 806-828.
		
		\bibitem{ey} EY Report: How Gen Z’s preference for digital is changing the payments landscape (2024) // https://www.ey.com/en\_us/insights/payments/how-gen-z-is-changing-the-payments-landscape
		
		\bibitem{simon} Simon, H. A. (1955). A behavioral model of rational choice. The quarterly journal of economics, 99-118.
		
		\bibitem{taler} Thaler, R. H. (2015). Misbehaving: The making of behavioral economics. WW Norton \& Company.
		
		\bibitem{nudge1} Aydogan, S., \& Van Hove, L. (2015). Nudging consumers towards card payments: A field experiment. In International Cash Conference 2014 (pp. 589-630). Deutsche Bundesbank
		
		\bibitem{nudge2} Story, P., Smullen, D., Acquisti, A., Cranor, L. F., Sadeh, N., \& Schaub, F. (2020). From intent to action: Nudging users towards secure mobile payments. In Sixteenth Symposium on Usable Privacy and Security (SOUPS 2020) (pp. 379-415.
		
		\bibitem{nudge_tax} Calvo-Gonzalez, O., Cruz, A., \& Hernandez, M. (2018). The Ongoing Impact of ‘Nudging’People to Pay Their Taxes. World Bank Blogs, 2
		
		\bibitem{Greene} Greene, C., Perry, J., \& Stavins, J. (2024). Consumer Payment Behavior by Income and Demographics.
		
		\bibitem{coase} Coase, R. H. (1937). ”The Nature of the Firm”. Economica. 4 (16): 386–405
		
		\bibitem{Williamson} Williamson, O. E. (1979). Transaction-cost economics: the governance of contractual relations. The journal of Law and Economics, 22(2), 233-261.
		
		\bibitem{bloom} Bloom, D. E., \& Williamson, J. G. (1998). Demographic transitions and economic miracles in emerging Asia. The World Bank Economic Review, 12(3), 419-455.
		
		\bibitem {kyc} Mullins, R. R., Ahearne, M., Lam, S. K., Hall, Z. R., \& Boichuk, J. P. (2014). Know your customer: How salesperson perceptions of customer relationship quality form and influence account profitability. Journal of Marketing, 78(6), 38-58.
		
		\bibitem{Vaportzis} Vaportzis, E., Giatsi Clausen, M., \& Gow, A. J. (2017). Older adults perceptions of technology and barriers to interacting with tablet computers: a focus group study. Frontiers in psychology, 8, 1687.
		
		\bibitem{Finkelstein} Finkelstein, M., \& Fishman, R. (2019). Demographics and Innovation: Evidence from Patent Data
		
		
	\end{thebibliography}
	
\end{document}
	
	