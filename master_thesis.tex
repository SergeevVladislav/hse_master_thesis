\documentclass[14pt,a4paper, oneside]{extreport}

%%%%%%%%%% Програмный код %%%%%%%%%%
% \usepackage{minted}
% Включает подсветку команд в программах!
% Нужно, чтобы на компе стоял питон, надо поставить пакет Pygments, в котором он сделан, через pip.

% Для Windows: Жмём win+r, вводим cmd, жмём enter. Открывается консоль.
% Прописываем pip install Pygments
% Заходим в настройки texmaker и там прописываем в PdfLatex или XelaTeX:
% pdflatex -shell-escape -synctex=1 -interaction=nonstopmode %.tex

% Для Linux: Открываем консоль. Убеждаемся, что у вас установлен pip командой pip --version
% Если он не установлен, ставим его: sudo apt-get install python-pip
% Ставим пакет sudo pip install Pygments

% Для Mac: Всё то же самое, что на Linux, но через brew.

% После всего этого вы должны почувствовать себя тру-программистами!
% Документация по пакету хорошая. Сам читал, погуглите!

%%%%%%%%%% Математика %%%%%%%%%%
\usepackage{amsmath,amsfonts,amssymb,amsthm,mathtools}
% Показывать номера только у тех формул, на которые есть \eqref{} в тексте.
%\mathtoolsset{showonlyrefs=true}
%\usepackage{leqno} % Нумерация формул слева
%\usepackage{tipa} %Для формулки из логитов


\usepackage{hyphenat}


%%%%%%%%%% Шрифты %%%%%%%%
\usepackage[english]{babel} % выбор языка для документа
\usepackage[utf8]{inputenc} % задание utf8 кодировки исходного tex файла
\usepackage[X2,T2A]{fontenc}        % кодировка

\usepackage{fontspec}         % пакет для подгрузки шрифтов
\setmainfont{Times New Roman}       % задаёт основной шрифт документа

\usepackage{unicode-math}      % пакет для установки математического шрифта
\setmathfont{Asana-Math.otf}    % шрифт для математики

% Конкретный символ из конкретного шрифта
% \setmathfont[range=\int]{Neo Euler}


%%%%%%%%%% Работа с картинками %%%%%%%%%
\usepackage{graphicx}                  % Для вставки рисунков
\usepackage{graphics}
\graphicspath{{images/}{pictures/}}    % можно указать папки с картинками
\usepackage{wrapfig}                   % Обтекание рисунков и таблиц текстом


%%%%%%%%%% Работа с таблицами %%%%%%%%%%
\usepackage{tabularx}            % новые типы колонок
\usepackage{tabulary}            % и ещё новые типы колонок
\usepackage{array,delarray}      % Дополнительная работа с таблицами
\usepackage{longtable}           % Длинные таблицы
\usepackage{multirow}            % Слияние строк в таблице
\usepackage{float}               % возможность позиционировать объекты в нужном месте

\usepackage{booktabs}            % таблицы как в книгах
% Заповеди из документации к booktabs:
% 1. Будь проще! Глазам должно быть комфортно
% 2. Не используйте вертикальные линни
% 3. Не используйте двойные линии. Как правило, достаточно трёх горизонтальных линий
% 4. Единицы измерения - в шапку таблицы
% 5. Не сокращайте .1 вместо 0.1
% 6. Повторяющееся значение повторяйте, а не говорите "то же"
% 7. Есть сомнения? Выравнивай по левому краю!

%  вычисляемые колонки по tabularx
\newcolumntype{C}{>{\centering\arraybackslash}X}
\newcolumntype{L}{>{\raggedright\arraybackslash}X}
\newcolumntype{Y}{>{\arraybackslash}X}
\newcolumntype{Z}{>{\centering\arraybackslash}X}


%%%%%%%%%% Графика и рисование %%%%%%%%%%
\usepackage{tikz, pgfplots}      % язык для рисования графики из latex'a

%%%%%%%%%% Гиперссылки %%%%%%%%%%
\usepackage{xcolor}              % разные цвета

\usepackage{hyperref}
\hypersetup{
	unicode=true,           % позволяет использовать юникодные символы
	colorlinks=true,       	% true - цветные ссылки, false - ссылки в рамках
	urlcolor =blue,         % цвет ссылки на url
	linkcolor=black,        % внутренние ссылки
	citecolor=black,        % на библиографию
	breaklinks              % если ссылка не умещается в одну строку, разбивать ли ее на две части?
}


%%%%%%%%%% Другие приятные пакеты %%%%%%%%%
\usepackage{multicol}       % несколько колонок
\usepackage{verbatim}       % для многострочных комментариев
\usepackage{cmap} % для кодировки шрифтов в pdf

\usepackage{enumitem} % дополнительные плюшки для списков
%  например \begin{enumerate}[resume] позволяет продолжить нумерацию в новом списке
	
\usepackage{todonotes} % для вставки в документ заметок о том, что  осталось сделать
% \todo{Здесь надо коэффициенты исправить}
% \missingfigure{Здесь будет Последний день Помпеи}
% \listoftodos --- печатает все поставленные \todo'шки



%%%%%%%%%%%%%% ГОСТОВСКИЕ ПРИБАМБАСЫ %%%%%%%%%%%%%%%

%%% размер листа бумаги
\usepackage[paper=a4paper,top=15mm, bottom=15mm,left=35mm,right=10mm,includehead]{geometry}


\usepackage{setspace}
\setstretch{1.33}     % Межстрочный интервал
\setlength{\parindent}{1.5em} % Красная строка.


%\flushbottom       % Эта команда заставляет LaTeX чуть растягивать строки, чтобы получить идеально прямоугольную страницу
\righthyphenmin=2  % Разрешение переноса двух и более символов
\widowpenalty=10000  % Наказание за вдовствующую строку (одна строка абзаца на этой странице, остальное --- на следующей)
%\clubpenalty=10000  % Наказание за сиротствующую строку (омерзительно висящая одинокая строка в начале страницы)
\tolerance=1000     % Ещё какое-то наказание.

\usepackage{zref-totpages}

% Нумерация страниц сверху по центру
\usepackage{fancyhdr}
\pagestyle{fancy}
\fancyhead{ } % clear all fields
\fancyfoot{ } % clear all fields
\fancyhead[C]{\thepage}
% Настройка fancyhdr для размещения номеров страниц внизу
\pagestyle{fancy}
\fancyhf{} % Очищаем все поля
\fancyfoot[C]{\thepage} % Номер страницы внизу по центру
\renewcommand{\headrulewidth}{0pt} % Убираем линию вверху страницы
\renewcommand{\footrulewidth}{0pt} % Убираем линию внизу страницы
% Чтобы не прорисовывалась черта!
\renewcommand{\headrulewidth}{0pt}


% Нумерация страниц с надписью "Глава"
\usepackage{etoolbox}
\patchcmd{\chapter}{\thispagestyle{plain}}{\thispagestyle{fancy}}{}{}


%%% Заголовки
\usepackage[indentfirst]{titlesec}{\raggedleft}
% Заголовки по левому краю
% опция identfirst устанавливает отступ в первом абзаце



% В Linux этот пакет сделан косячно. Исправляет это следующий непонятный кусок кода.
\makeatletter
\patchcmd{\ttlh@hang}{\parindent\z@}{\parindent\z@\leavevmode}{}{}
\patchcmd{\ttlh@hang}{\noindent}{}{}{}
\makeatother


% Редактирования Глав и названий
\titleformat{\chapter}
{\normalfont\large\bfseries}
{\thechapter }{0.5 em}{}

% Редактирование ненумеруемых глав chapter* (Введение и тп)
\titleformat{name=\chapter,numberless}
{\centering\normalfont\bfseries\large}{}{0.25em}{\normalfont}

% Убирает чеканутые отступы вверху страницы
\titlespacing{\chapter}{0pt}{-\baselineskip}{\baselineskip}

% Более низкие уровни
\titleformat{\section}{\bfseries}{\thesection}{0.5 em}{}
\titleformat{\subsection}{\bfseries}{\thesubsection}{0.5 em}{}

\titlespacing*{\section}{0 pt}{\baselineskip}{\baselineskip}
\titlespacing*{\subsection}{0 pt}{\baselineskip}{\baselineskip}


% Содержание. Команды ниже изменяют отступы и рисуют точечки!
\usepackage{titletoc}

\titlecontents{chapter}
[1em] %
{\normalsize}
{\contentslabel{1 em}}
{\hspace{-1 em}}
{\normalsize\titlerule*[10pt]{.}\contentspage}

\titlecontents{section}
[3 em] %
{\normalsize}
{\contentslabel{1.75 em}}
{\hspace{-1.75 em}}
{\normalsize\titlerule*[10pt]{.}\contentspage}

\titlecontents{subsection}
[6 em] %
{\normalsize}
{\contentslabel{3 em}}
{\hspace{-3 em}}
{\normalsize\titlerule*[10pt]{.}\contentspage}


% Правильные подписи под таблицей и рисунком
% Документация к пакету на русском языке!
\usepackage[tableposition=top, singlelinecheck=false]{caption}
\usepackage{subcaption}


\counterwithout*{footnote}{chapter}

\DeclareCaptionStyle{base}%
[justification=centering,indention=0pt]{}
\DeclareCaptionLabelFormat{gostfigure}{Figure #2}
\DeclareCaptionLabelFormat{gosttable}{Table #2}

\DeclareCaptionLabelSeparator{gost}{~---~}
\captionsetup{labelsep=gost}

\DeclareCaptionStyle{fig01}%
[margin=5mm,justification=centering]%
{margin={3em,3em}}
\captionsetup*[figure]{style=fig01,labelsep=gost,labelformat=gostfigure,format=hang}

\DeclareCaptionStyle{tab01}%
[margin=5mm,justification=centering]%
{margin={3em,3em}}
\captionsetup*[table]{style=tab01,labelsep=gost,labelformat=gosttable,format=hang}


% межстрочный отступ в таблице
\renewcommand{\arraystretch}{1.2}



% многостраничные таблицы под РОССИЙСКИЙ СТАНДАРТ
% ВНИМАНИЕ! Обязательно за CAPTION !
\usepackage{fr-longtable}


\usepackage{totcount}

\newtotcounter{citnum} %From the package documentation
\def\oldbibitem{} \let\oldbibitem=\bibitem
\def\bibitem{\stepcounter{citnum}\oldbibitem}



%Более гибкие спсики
\usepackage{enumitem}


%%% ГОСТОВСКИЕ СПИСКИ

% Первый тип списков. Большая буква.
\newlist{Enumerate}{enumerate}{1}

\setlist[Enumerate,1]{labelsep=0.5em,leftmargin=1.25em,labelwidth=1.25em,
	parsep=0em,itemsep=0em,topsep=0ex, before={\parskip=-1em},label=\arabic{Enumeratei}.}


% Второй тип списков. Маленькая буква.
\setlist[enumerate]{label=\arabic{enumi}),parsep=0em,itemsep=0em,topsep=0.75ex, before={\parskip=-1em}}


% Третий тип списков. Два уровня.
\newlist{twoenumerate}{enumerate}{2}
\setlist[twoenumerate,1]{itemsep=0mm,parsep=0em,topsep=0.75ex,, before={\parskip=-1em},label=\asbuk{twoenumeratei})}
\setlist[twoenumerate,2]{leftmargin=1.3em,itemsep=0mm,parsep=0em,topsep=0ex, before={\parskip=-1em},label=\arabic{twoenumerateii})}


% Четвёртый тип списков. Список с тире.
\setlist[itemize]{label=--,parsep=0em,itemsep=0em,topsep=0ex, before={\parskip=-1em},after={\parskip=-1em}}


%%% WARNING WARNING WARNIN!
%%% Если в списке предложения, то должна по госту стоять точка после цифры => команда Enumerate! Если идет перечень маленьких фактов, не обособляемых предложений то после цифры идет скобка ")" => команда enumerate! Если перечень при этом ещё и двууровневый, то twoenumerate.




%%%%%%%%%% Список литературы %%%%%%%%%%

%\usepackage[%
%backend=biber, %подключение пакета biber (тоже нужен)
%bibstyle=gost-numeric, %подключение одного из четырех главных стилей biblatex-gost
%sorting=ntvy, %тип сортировки в библиографии
%]{biblatex}
\usepackage[backend=biber,style=gost-numeric, maxbibnames=9,maxcitenames=2,uniquelist=false, babel=other]{biblatex}

% Справка по 4 главным стилям для ленивых:
% gost-inline  ссылки внутри теста в круглых скобках
% gost-footnote подстрочные ссылки
% gost-numeric затекстовые ссылки
% gost-authoryear тоже затекстовые ссылки, но немного другие

% Подробнее смотри страницу 4 документации. Она на русском
\DefineBibliographyStrings{english}{%
	pages = {P\adddot},
	number = {№},
}

\DeclareSourcemap{
	\maps[datatype=bibtex]{
		\map{
			\step[fieldsource=langid, match=english, final]
			\step[fieldset=presort, fieldvalue={a}]
		}
		\map{
			\step[fieldsource=langid, notmatch=english, final]
			\step[fieldset=presort, fieldvalue={z}]
		}
	}
}


% Ещё немного настроек
\DeclareFieldFormat{postnote}{#1} %убирает с. и p.
\renewcommand*{\mkgostheading}[1]{#1} % только лишь убираем курсив с авторов
% Переопределение названия оглавления
\renewcommand{\contentsname}{Contents}


\begin{document} % Начала документа
	
	\thispagestyle{empty} % Чтобы избежать нумерации титульника
	
	\begingroup
	
	\begin{center}
		
		% Первая строка: FEDERAL STATE AUTONOMOUS EDUCATIONAL
		\fontsize{13.5}{15}\selectfont
		\textbf{FEDERAL STATE AUTONOMOUS EDUCATIONAL}
		%\vspace{-\baselineskip}
		
		% Остальной текст: INSTITUTION FOR HIGHER EDUCATION и далее
		\fontsize{13}{15}\selectfont
		\setstretch{1.5} % Устанавливаем межстрочный интервал 1.5
		INSTITUTION FOR HIGHER EDUCATION \\
		NATIONAL RESEARCH UNIVERSITY \\
		HIGHER SCHOOL OF ECONOMICS \\
		Faculty of Social Sciences
		
		\includegraphics[width=0.1\textwidth]{hse_logo.png}
		
		\vspace{1em}
		
		\fontsize{13}{15}\selectfont
		\textbf{Sergeev Vladislav Alexandrovich}
		
		\vspace{1em}
		
		\fontsize{13}{15}\selectfont
		\textbf{Master Thesis}
		
		\vspace{1em}
		
		\fontsize{14}{16}\selectfont
		\textbf{The role of demographic factors in the development of the National Payment System in Russia}
		
		\vspace{1em}
		
		field of study 38.04.04 Public Administration\\
		Master’s program ‘Population and Development’

		\vspace{4em}

	\end{center}
	
	
	\begin{minipage}[t]{0.45\textwidth}
		\raggedright
		Reviewer \\
		Candidate of Sciences (PhD) \\
		\vspace{1em}
		Larionov Alexander Vitalievich
	\end{minipage}
	\hfill
	\begin{minipage}[t]{0.45\textwidth}
		\raggedleft
		Scientific Supervisor \\
		Candidate of Sciences (PhD) \\
		\vspace{1em}
		Larionov Alexander Vitalievich
	\end{minipage}
	
	
	\vfill
	
	\begin{center}
		\normalsize Moscow, 2025
	\end{center}
	
	\endgroup
	
	%%%%%%%%%%%%%%%%%%% Introduction %%%%%%%%%%%%%%%%%%%%%%%%%%%%%%%%%%%%%%
	
	\tableofcontents  % Команда, которая создаёт оглавление
	
	\chapter*{Introduction}
	\addcontentsline{toc}{chapter}{Introduction}
	
	National payment systems are a key element of the modern economy, ensuring the uninterruptible conduct of financial transactions and contributing to the development of trade, investment and economic growth. In the context of digital transformation, they are becoming not only a calculation tool, but also a factor in increasing financial accessibility and inclusivity. The regulation of payment systems carried out by central banks is aimed at ensuring their stability, security and efficiency, which is especially important in the context of growing cyber threats, changes in consumer behavior and global economic trends. National payment systems, as the infrastructural framework of the economy, reflect the level of technological development of the country, the degree of integration of financial services into the daily life of the population and the ability to adapt to the challenges of the times.
	
	
	In Russia, the development of the national payment system has gained strategic importance in the context of achieving technological sovereignty and ensuring stability of the financial sector. Over the past decade, this system has made significant progress. It started with the creation of the MIR payment card and culminated in the introduction of the Faster Payments System. This system has become a key driver for the transition to a cashless economy in Russia. In 2024, non-cash transactions accounted for more than 85.8\% \footnote{https://www.cbr.ru/PSystem/} of retail transactions. However, it is important to note that the Russian market continues to exhibit heterogeneity in terms of digital adoption, with varying dynamics across regions, age groups and income levels. This highlights the need for a tailored approach to further development of the national payment system. The regulatory policy of the Central Bank of Russia, including piloting the introduction of the digital ruble and supporting financial technology innovation, aims to reduce these imbalances. Overcoming these imbalances, however, requires taking into consideration fundamental socio-demographic trends.
	
	
	Russia faces several significant demographic challenges, which pose both long-term difficulties and opportunities for its financial sector. One of the most significant is the aging population, which is set to continue. By 2030, according to OECD\footnote{https://mintrud.gov.ru/ministry/programms/12} estimates, the proportion of people aged 60 years and over will reach 25\%, and by 2050 it will be 30\%. This group has historically been less inclined to use digital payment methods. Only 42\% of people 55 or older are actively using online banking, according to NAFI's report from 2024\footnote{https://nafi.ru/analytics/dolya-polzovateley-mobilnogo-banka-rastet-no-rossiyane-stanovyatsya-menee-bditelnymi/}. Another significant challenge is the decline in the birth rate, which has led to a decrease in the number of young people. These age groups are crucial for credit products and financial services. Additionally, there are disparities between regions due to migration from smaller towns and rural areas. According to the Central Bank of Russia, only 48\% of residents in these areas have access to internet banking services in 2024\footnote{Основные направления повышения доступности финансовых услуг в Российской Федерации на период 2025-2027 годов}. Additionally, migration – both internally (movement to larger cities) and externally (labour migration) – is changing the composition of the workforce, creating a demand for cross-border transactions and communications in multiple languages.
	
	
	These trends create conflicting pressures on national payment systems. On the one hand, urbanization and increased digital literacy among young people are driving innovation. This is evidenced by an increase in the share of payments made through QR codes among people aged 18-35 from 12\% in 2021 to 27\% by 2033\footnote{https://www.cbr.ru/press/event/?id=23262}. However, on the other hand, ageing populations and regional inequalities are hindering the consolidation of payment infrastructure. This can be seen in regions such as the Far East and North Caucasus, where cash payments continue to account for 45\% of transactions (versus 20\% in Moscow\footnote{https://cbr.ru/press/regevent/?id=28556}) due to a larger proportion of older people and a slower adoption of digital payment methods.

	
	
	The aim of the research is to evaluate the influence of demographic variables on the evolution of the Russian national payment system and to formulate proposals for its adjustment to shifting socio-demographic circumstances.
	
	
	\vspace{2em}
	
	
	The following tasks were set in the course of the study:
	\begin{enumerate}
		\item systematize theoretical approaches to the analysis of the relationship between demographic changes and the evolution of payment systems, including foreign experience in adapting to population aging;
		\item identify key demographic trends in Russia (aging, urbanization, migration, regional differentiation) and their relationship to the dynamics of payment preferences;
		\item conduct a quantitative analysis of demographic factors' influence on national payment system metrics (non-cash payment share, digital instrument penetration, loan portfolio volume) through econometric modeling;
		\item formulate practical recommendations for regulatory bodies and payment industry players to mitigate imbalances stemming from demographic changes.
	\end{enumerate}
	
	
	The object of the study is the national payment system of Russia.
	
	
	The subject of the study is the influence of demographic factors on the development of the Russian national payment system.
	
	The following research questions have been formulated:
	
	\begin{enumerate}
		\item How do population aging and regional variations impact the pace of digitalization of payment systems and the degree of financial inclusion?
		\item What regulatory initiatives can assist in enhancing the adaptability of payment services to demographic shifts, considering the diverse payment patterns of various age groups?
	\end{enumerate}
	
	
	
	The following hypotheses were put forward:
	\begin{enumerate}
		\item Hypothesis of the impact of population aging on payment preferences (\textbf{H1}): the aging of the population leads to a decrease in the share of active users of digital payment instruments, which slows down the pace of digitalization of the national payment system;
		\item Hypothesis of regional differences in the development of the payment system (\textbf{H2}): Regions with a higher proportion of the elderly population and a low level of urbanization demonstrate lower rates of introduction of digital payment instruments compared to large cities.
	\end{enumerate}
	
	
	The first section of this study explores the operational mechanisms of Russia's national payment system and the theoretical frameworks that explain the impact of demographic changes on the payment systems.
	
	The second section examines empirical studies and international best practices to understand the intricate relationship between demographic variables and the development of national payment infrastructures, as well as how demographic factors influence the regulation of these systems. In the next part, an econometric model is being developed that analyzes the relationship between demographic trends and the development of national payment systems, which will be characterized by a new generalized index developed during the study.
	
	
	Drawing on the findings from the empirical analysis, this final section interprets the results of the model and offers methodological recommendations to enhance the resilience and efficiency of national payment systems in light of demographic changes. A comprehensive strategy will be developed for the improvement of the national payment infrastructure, tailored to meet the specific socio-demographic needs.
	
	
	The structure of the final thesis includes an introduction, 3 chapters, and a conclusion. The total number of pages is \textrm{\ztotpages} pages. The total number of literature sources is \total{citnum}.
	
	
	\chapter{Theoretical foundations and operational principles of the Russian National Payment System in the context of demographic changes}
	
	\section{The operational principles of the Russian national payment system}
	
	The Russian national payment system is a set of institutions, technologies and mechanisms that ensure non-cash payments and processing of payment transactions on the territory of the country. Its structure includes several key components, each of which performs a significant function in ensuring the stability and efficiency of financial transactions.
	
	
	The central link of the system is the payment system of the Bank of Russia, through which interbank settlements, government payments, and operations related to the implementation of monetary policy take place. It ensures a high degree of reliability and security of settlement transactions between all financial market participants.
	
	
	The role is played by the National Payment Card System, operated by JSC NSPK. It is she who ensures the operation of the Mir payment card, which has become a national alternative to international payment systems and has increased the independence of the internal payment infrastructure.
	One of the most modern elements of the system is the Fast Payment System created by the Bank of Russia for instant transfers between individuals and legal entities. The system allows you to make transfers by phone number and use QR codes to pay for goods and services, which makes payment transactions accessible and convenient for a wide range of users. Developed by the Bank of Russia, the SBP allows citizens and organizations to make instant transfers between accounts in different banks around the clock and seven days a week. Since its launch in 2019, more than 21.8 billion transactions totaling 104.3 trillion rubles have been conducted through the SBP.
	
	
	The structure of the national payment system includes payment infrastructure operators, such as money transfer operators, electronic money operators, and organizations providing payment services. Their activities are regulated and controlled by the Bank of Russia in order to ensure the stability and transparency of all processes related to money transfers [6, p. 98].
	
	
	The Russian national payment system is a well-coordinated mechanism that ensures the reliability, accessibility and security of payment services, which is especially important in the context of the rapid development of the digital economy and the need to strengthen the country's financial sovereignty.
	
	
	The principle of data security and protection is the basis for the functioning of the Russian national payment system. It includes a set of legal, organizational, and technical measures aimed at preventing unauthorized access, leaks, modifications, and other threats that could compromise the integrity and confidentiality of payment information.


	In accordance with the Regulation of the Bank of Russia No. 552-P [1], payment system operators are required to ensure the protection of information by implementing legal, organizational and technical measures aimed at: ensuring the protection of information from unauthorized access, destruction, modification, blocking, copying, provision and distribution;
	, confidentiality of information;
	realization of the right of access to information in accordance with the legislation of the Russian Federation.


	Compliance with the established requirements is monitored by the Bank of Russia within the framework of supervision in the national payment system in accordance with the procedure established by it, agreed with the FSB of Russia and the FSTEC of Russia.
	
	
	To ensure the security of payment data in Russia, a Set of Bank of Russia Information Security Standards (STO BR IBBS) has been developed and is being applied. It includes requirements for information protection, personal data processing, information security risk management, and incident response. 
	
	
	In 2024, new information protection conditions came into force, which strengthen the requirements for payment system participants. In particular, banks are required to ensure that their counterparties receive information about violations of information security requirements when exchanging electronic messages. 
	
	
	The principle of accessibility and universality is one of the key aspects of the functioning of the Russian national payment system, ensuring broad and equal access of citizens and organizations to payment services throughout the country.
	
	
	Within the framework of the Russian national payment system, there are several elements aimed at ensuring the availability of payment services. The Fast Payment System (SBP), developed by the Bank of Russia, allows citizens and organizations to make transfers between accounts in different banks around the clock and seven days a week. Since its launch in 2019, SBP has significantly expanded the availability of payment services, providing users with the opportunity to make instant transfers by phone number or QR code [9, p. 383].
	
	
	The National Payment Card System (NSPK) processes domestic payment transactions, including Mir card transactions, which helps strengthen the country's financial independence and reduce dependence on foreign payment systems. NSPK is actively developing its infrastructure, including the introduction of new technologies such as contactless payments and universal QR codes, which increases the convenience and accessibility of payment services for users. The principle of accessibility and universality in the Russian national payment system is implemented through the creation and development of infrastructure that provides equal and convenient access to payment services for all users, regardless of their location or social status.
	The principle of transparency in the Russian National payment system (NPS) is to ensure the openness, accessibility and reliability of information about the functioning of the payment infrastructure, rules for the provision of payment services and respect for user rights. It is aimed at increasing the confidence of market participants, reducing risks and improving the efficiency of the system as a whole.
	
	
	In accordance with Federal Law No. 161-FZ "On the National Payment System" [2], the Bank of Russia oversees the activities of payment systems, ensuring their compliance with established requirements and standards. The regulator publishes information on the state of the payment infrastructure, transaction statistics, and reports on inspections and assessments of compliance with international standards. The information is available on the official website of the Bank of Russia and other government resources, which contributes to the openness and accessibility of data for all interested parties.
	
	
	The Bank of Russia actively interacts with participants in the payment market, organizing consultations, discussions and public hearings on the regulation and development of the payment system, which allows taking into account the opinions of various parties and making decisions aimed at improving the functioning of the NPS. Regular assessment of the compliance of payment systems with international standards is carried out, which confirms the desire to maintain a high level of transparency and quality of services. The principle of transparency in the Russian national payment system is implemented through openness of information, accessibility of data for all market participants and active interaction with stakeholders, which helps to increase the trust and efficiency of the payment infrastructure.
	
	
	The principle of reliability and stability is the cornerstone of the functioning of the Russian National Payment system (NPS). It is aimed at ensuring the smooth, secure and efficient provision of payment services, minimizing operational risks and protecting the interests of users.
	
	
	Regulatory acts such as Federal Law No. 161-FZ "On the National Payment System" [2] and standards developed by the Bank of Russia serve as the basis for the implementation of this principle. In particular, the Bank of Russia's Information Security Standards Set includes requirements for information protection, personal data processing, information security risk management, and incident response. The standards are mandatory for all participants in the payment system and serve as the basis for assessing their compliance with security requirements [8, p. 154].
	
	
	The Bank of Russia applies a risk-based approach when setting requirements for payment systems that are of particular importance for financial stability or consumer confidence in non-cash payments. Payment systems are recognized as systemically important and are subject to regular assessment of their compliance with the requirements of the legislation on NPS and international standards. The results of the assessments show a high level of compliance and are published on the website of the Bank of Russia. 
	
	
	To ensure the stability of the payment infrastructure, the Bank of Russia is actively developing and implementing technological solutions aimed at improving the reliability and fault tolerance of the system. Such solutions include the creation of backup data centers, the use of modern information security technologies, and regular stress testing of payment systems. 
	
	
	The principle of reliability and stability is the foundation for the effective functioning of the Russian national payment system. Its implementation requires an integrated approach, including the development and implementation of regulations, standards, and continuous improvement of the technological infrastructure. These efforts ensure the stability and security of payment services, which helps to strengthen user confidence and develop the country's financial system.
	
	
	The principle of efficiency and innovation in the Russian national payment system (NPS) is aimed at ensuring fast, convenient and economically feasible provision of payment services, the introduction of modern technologies that contribute to the development of the financial market.
	
	
	During the period from 2021 to 2023, the number and volume of payments to NPS increased 1.5–2 times, the share of non-cash payments in retail turnover increased from 70.3\% at the beginning of 2021 to 83.4\% at the beginning of 2024. The indicators indicate an increase in the efficiency of the payment system, improved accessibility and quality of payment services for citizens and businesses [7, p. 90].
	
	
	The introduction of innovative technologies is an aspect of NPS development. The Fast Payment System (SBP), developed by the Bank of Russia, allows transfers between accounts in different banks around the clock and seven days a week, which significantly increases the convenience and speed of transactions. Technologies such as mobile payments, biometric identification, and the use of artificial intelligence are actively developing to improve the security and efficiency of payment transactions.
	
	
	The "Main directions for the development of the national payment system for 2025-2027", approved by the Bank of Russia, provides for continued work on improving the payment infrastructure, developing regulation, stimulating product competition in the payment services market and introducing innovative solutions. Special attention is paid to digitalization of payment processes, increasing their accessibility and security, and integration with international payment systems. 
	
	
	The principle of efficiency and innovation is the basis for the further development of the Russian national payment system. Its implementation contributes to improving the quality of payment services, improving financial accessibility and security, and stimulates the introduction of modern technologies, which in turn helps strengthen the economy and increase user confidence in the payment system [5, p. 29].
	
	
	In conclusion, the principles of the Russian national payment system play a key role in ensuring its sustainability and development. Data security, accessibility, transparency, reliability, and innovation are the main factors contributing to the effective functioning of the system and building user trust. Modern standards and regulations, such as Federal Law No. 161-FZ and the Bank of Russia's Set of Standards, provide the legal basis for compliance with these principles. Innovative solutions, such as the fast payment system, the active introduction of information security technologies and mobile payments, make payment services more convenient and accessible to citizens and organizations. As a result, the Russian national payment system continues to evolve, increasing its efficiency and competitiveness both domestically and internationally.
	
	\section{Theoretical aspects of the relationship between demography and national payment systems}
	
	
	National payment systems, which form the basis of the financial infrastructure, are closely linked to demographic trends that affect demand for specific financial products. Demographic shifts, including changes in population demographics such as age, migration patterns and family structures, have a direct impact on consumer behaviour and preferences. By understanding these correlations, payment systems can be customized to address the diverse needs of users and anticipate future changes in demand for financial services.
	
	
	Theoretical models provide a valuable framework for analyzing these connections. This knowledge is crucial for payment system regulators, enabling them to create efficient and inclusive financial solutions that meet societal needs.
	In general, there are several approaches to examining the relationship between the development of the national payment system and demographic factors. These approaches can be categorized into the following broad groups: lifecycle hypothesis, microeconomic and financial theories, technology adoption theories, and transaction cost theory.
	
	
	
	\centerline{\textit{\underline{Life cycle hypothesis}}}


	The national payment systems, which constitute the foundation of the financial infrastructure, are closely intertwined with demographic trends that influence the demand for specific financial services. The life cycle hypothesis\footnote{Modigliani, F., \& Brumberg, R. (1954). Utility analysis and the consumption function: An interpretation of cross-section data. Franco Modigliani, 1(1), 388-436.}, a classical economic theory proposed by the economist Franco Modigliani in the mid-twentieth century, illuminates this intricate relationship. Originally conceived to explore consumer behavior, this hypothesis has proved valuable in comprehending the development of payment systems, especially in the context of an aging population and regional disparities.


	The life cycle hypothesis posits that individuals strive to optimize their financial choices throughout their lives, striking a balance between consumption, savings, and debt\footnote{Modigliani, F. (1966). The life cycle hypothesis of saving, the demand for wealth and the supply of capital. Social research, 160-217.}. Young individuals, in the process of accumulating human capital, often turn to borrowing to finance their education, home purchases, or entrepreneurial ventures. Conversely, the mature generation, having reached the pinnacle of their earnings, tend to prioritize saving, while older citizens gradually deplete their accumulated assets.


	According to theory\footnote{Ando, A., \& Modigliani, F. (1963). The “Life Cycle” Hypothesis of Saving: Aggregate Implications and Tests. The American Economic Review, 53(1), 55–84. http://www.jstor.org/stable/1817129}, there are several stages in people's lives:

	\begin{itemize}
		\item youth -- when people invest in education;
		\item working age -- when people are actively working and saving money;
		\item retirement age -- when people spend their savings.
	\end{itemize}

	Each stage is associated with specific preferences when it comes to financial instruments. Young people often opt for digital payments and mobile applications, whereas older individuals may favour traditional payment methods such as cash or bank transfers. In different periods of life, people can change their financial management strategies, which affects the amount of funds placed in digital banks.


	The life cycle hypothesis also helps to explain regional imbalances in urbanization\footnote{Modigliani, F. (1970). The life cycle hypothesis of saving and intercountry differences in the saving ratio (pp. 197-225). WA Eltis, M. FG. Scott and JN Wolfe, eds., Induction, trade, and growth: Essays in honour of Sir Roy Harrod (Clarendon Press, Oxford).}. Young people's migration to large cities results in the concentration of innovative users in megacities, leaving rural areas and smaller towns as the <<demographic reservoir>> for the older generation. This leads to a negative cycle: low demand for digital services prevents their implementation, while the lack of necessary infrastructure contributes to regional inequality.


	However, like any theory, the Modigliani theory has its limitations, which must be taken into account.
	First, it focuses on long-term trends, but it cannot explain the drastic changes caused by external factors\footnote{Taleb, N.N. (2007) The Black Swan: The Impact of the Highly Improbable. Random House, New York}. For example, the COVID-19 pandemic has accelerated the transition to cashless payments even among the elderly, which contradicts the initial assumptions of life cycle hypothesis. Secondly, the theory does not always take into account cultural and institutional features\footnote{Tversky, A., \& Kahneman, D. (1992). Advances in prospect theory: Cumulative representation of uncertainty. Journal of Risk and uncertainty, 5, 297-323.}. Nevertheless, despite these nuances, the life cycle hypothesis remains an important tool for strategic planning, especially in the context of demographic transition. 
	
	
	\centerline{\textit{\underline{Technology adoption theory}}}
	
	
	Technology Adoption Theory is an important tool for understanding the process of how people adopt and utilize new technologies. This theory becomes particularly relevant when considering the context of national payment systems, as financial technologies continue to rapidly evolve and become integrated into the daily lives of citizens. Two significant approaches within this field are the Theory of Diffusion of Innovation and <<Unified Theory of Acceptance and Use of Technology>> (UTAUT). These theories combine various concepts and models in order to explain the factors that influence technology adoption.
	
	
	UTAUT identifies four key factors influencing technology adoption\footnote{Venkatesh, V., Morris, M. G., Davis, G. B., \& Davis, F. D. (2003). User acceptance of information technology: Toward a unified view. MIS quarterly, 425-478.}: 
	\begin{enumerate}
		\item performance expectancy -- the perceived benefits of technology;
		\item effort expectancy -- ease of mastering the technology;
		\item social influence -- social pressure;
		\item facilitating conditions -- infrastructure accessibility.
	\end{enumerate}
	
	These factors not only make it possible to predict the success of the introduction of new technology, but also take into account demographic differences that may affect the perception of these aspects. For example, young people who have grown up in the digital age may perceive new payment systems as more convenient and easier to use compared to the older generation, who may be less familiar with digital tools. This highlights the importance of taking demographic factors into account when developing and implementing new payment systems.
	
	Demographic shifts, such as population aging or increased migration patterns, significantly influence the adoption of novel technologies. Older individuals may encounter obstacles when using mobile payments or digital banking services due to unfamiliarity with or apprehension about new technologies. Younger individuals, on the other hand, may be more open to exploring new financial instruments like cryptocurrencies and digital wallets. Given these demographic differences, developers of national payment systems need to tailor their offerings to meet the specific needs of different demographic groups in order to promote wider adoption of these technologies.


	Another theory that describes the adoption of new technologies is Rogers Diffusion of Innovations\footnote{Rogers, E. M., Singhal, A., \& Quinlan, M. M. (2014). Diffusion of innovations. In An integrated approach to communication theory and research (pp. 432-448). Routledge.}. A key aspect of this theory is how innovations spread within society. The theory emphasizes that adoption of a new technology doesn't happen instantaneously, but rather goes through several stages: from awareness of the technology to its eventual acceptance and use.
	
	
	The theory identified five characteristics that determine the speed of technology proliferation\footnote{Oldenburg, Brian, and Karen Glanz. "Diffusion of innovations." Health behavior and health education: Theory, research, and practice 4 (2008): 313-333}:
	
	
	\begin{enumerate}
		\item relative advantage -- the perceived benefits of technology;
		\item compatibility -- ease of mastering the technology;
		\item complexity;
		\item trialability;
		\item observability.
	\end{enumerate}
	
	
	
	According to the model, the population is divided into categories based on their willingness to innovate: innovators, early adopters, early majority, late majority, and laggards. In the context of national payment systems, this is manifested in the fact that young people and residents of megacities are more often referred to as <<innovators>>, while the elderly and rural residents are considered the <<late majority>> or <<laggards>>. In turn, for the successful implementation of new technologies, it is necessary to take into account not only the functional characteristics of the new technology, but also how it is perceived by various groups of the population. For example, if a new mobile payment system is perceived as complex or unreliable, it may slow down its spread among certain demographic groups.
	
	
	Another important aspect to consider is the influence of the social environment on technology adoption. Social influence plays a significant role in how individuals make decisions about adopting new technologies, as discussed in the technology adoption model. People often turn to the opinions and behaviors of their peers and family members when deciding whether to adopt new financial tools. This emphasizes the need for actively promoting and educating users about new technologies through various social channels and platforms. For instance, if young individuals start actively using a new payment system and share positive experiences with others, it can significantly accelerate the adoption process among older generations. By promoting these positive experiences, we can encourage others to try out the new technology and make informed decisions based on their own experiences.
	
	
	Thus, theories explaining how people adopt new technologies and how they spread are valuable tools for understanding how demographic factors influence the implementation of national payment systems. Knowing these relationships helps developers and regulators more effectively adapt their strategies to the changing needs of society. In an era of rapid technological progress, it is important not only to offer new solutions, but also to ensure that they are accessible and understandable to all segments of the population.
	
	
	\centerline{\textit{\underline{Human Capital Theory}}}
	
	The theory of human capital, as developed by Nobel Laureate Gary Becker, is crucial for understanding the relationship between demographic change and national payment systems evolution. This theory posits that knowledge, abilities, and public health constitute the basis of economic progress, as investments in education and training have a direct impact on productivity and a society's capacity to adapt to technological advancements.
	
	
	According to Becker\footnote{Becker, G. S. (1962). Investment in human capital: A theoretical analysis. Journal of political economy, 70(5, Part 2), 9-49.}, investments in human capital can be considered similar to investments in physical capital. He identifies several key aspects: firstly, individuals make decisions about how much to invest in their education and health based on the expected return on these investments; secondly, the level of education and qualifications directly affects the earnings and economic activity of individuals. Becker also emphasizes that differences in the level of human capital between population groups can lead to inequality in income and opportunities.
	
	
	Demographic factors play an important role in understanding the theory of human capital as they affect investments in this area. The structure of a population, including age groups, levels of education and gender as well as migration patterns all influence how human capital is developed and used. For example, an aging population requires more resources for health care and education. On the other hand, young people are more likely to adopt new technologies and innovation. Galor and Weil\footnote{Galor, O., \& Weil, D. N. (2000). Population, technology, and growth: From Malthusian stagnation to the demographic transition and beyond. American economic review, 90(4), 806-828.} show that the level of education depends on demographic factors such as age structure and income. The article discusses how changes in birth rates and income affect the demand for human capital. Also, Galor and Weile's model explain how migration and urbanisation lead to concentration of human resources in certain areas.
	
	
	Furthermore, demographic changes can significantly impact the development of national payment systems\footnote{EY report 2024: How Gen Z’s preference for digital is changing the payments landscape. } With increasing levels of education and technological literacy among populations, there is a growing demand for efficient and innovative payment methods. Younger generations are more likely to prefer digital transactions compared to traditional cash, prompting financial institutions to adjust their services accordingly. This trend not only improves convenience but also promotes financial inclusion by making digital payment systems accessible to underserved populations that may lack access to traditional banking services.


	Moreover, as demographic patterns evolve, so too do the demands and preferences of customers. For example, an increasing elderly population may necessitate payment systems that are intuitive and accessible, highlighting the significance of user experience in financial technology. Conversely, younger consumers may prioritize speed and safety in their transactions, promoting innovations such as contactless payments and blockchain technology. Additionally, demographic shifts can influence the regulatory framework governing national payment systems. Policy makers must consider how transformations in population dynamics impact economic stability and customer protection in digital finance. With societies becoming more diverse, regulatory frameworks must also adapt to accommodate various attitudes towards money management and technological adoption.
	
	
	In summary, the interaction between human capital theory, demographic shifts, and national payment systems is complex and significant. By investing in education and healthcare, which are key components of human capital, societies not only increase individual productivity but also shape the financial landscape. This influence extends to how payment systems are designed and utilized.


	\centerline{\textit{\underline{Behavioral Economics Theory}}}
	
	
	Behavioral economics, which emerged at the intersection of psychology and economics, challenges traditional neoclassical models that assume complete rationality in economic agents. Its key idea is that human decision-making often deviates from optimal choices due to cognitive biases, emotions, social norms, and lack of information. The seminal works of Daniel Kahneman and Amos Tversky\footnote{Tversky, A., \& Kahneman, D. (1992). Advances in prospect theory: Cumulative representation of uncertainty. Journal of Risk and uncertainty, 5, 297-323.}, such as Prospect Theory, have laid the foundation for understanding how people evaluate risks and rewards in an uncertain environment. According to this theory, individuals tend to be <<loss averse>> -- an emotional response to losses is greater than the equivalent gain. In addition, people often use heuristics - simplified mental shortcuts such as availability (estimating probability based on ease of recall) or anchoring (reliance on initial information). These cognitive biases systematically influence financial behavior, including the choice of payment instruments, savings management, and risk management.
	
	
	An important aspect of behavioral economics is the concept of <<bounded rationality>>, introduced by Herbert Simon\footnote{Simon, H. A. (1955). A behavioral model of rational choice. The quarterly journal of economics, 99-118.}. She emphasizes that individuals make decisions in conditions of limited computing power, time, and information, which leads to satisfying instead of optimizing. For example, when choosing between several payment methods, a person may choose the first available option rather than analyzing all possible alternatives. This behavior is especially typical for groups with low levels of financial literacy or under stress. Richard Thaler\footnote{Thaler, R. H. (2015). Misbehaving: The making of behavioral economics. WW Norton \& Company.}, developing these ideas, introduced the concept of <<nudging>> -- a gentle influence on people's choices through changing the decision-making architecture, for example, by setting a default option for automatic replenishment of a digital wallet.
	
	
	An example of using nudge for a national payment system is soft interventions aimed at encouraging consumers to switch from cash to safer and more convenient electronic payments. For example, a study conducted\footnote{Aydogan, S., \& Van Hove, L. (2015). Nudging consumers towards card payments: A field experiment. In International Cash Conference 2014 (pp. 589-630). Deutsche Bundesbank.} in a university cafeteria used posters with calls for card payments that appealed to a sense of loyalty and belonging to the university. This led to a 6\% increase in the share of non-cash payments, although the effect was temporary.
	
	
	Another example is nudges aimed at increasing the use of mobile payments (Apple Pay, Google Pay, etc.) in the United States. The researchers\footnote{Story, P., Smullen, D., Acquisti, A., Cranor, L. F., Sadeh, N., \& Schaub, F. (2020). From intent to action: Nudging users towards secure mobile payments. In Sixteenth Symposium on Usable Privacy and Security (SOUPS 2020) (pp. 379-415).} used informational messages that corrected users misconceptions about the security of mobile payments and helped them formulate plans for regular use of such services. This has helped to increase the adoption of more secure payment methods.
	
	
	Also in the tax sphere, nudges have been successfully applied\footnote{Calvo-Gonzalez, O., Cruz, A., \& Hernandez, M. (2018). The Ongoing Impact of ‘Nudging’People to Pay Their Taxes. World Bank Blogs, 2.} to improve the timeliness of tax payments, using messages about fines and social approval, which is indirectly related to behavior in payment systems and can be adapted to national payment systems.
	
	
	Demographic changes also have a significant impact on people's economic behavior. Different age groups, educational levels, and cultural conditions create unique patterns of behavior that can be explained in terms of behavioral economics.
	Thus, Greence\footnote{Greene, C., Perry, J., \& Stavins, J. (2024). Consumer Payment Behavior by Income and Demographics.} concludes that demographic factors such as age, education, income, race, and gender play a key role in shaping payment behavior and the development of the national payment system. Young and highly educated people are actively using digital solutions such as mobile apps and cryptocurrencies, while the older generation and low-income segments of the population prefer traditional payment methods such as cash and debit cards. Gender and ethnic differences, such as the higher propensity of women and members of minorities to use BNPL (Buy Now, Pay Later), emphasize the need to take into account a variety of needs when developing inclusive payment instruments.
	
	
	To summarize, behavioral economics provides valuable tools for designing national payment systems that consider demographic diversity. By understanding cognitive biases, social norms, and age-related characteristics, we can create inclusive solutions that reduce resistance to innovation.  In the context of demographic changes, such as aging populations, this approach ensures the sustainability and adaptability of financial infrastructure.
	
	
	
	\centerline{\textit{\underline{Transaction Cost Theory}}}
	
	
	The theory of transaction costs formulated by Ronald Coase\footnote{Coase, R. H. (1937). ”The Nature of the Firm”. Economica. 4 (16): 386–405.}, occupies a central place in understanding economic processes related to the exchange of resources. Transaction costs include all costs incurred in the preparation, conclusion, and implementation of transactions: information retrieval, negotiation, contract execution, control over their execution, and conflict resolution. Coase showed that the existence of firms is conditioned by the desire to minimize these costs through internal coordination of actions rather than market interactions. Later, Oliver Williamson\footnote{Williamson, O. E. (1979). Transaction-cost economics: the governance of contractual relations. The journal of Law and Economics, 22(2), 233-261.} expanded on this theory, focusing on the role of institutions, information asymmetry, and opportunistic behavior. He emphasized that the structure of transaction management -- market, hierarchy, or hybrid form -- depends on the specificity of assets, frequency of transactions, and uncertainty of conditions. These ideas formed the basis for analyzing the effectiveness of financial systems, including national payment systems. National payment systems, as an infrastructure element of the economy, directly affect the amount of transaction costs. Their main function is to provide secure, fast, and affordable settlements between market participants. For example, the introduction of electronic payments reduces the costs associated with cash processing, such as storage, transportation, and authentication. As digitalization progresses, payment systems reduce the cost of information retrieval (real-time access to balances) and transaction monitoring (automated controls). However, the development of these systems requires significant investment in technological infrastructure, protocol standardization, and regulation, creating new types of costs, such as cybersecurity and user adaptation costs.
	
	
	Demographic factors play a key role in determining the structure and dynamics of transaction costs, influencing the supply and demand for financial services. The age structure of the population, the level of education, urbanization, and migration flows shape payment behavior patterns that, in turn, determine requirements for national payment systems. Younger generations, who have grown up in the digital age, demonstrate a higher willingness to use innovative tools such as mobile applications and cryptocurrencies, which reduces the cost of implementing new technologies. Their preferences drive the development of instant payments and open APIs and decentralized solutions. In contrast, older generations tend to be more conservative in their choice of payment methods and prefer cash or traditional bank transfers, requiring financial institutions to maintain a redundant infrastructure to support these users, increasing transaction costs.
	
	
	Demographic aging of the population creates an additional burden on national payment systems. The growing proportion of older people requires the development of inclusive solutions: large fonts in interfaces, voice control, simplified authentication procedures. These changes, while increasing accessibility, lead to increased development and testing costs. At the same time, reducing the share of youth as the main driver of innovation may slow down the adoption of breakthrough technologies (blockchain, artificail intellegence), which in the long run will lead to increased transaction costs due to infrastructure obsolescence. The theory of the <<demographic dividend>> described by Bloom\footnote{Bloom, D. E., \& Williamson, J. G. (1998). Demographic transitions and economic miracles in emerging Asia. The World Bank Economic Review, 12(3), 419-455.} explains how changes in the age structure affect economic growth through productivity and savings, which indirectly affects the efficiency of payment systems.
	
	
	The relationship between demography and transaction costs is also evident in the regulatory sphere. Policy makers need to balance between stimulating innovation (reducing costs for businesses) and protecting vulnerable groups (increasing costs through compliance). For example, the introduction of strict Know Your Customer\footnote{Mullins, R. R., Ahearne, M., Lam, S. K., Hall, Z. R., \& Boichuk, J. P. (2014). Know your customer: How salesperson perceptions of customer relationship quality form and influence account profitability. Journal of Marketing, 78(6), 38-58.} requirements increases banks costs for customer verification, but reduces fraud risks, which is especially important in the context of the growth of digital transactions among the elderly. 
	
	
	Thus, the theory of transaction costs provides a powerful analytical tool for understanding the evolution of national payment systems in the context of demographic changes. Demographic factors, influencing behavioral patterns, technological adoption, and regulatory priorities, shape cost dynamics at the micro and macro levels. Minimizing these costs requires a flexible approach that combines technological innovation, educational initiatives, and the adaptation of institutions to a changing age landscape. 
	
	
	\centerline{\textit{\underline{Comparison of theoretical approaches}}}
	
	Given the diverse range of theoretical approaches available, it is necessary to select one in order to undertake a more in-depth exploration of the subject through the chosen methodology. To make this decision, we will employ the method of comparative analysis, based on a specific formula:
	
	\begin{equation}
		U_j = \sum_{i=1}^{n} u_{ij}, 
	\end{equation}
	
	
	where $n$ -- amount of criteria, $U_j$ -- total score for alternatives by criteria $j$, $u_{ij}$ -- score for $j$ alternatives by criteria $i$.
	
	
	A framework for assessing options:
	
	
	\begin{itemize}
		\item 1 -- hardly applicable;
		\item 2 -- generally applicable;
		\item 3 -- applicable.
	\end{itemize}
	
	
	The following characteristics were identified as criteria:
	
	
	\begin{itemize}
		\item Relevance;
		\item Considers Demographic Factors;
		\item Considers Geographic Component;
		\item Considers Dynamics.
	\end{itemize}
	
	
	
	\begin{longtable}
		{|>{\centering\small\arraybackslash}p{2.5cm}
			|>{\centering\small\arraybackslash}p{1.8cm}
			|>{\centering\small\arraybackslash}p{3cm}
			|>{\centering\small\arraybackslash}p{2cm}
			|>{\centering\small\arraybackslash}p{3cm}
			|>{\centering\small\arraybackslash}p{1.5cm}|
		}
		\caption{The application of the method of comparative alternatives in order to study theories in the context of impact demographic factors on the development of the national payment systems.}\label{tab:conclusion_theory}\\
		
		\hline
		\textbf{Theory} & \textbf{Relevance} & \textbf{Considers Demographic Factors} & \textbf{Considers Dynamics} & \textbf{Considers Geographic Component} & \textbf{Total}
		\\\hline
		\endfirsthead
		
		
		\multicolumn{6}{r}{Continuation of the table \ref{tab:conclusion_theory}}\\\hline
		\endlasthead
		
		\multicolumn{6}{@{} l}{\small\centering{Source: compiled by the author based on the analysis of theoretical framework}}
		\endfoot
		
		Life Cycle Hypothesis & 3 & 3 & 3 & 3  & 12\\\hline
		Technology adoption theory  & 2 & 2 & 3 & 2 & 9\\\hline
		Human Capital Theory  & 2 & 3 & 2 & 2 & 9\\\hline
		Behavioral Economics Theory  & 3 & 3 & 1 & 1 & 8\\\hline
		Transaction Cost Theory  & 2 & 1 & 1 & 2 & 6\\\hline
		
	\end{longtable}
	
	
	Based on the evaluation results (Table \ref{tab:conclusion_theory}), it appears that the life cycle hypothesis is the most suitable approach, as it comprehensively reflects the impact of demographic factors on the evolution of national payment systems, considering both dynamic and geographical variations. Therefore, our future work will be based on the theory proposed by Franco Modigliani, whose studies have demonstrated their applicability and relevance in this context.
	
	\section{Application of the  Modigliani Life Cycle Hypothesis to analyze the impact of demographic factors on the development of the national payment system}
	
	\chapter{An empirical study of the impact of demographic factors on the development of the Russian national payment system}
	
	
	\section{Analysis of empirical approaches to assess the impact of demographic factors on the development of national payment systems}
	
	
	This section will review empirical research on the impact of demographic factors on the development of national payment systems. Variables will then be identified for subsequent econometric research.
	
	
	One of the traditional approaches to examining the influence of demographic variables on the evolution of national payment systems involves the utilization of survey data. Stavins\footnote{Stavins, J. (2016). The effect of demographics on payment behavior: panel data with sample selection (No. 16-5). Working Papers.} article explores the role of demographic and socioeconomic variables in shaping consumer payment habits in the United States, utilizing data from the <<Consumer Payment Choice Survey>> (CPCS) for the years 2009–2013. The writer analyzes how factors such as age, educational attainment, income, racial background, gender, and other attributes influence the adoption and use of various payment methods, including cash, credit cards, debit cards, online banking, and checks.
	The primary objective of the research is to identify trends that persist even when accounting for the attributes of each payment method (convenience, safety, and cost) and other factors. The study employs a panel model incorporating a selection adjustment proposed by Wooldridge\footnote{Wooldridge, J. M. 1995. “Selection Corrections for Panel Data Models Under Conditional Mean
		Independence Assumptions.” Journal of Econometrics 68 (1): 115–132.}.
	At the initial stage, a probabilistic regression model is used to estimate the probability of acceptance of a particular payment instrument $j$, where the probability of using the payment instrument is predicted for each year $t$:
	
	\begin{equation}
		P(s_{ijt} = 1 |  x_{it}) = \Phi \left( \delta_{t0} + x_{it} \delta_t + \vartheta_{it} \right),
	\end{equation}
	
	
	where $s_{ijt} = 1$ -- if the consumer $i$ accepted the instrument $j$ in the year $t$, and 0 otherwise; $x_{it}$ is a vector of independent variables (demographics, income, region, instrument characteristics).
	
	
	Then, the degree of its use is estimated using the usual least squares regression with random effects, which allowed the authors to take into account the fact that a person's propensity to use a particular instrument may vary in depending on his mastery of this tool:
	
	\begin{equation}
		y_{ijt} = c + x_{it} \beta + \gamma_t \hat{\lambda}_{ijt} + \alpha_i + u_{it},
	\end{equation}
	
	where $y_{ijt}$ is a proportion of transactions using the $j$ instrument; $\alpha_i$ is a random individual effects; $\gamma_t$ is a coefficient reflecting the selection adjustment; $\hat{\lambda}_{ijt}$ is a Inverse Mills Ratio. The selection of random effects over fixed effects is justified by the results of the Hausman test, which confirms the absence of a correlation between individual-specific effects and the independent variables.
	
	
	The author concludes that young people aged 18-25 are more likely to use online banking and debit cards, while older people over 65 continue to use checks. Women also spend 5\% less in cash, but they use debit cards and check more frequently, which may be due to their household budget management strategies.
	
	
	A further study\footnote{Greene, C., Perry, J., \& Stavins, J. (2024). Consumer Payment Behavior by Income and Demographics.
	} using the same CPCS database for 2023 also shows similar results. The authors demonstrate a stable relationship between socio-economic factors and preferences for both traditional and innovative payment methods. They apply the two-stage Heckman model to analyze the extensive and intensive margins of payment instrument use. The interesting result is that university graduates are 2.5 times more likely to use credit cards than those who have not completed high school. However, the least educated and lower-income groups continue to prefer cash and debit cards, consistent with the theory of financial vulnerability.
	
	
	The article by Camilleri and Agius\footnote{Camilleri, S. J., \& Agius, C. (2021). Choosing between innovative and traditional payment systems: an empirical analysis of European trends. Journal of Innovation Management, 9(4), 29-57.
	} examines the factors that influence the choice between traditional and innovative payment systems in Europe, with a particular focus on demographic characteristics. The study combines econometric analysis at the macro level (across 28 EU countries) with a survey of individual preferences in Malta. This allows us to identify both general trends and specific characteristics of a particular region. Demographic variables, such as age, education, gender, and employment, are central to this work, as they highlight their role in adopting new technologies. Age stratification has proven to be a significant factor. The econometric model revealed an ambiguous relationship, with the proportion of people aged 15-64 being negatively correlated with the use of traditional methods, consistent with the hypothesis that the economically active population has greater technological literacy. However, there was also a positive correlation between the proportion of 20-39 year olds and the use of these traditional systems, which was unexpected. The authors suggest that this group, though technologically oriented, may still maintain habits due to convenience, such as using debit cards, and this requires further investigation. The results of a survey in Malta with 90 respondents confirmed the significance of age. Older respondents over the age of 60 showed a strong preference for cash and a complete lack of interest in mobile banking, while younger respondents aged 16-29 were more likely to use online banking and mobile payments.
	
	
	Another fascinating study\footnote{Тишин, А. (2020). Влияние демографии на развитие финансового сектора Российской Федерации. Аналитическая записка Департамента исследований и прогнозирования, Банк России.} is the analysis of the impact of demographic factors on the development of the financial sector in Russia. This study is based on Rosstat's demographic forecasts, which cover three scenarios: low, medium and high. In addition, the work uses data from household surveys conducted by the Russian Ministry of Finance in 2013 and 2015. The research methodology includes logistic regression to assess the probability of owning financial assets and liabilities, as well as predictive calculations based on the age structure of the population. It is important to note that in the modeling, assumptions were made that the younger generation, as a rule, seeks to minimize the use of cash and adopts the habit of using cards. The results indicate that, for instance, the proportion of the population requiring loans will decrease by 3-3.5 percentage points by 2036, while the use of credit cards will increase by 4-4.5 percent. Meanwhile, the amount of deposits and mortgages will slightly decrease. A significant aspect of this study was an analysis of age groups contributions: reductions among those aged 25-44 had a more significant negative impact on credit activity, while older groups maintained more conservative preferences such as saving money instead of spending it.
	
	
	In the context of studying the influence of demographic factors on the development of the national payment system, the Arkhipova paper\footnote{Arkhipova, N. (2022). Impact of Demographic Trends on Retail Banking. Procedia Computer Science, 214, 831-836.
	} is of particular interest. Her research focuses on analyzing the relationship between socio-demographic trends and retail banking dynamics in the regions of Russia. The author identifies key demographic and social variables that influence public demand for banking products, such as lending and savings. The research is based on econometric analysis using panel data from 78 Russian regions over the period 2010 - 2020. Regression models, including autoregressive components, are used to assess the impact of demographic, social, and economic factors on lending volumes. The formalized model takes into account various variables and their interactions, allowing for a more accurate assessment of the complex relationship between demographic trends and banking activity:
	
	\begin{equation}
		BI_t = \sum_{i,t} \alpha_i FC_{i,t} + \sum_{j,t} \beta_j D_{j,t} + \sum_{k,t} \gamma_k FI_{t} + \sum_{l,t,n} \mu_l BI_{t-n} + \epsilon_t,
	\end{equation}
	
	
	where $BI$ is indicators of banking product penetration; $FC$ is a control variables (income level, etc.); $D$ is a demographic characteristics of Russian regions; $FI$ is an indicators of financial innovation.
	
	Arkhipova identifies several key demographic trends relevant to the financial sector:
	
	\begin{itemize}
		\item the increasing number of older adults is changing the way people save and borrow money;
		\item high proportion of the urban population is associated with a greater propensity to use innovative financial products, including non-cash transactions;
		\item growing number of single-parent and same-sex families requires the adaptation of credit products, which affects the design of payment instruments.
		
	\end{itemize}
	
	
	As part of the research on the impact of demographic factors on the evolution of the national payment system, a paper by Chawla and Joshi\footnote{Chawla, D., \& Joshi, H. (2018). The moderating effect of demographic variables on mobile banking adoption: An empirical investigation. Global Business Review, 19(3\_suppl), S90-S113.} is of significant interest. This study examines the moderating impact of demographic variables on the adoption of mobile banking. The authors integrated the Technology Acceptance Model and Innovation Diffusion Theory to develop a model that incorporates variables such as trust, convenience, efficiency, compatibility with lifestyle, and ease of use. Their focus was on identifying the significance of demographic traits in determining the magnitude of these factors impact on users attitudes toward mobile banking services. The study was conducted on data from 367 participants in India, predominantly male (74.4\%), with a high educational background (88.3\%). Multiple linear regression and Fisher's transformation were employed for analysis. The authors concluded that female participants were more likely to be influenced by ease of use, suggesting that demographic factors significantly influenced attitudes towards mobile banking services. Participants over the age of 30 demonstrated a stronger link between trust and attitudes, which can be attributed to their cautious approach towards digital risks. Furthermore, married participants tended to prioritize convenience, likely owing to their responsibility for managing family finances.
	
	
	Using data from a survey conducted in China in 2017, the Vatsa study\footnote{Vatsa, P., Ma, W., \& Zheng, H. (2024). Mobile payment adoption in China: Do demographic and socioeconomic factors matter?. Managerial and Decision Economics, 45(3), 1428-1434.} examined the relationship between various factors and the adoption of mobile payments. The results showed that factors such as education, perceived health status, economic status, political affiliation, employment status, income, social activity, car ownership, Internet access, and geographical location positively correlate with the use of mobile payment systems. However, older people and men were less likely to use these systems compared to their younger counterparts and women. Interestingly, the study found no correlation between subjective well-being and the use of mobile payments.
	
	
	Another approach to assessing the adoption of cashless payment technology is an article by Crouzet \footnote{Crouzet, N., Ghosh, P., Gupta, A., \& Mezzanotti, F. (2024). Demographics and technology diffusion: Evidence from mobile payments. Available at SSRN.
	} who explore the impact of the age structure of the population on the rate of technology diffusion, focusing on mobile payments in India. The authors use transaction data from 200,000 Indian bank customers and data on the introduction of QR terminals by a fintech company. Empirical analysis is combined with a theoretical model where the age of consumers determines preferences in choosing technologies. The authors model formalizes the relationship between demographics and business decisions. The demand for technology among young consumers is described:
	
	\begin{equation}		
		s_{Y}(p(j), a(j)) = \frac{a(j)}{J} (\frac{p(j)}{P_{Y}})^{-\frac{1}{v-1}},
	\end{equation}
	
	
	where $a(j)$ is the level of technology adoption by businesses $j$; $P_{Y}$ is a price index for the young population; $v$ ia an elasticity of substitution.
	
	
	The key conclusion of the article is that age is a significant factor: 38\% of the variation in the share of mobile payments is explained by the age of users. Young consumers (18-30 years old) use mobile payments twice as often as older groups (60+). This is consistent with Rogers theory of innovation diffusion, where young people act as <<early adopters>>.
	
	
	It is also important to identify the control variables that influence the development of national payment systems. For example, Carvalho's article analyzes the impact of demographic trends on real interest rates in countries with insufficient capital mobility. The authors have developed a multidimensional general equilibrium model that takes into account various demographic characteristics and constraints on international capital flows. The main conclusion of the study is that the aging of the population, a decrease in the birth rate and an increase in life expectancy account for about a third of the global decline in real interest rates between 1990 and 2019. Another important factor to consider is the Covid-19 pandemic. In this regard, the authors\footnote{Graziano, E. A., Musella, F., \& Petroccione, G. (2024). Cashless payment: behavior changes and gender dynamics during the COVID-19 pandemic. EuroMed Journal of Business, 20(5), 54-74.
	} examine the impact of the pandemic on the adoption of cashless payments in Italy, focusing on demographic factors, financial literacy, and gender differences. The study is based on data collected through an online survey conducted among 836 respondents between November 2021 and February 2022. Panel regression analysis was used to analyze the data. Fear of COVID-19 was found to have a direct positive impact on the use of non-cash payment methods. However, the level of financial literacy was found to be less influential. Gender analysis revealed no significant differences in the acceptance of non-cash payments between men and women. Additionally, it was noted that media coverage of the pandemic indirectly influenced attitudes towards cashless payments through the fear of infection.
	
	
	
	\begin{longtable}
		{|>{\centering\small\arraybackslash}p{6cm}
			|>{\centering\small\arraybackslash}p{9.3cm}|
		}
		\caption{Conclusions obtained as a result of a literary review of empirical works}\label{tab:conclusion}\\
		
		\hline
		\textbf{Paper} & \textbf{Brief conclusions}
		\\\hline
		\endfirsthead
		
		
		\multicolumn{2}{r}{Continuation of the table \ref{tab:conclusion}}\\\hline
		\endlasthead
		
		\multicolumn{2}{@{} l}{\small\centering{Source: compiled by the author based on the analysis of empirical works}}
		\endfoot
		
		\multicolumn{2}{|c|}{\small\textbf{Analysis of demographic factors at the respondent level}}\\\hline
		
		Stavins (2016); Greene, Perry, Stavins (2024) &  Random effects model, with a correction for sampling bias, and the two-stage Heckman model, show that young individuals aged 18-25 are more likely to choose online banking and debit cards. In contrast, older individuals aged over 65 tend to use checks\\\hline
		
		Camilleri, Agius (2021) & Age stratification: the economically active population uses traditional methods less often, but the 20-39 age group still uses them because of habit\\\hline
		
		
		\multicolumn{2}{|c|}{\small\textbf{Analysis of aggregated data by regions}}\\\hline
		
		 Arkhipova (2022) & An analysis of data collected in 78 regions of Russia shows that with an increase in the proportion of elderly people in society, the structure of savings is changing. In particular, urban residents are more actively using innovative payment solutions.\\\hline
		
		Tishin (2020) & Examination of effects population shifts on Russia and specific financial metrics, based on the demographic projections provided by Rosstat until 2036\\\hline
		
		
		\multicolumn{2}{|c|}{\small\textbf{Analysis of control variables}}\\\hline
		
		Graziano et al. (2024) & The impact of COVID-19 on cashless payments in Italy: Due to the fear of COVID-19, there has been an increase in the use of cashless payment methods, without any gender differences\\\hline
		
		\multicolumn{2}{|c|}{\small\textbf{Technology adoption analysis}}\\\hline
		
		Chawla, Joshi (2018) & Demographic factors influence the use of mobile banking in India: women like it more because it is more convenient. People over the age of 30 trust more\\\hline
		
		Crouzet et al. (2024) & Models of innovation dissemination proposed by Rogers, as well as the analysis of demand elasticity, led to the conclusion that young people are among the <<early adopters>>\\\hline
		
	\end{longtable}
	
	
	In conclusion, to track the progress of the national payment system, the Bank of Russia can rely on demographic data such as age and gender differences, as well as control variables, including the key interest rate. By analyzing changes in these indicators, the Bank of Russia is able to identify potential areas for improvement and take appropriate steps. The review of empirical studies provides a deeper understanding of the effectiveness of various indicators and methods, which are summarized in the table \ref{tab:conclusion}.
	
	
	
	\section{Construction of an index for the development of the national payment system in Russia}
	
	
	The level of development of national payment systems in Russia is often characterized by a high degree of regional heterogeneity, making it difficult to implement a unified strategy for regulating and supporting them. To address this issue, it is necessary to create a comprehensive index that can quantify the state of payment infrastructure across different regions, identify imbalances, and serve as a basis for Central Bank policy. In this chapter, we will develop a methodology to calculate an integral index for the development of the national payment system (NPSI) in Russia. This index will allow us to measure objectively the level of payment infrastructure development across regions and establish a connection between the development of payment systems and demographic factors.
	
	
	The construction of integral indexes is a common practice in economic and social research. Such indexes, which combine several different indicators into a single indicator, allow for a comprehensive and comparable assessment of the phenomenon under study. Among the most famous examples are the Human Development Index (HDI)\footnote{Ul Haq, M. (1995). Reflections on human development. oxford university Press.}, Digital Economy and Society Index (DESI) \footnote{Bánhidi, Z., Dobos, I., \& Nemeslaki, A. (2020). What the overall Digital Economy and Society Index reveals: A statistical analysis of the DESI EU28 dimensions. Regional Statistics, 10(2).}, the index, as well as Urban Health Index (UHI)\footnote{World Health Organization. (2014). The urban health index: A handbook for its calculation and use. In Kobe, Japan.} from the World Health Organization.
	
	
	Sympathize with the UHI methodologies used to build the indexes in this study, they are versatile and flexible. This methodology provides for the standardization of individual indicators with subsequent aggregation by means of a geometric mean, which minimizes the impact of extreme values and makes the index resistant to data variability. This approach makes it possible to effectively assess and compare the development of complex multifactorial phenomena, which include the national payment system.
	
	
	The UHI methodology was selected to evaluate the development of Russia's national payment system due to several factors. First, a payment system similar to urban infrastructure and healthcare has a wide range of components with different units of measurement. Second, this methodology takes into account spatial and regional differences in Russia's diverse socio-economic and demographic landscape. Third, standardization and averaging minimize potential errors caused by extreme regional variations.
	
	To calculate the integral index, the following sets of Bank of Russia indicators were selected for the period from 2013 to 2023 for all regions\footnote{Except of Crymia, Sevastopol, DPR, LPR, Zaporizhia Oblast and Kherson Oblast of the fact that there is lack of data for these regions} of Russia\footnote{https://www.cbr.ru/statistics/nps/psrf/}:
	
	\begin{enumerate}
		\item Institutional provision of payment services in the territorial context -- It shows the degree of development of the infrastructure of financial institutions and the availability of banking services in a particular region;
		\item Number of accounts opened by institutions of the banking system, by territory  -- it reflects the level of financial accessibility and involvement of the region's population in the banking sector;
		\item Number of payments made through credit institutions (by payment instruments) in the territorial context -- it allows to assess the intensity of the population's use of payment services and the prevalence of non-cash payments;
		\item Volume of payments made through credit institutions (for payment instruments) in the territorial context -- complements the previous indicator, characterizing not only the number of transactions, but also their financial significance;
		\item Number of electronic payment orders  -- it shows the level of digitalization of the payment behavior of the population and the degree of penetration of electronic payment channels;
		\item Volume of electronic payment orders  -- assessment of the degree of integration of electronic payment technologies into the daily economic life of the region.
	\end{enumerate}
	
	
	In general, the resulting dataset contains 40 different variables that are related to each other. Due to their large number and the fact that they clearly correlate, a primary correlation analysis was conducted using Spearman's rank correlation. The final set of indicators and their descriptive statistics are presented in Table \ref{tab:stats}. Their correlation matrix is shown in the figure \ref{fig:corr}.
	
	The next step in building an index is to standardize the data (calculated separately for each year). The standardization method is applied to convert the data into a uniform, comparable format:
	
	\begin{equation}
		S_i = \frac{I_i - \min(I_i)}{\max(I_i) - \min(I_i)},
	\end{equation}
	
	
	where \( S_i \) -- standardized indicator value;  \( I_i \) -- the initial value of the indicator for the ith region; \( \min(I_i)\), \( \max(I_i)\) are the minimum and maximum values of the corresponding indicator among all regions of Russia.
	
	
	\begin{longtable}
		{|>{\centering\scriptsize}p{10em}
			|>{\centering\scriptsize}p{14em}
			|>{\centering\scriptsize}p{3em}
			|>{\centering\scriptsize\arraybackslash}p{2.3em}|
		}
		
		\caption{Descriptive statistics of the data used to build the index of development of the national payment system of Russia}\label{tab:stats}
		\\
		
		\hline
		\textbf{Variable Name} & \textbf{Source} & \textbf{Mean} & \textbf{Standart Deviation}
		\\\hline
		\endfirsthead
		
		\multicolumn{4}{r}{Continuation of the table \ref{tab:conclusion}}\\\hline
		\endhead
		
		\multicolumn{4}{r}{End of the table \ref{tab:conclusion}}\\\hline
		\endlasthead
		
		\multicolumn{4}{@{} l}{\centering{Source: compiled by the author based on the data analyzed}}
		\endfoot
		
		Total number of banking system institutions & Institutional provision of payment services in the territorial context, Bank of Russia & 432.85 &  639.38\\\hline
		
		The number of banks per million people & Institutional provision of payment services in the territorial context, Bank of Russia & 235.45 & 85.07\\\hline
		
		The number of accounts opened by banking institutions per resident & Number of accounts opened by institutions of the banking system, by territory, Bank of Russia & 4.33 & 2.02\\\hline
		
		Total amount of payment requirements with internet & Number of payments made through credit institutions (by payment instruments) in the territorial context, Bank of Russia & 40.47 & 127.35\\\hline
		
		Volume of bank orders in billion rubbles & Volume of payments made through credit institutions (for payment instruments) in the territorial context, Bank of Russia & 29.71 & 173.83\\\hline
		
		Volume of all orders in electronic form with internet & Volume of electronic payment orders, Bank of Russia & 1808.66 & 12677.03\\\hline
		
		
	\end{longtable}
	
	
	\begin{figure}[H]
		\centering
		\noindent\includegraphics[width=0.5\linewidth]{corr.png}
		\caption{Correlation matrix\\Source: compiled by the author based on the data analyze}
		\label{fig:corr}
	\end{figure}
	
	
	The last step is to aggregate standardized indicators into an integral index. A geometric mean is used to combine standardized values:
	
	\begin{equation}
		Index = \left( \prod_{i=1}^{n} S_i \right)^{\frac{1}{n}}
	\end{equation}
	
	Using a geometric mean minimizes the impact of sharp regional disparities and provides a more balanced assessment. Logarithmization was also applied to balance outliers in the data before calculating the geometric mean.
	
	
	The index of development of the national payment system in Russia, obtained during the study, shows interesting results (Table \ref{tab:index_final}). The index takes values from 0 to 1. The closer the value is to 1, the higher the level of development of the payment system in the region.
	
	
	\begin{table}[H]
		\caption{The top 5 and bottom 5 regions of Russia according to the national payment system development index for October 2023}\label{tab:index_final}
		\centering
		\begin{tabular}%{\linewidth}
			{|c|c|c|c|}\hline
			\textbf{Regions-<<leaders>>} & \textbf{Index Value} & \textbf{Regions-<<losers>>}& \textbf{Index Value}\\\hline
			
			Moscow & 0.94 &  Republic of Ingushetia & 0.09\\\hline
			
			St. Petersburg & 0.69 &  Republic of Dagestan & 0.20\\\hline
			
			Novosibirsk Oblast & 0.63 &  Chechen Republic & 0.21\\\hline
			
			Sverldlovsk Oblast & 0.62 & Republic of Kalmykia & 0.26\\\hline
			
			Voronezh Oblast & 0.61 & Jewish Autonomous Oblast & 0.28\\\hline
			
		\end{tabular}
		Source: compiled by the author based on data from the Bank of Russia
	\end{table}
	
	
	It is worth noting that Moscow and St. Petersburg occupy leading positions in this index, which is understandable, since they are the financial centers of the country with the largest population. In addition, technologically advanced and urbanized regions such as Novosibirsk and Sverdlovsk regions are on the list of leaders. The underdeveloped southern regions complete the ranking according to this index. For a full-fledged comparison of the index, competence diagrams were constructed for 3 types of regions: top, middle, and losers (Figure \ref{fig:compare}). Based on the diagrams, only one conclusion can be drawn -- in Russia there is a high differentiation between the top regions (Moscow, St. Petersburg) and outsiders (Dagestan, Chechnya).
	
	
	\begin{figure}[H]
		\centering
		\includegraphics[width=0.3\textwidth]{losers.png}
		\hfill % Заполнитель пространства между изображениями
		\includegraphics[width=0.3\textwidth]{top.png}
		\hfill
		\includegraphics[width=0.3\textwidth]{middle.png}

		\caption{Competence diagrams\\Source: compiled by the author based on the data analisys}
		\label{fig:compare}
	\end{figure}


	It is also possible to compile a generalized index that will reflect the situation in Russia as a whole. However, given the significant differences between regions, the distribution of the index will have a wide range of values, which will undoubtedly affect the integral indicator. Therefore, instead of the regional average, it was decided to use the median. The graph of this indicator(National Payment System Development Index -- NPSD Index), was built from 2013 to 2023 (Figure \ref{fig:index_whole}). As can be seen from the graph, there was a decrease in the index in 2014-2015, which can be explained by sanctions against Russia. It was during this period that the National Card Payment System and the MIR payment system were created. Until 2020, the index showed significant growth due to the development of these systems and the introduction of new technologies and processes into the Russian national payment system, for example, the fast payment system in 2019. However, growth has slowed during the pandemic. Since 2022, the index has begun to decline, which is associated with the general economic crisis in Russia. During this period, the key interest rate was high, and the ruble was unstable, which aggravated the situation. The dynamics of the index was also affected by internal shocks, which also had a negative impact.
	
	\begin{figure}[H]
		\centering
		\noindent\includegraphics[width=0.5\linewidth]{index_whole.png}
		\caption{Aggregated National Payment System Development Index for Russia\\Source: compiled by the author based on the data analyze}
		\label{fig:index_whole}
	\end{figure}
	
	
	Summing up, the following conclusions are drawn:
	
	\begin{enumerate}
		\item It is feasible to develop an index that would objectively reflect the development of the Russian National Payment System both at the regional level and nationwide, based on established methodologies for index development (UHI) -- which is related to the similarity of tasks: the need to aggregate heterogeneous indicators, taking into account regional specifics;
		\item The results of the calculations revealed significant differences between the regions. Moscow (0.94) and St. Petersburg (0.69) were the leaders, which is explained by their status as financial centers with a high concentration of infrastructure and population. The Republic of Ingushetia (0.09) and Dagestan (0.20), on the other hand, are among the regions with the lowest levels of development of payment systems;
		\item The calculated index demonstrates not only the current state of the payment system, but also its vulnerability to macroeconomic factors. Its application allows the Bank of Russia to identify problem regions, evaluate the effectiveness of regulatory measures and develop targeted strategies to reduce imbalances.
	\end{enumerate}
	
	
	\section{Developing an econometric model to estimate the impact of demographic variables on the share of cashless transactions}

	
	%%%%%%%%%%% References	%%%%%%%%%%%


	\newpage
	\addcontentsline{toc}{chapter}{References}
	\titleformat*{\section}{\bfseries\normalsize\fontsize{14}{2.5mm}\centering}
	\begin{thebibliography}{3}

		\bibitem{cental_bank_russia_natiional_statistics} Bank of Russia. National Payment System statistics (2025). Date of the application 01.04.2025\\
		https://www.cbr.ru/PSystem/

		\bibitem{cental_bank_russia_natiional_payment_dir} Центральный Банк Российской Федерации. Основные направления развития национальной платежной системы на период 2025-2027 годов. // https://www.cbr.ru/Content/Document/File/170680/onrnps\_2025-27.pdf

		\bibitem{cental_bank_russia_natiional_fin_serv} Центральный Банк Российской Федерации. Основные направления повышения доступности финансовых услуг в Российской Федерации на период 2025-2027 годов. // https://www.cbr.ru/Content/Document/File/170684/onpdfu\_2025-2027.pdf

		\bibitem{labour_min} Распоряжение Правительства РФ об утверждении Стратегии действий в интересах граждан старшего поколения в Российской Федерации до 2030 года //
		https://mintrud.gov.ru/ministry/programms/12

		\bibitem{modigliani_brumberg} Modigliani, F., \& Brumberg, R. (1954). Utility analysis and the consumption function: An interpretation of cross-section data. Franco Modigliani, 1(1), 388-436.

		\bibitem{modigliani} Ando, A., \& Modigliani, F. (1963). The “Life Cycle” Hypothesis of Saving: Aggregate Implications and Tests. The American Economic Review, 53(1), 55–84. http://www.jstor.org/stable/1817129

		\bibitem{modigliani_1966} Modigliani, F. (1966). The life cycle hypothesis of saving, the demand for wealth and the supply of capital. Social research, 160-217.
		
		\bibitem{modigliani_1970} Modigliani, F. (1970). The life cycle hypothesis of saving and intercountry differences in the saving ratio (pp. 197-225). WA Eltis, M. FG. Scott and JN Wolfe, eds., Induction, trade, and growth: Essays in honour of Sir Roy Harrod (Clarendon Press, Oxford).

		\bibitem{Kotlikoff} Kotlikoff, L. J. (1989). What determines savings?. MIT Press Books, 1.	
		
		\bibitem{taleb} Taleb, N.N. (2007) The Black Swan: The Impact of the Highly Improbable. Random House, New York
		
		\bibitem{tversky} Tversky, A., \& Kahneman, D. (1992). Advances in prospect theory: Cumulative representation of uncertainty. Journal of Risk and uncertainty, 5, 297-323.
		
		\bibitem{rogers} Rogers, E. M., Singhal, A., \& Quinlan, M. M. (2014). Diffusion of innovations. In An integrated approach to communication theory and research (pp. 432-448). Routledge.
		
		\bibitem{Venkatesh} Venkatesh, V., Morris, M. G., Davis, G. B., \& Davis, F. D. (2003). User acceptance of information technology: Toward a unified view. MIS quarterly, 425-478.
		
		\bibitem{Oldenburg} Oldenburg, Brian, and Karen Glanz. "Diffusion of innovations." Health behavior and health education: Theory, research, and practice 4 (2008): 313-333.
		
		\bibitem{becker} Becker, G. S. (1962). Investment in human capital: A theoretical analysis. Journal of political economy, 70(5, Part 2), 9-49.
		
		\bibitem{galor} Galor, O., \& Weil, D. N. (2000). Population, technology, and growth: From Malthusian stagnation to the demographic transition and beyond. American economic review, 90(4), 806-828.
		
		\bibitem{ey} EY Report: How Gen Z’s preference for digital is changing the payments landscape (2024) // https://www.ey.com/en\_us/insights/payments/how-gen-z-is-changing-the-payments-landscape
		
		\bibitem{simon} Simon, H. A. (1955). A behavioral model of rational choice. The quarterly journal of economics, 99-118.
		
		\bibitem{taler} Thaler, R. H. (2015). Misbehaving: The making of behavioral economics. WW Norton \& Company.
		
		\bibitem{nudge1} Aydogan, S., \& Van Hove, L. (2015). Nudging consumers towards card payments: A field experiment. In International Cash Conference 2014 (pp. 589-630). Deutsche Bundesbank
		
		\bibitem{nudge2} Story, P., Smullen, D., Acquisti, A., Cranor, L. F., Sadeh, N., \& Schaub, F. (2020). From intent to action: Nudging users towards secure mobile payments. In Sixteenth Symposium on Usable Privacy and Security (SOUPS 2020) (pp. 379-415.
		
		\bibitem{nudge_tax} Calvo-Gonzalez, O., Cruz, A., \& Hernandez, M. (2018). The Ongoing Impact of ‘Nudging’People to Pay Their Taxes. World Bank Blogs, 2
		
		\bibitem{Greene} Greene, C., Perry, J., \& Stavins, J. (2024). Consumer Payment Behavior by Income and Demographics.
		
		\bibitem{coase} Coase, R. H. (1937). ”The Nature of the Firm”. Economica. 4 (16): 386–405
		
		\bibitem{Williamson} Williamson, O. E. (1979). Transaction-cost economics: the governance of contractual relations. The journal of Law and Economics, 22(2), 233-261.
		
		\bibitem{bloom} Bloom, D. E., \& Williamson, J. G. (1998). Demographic transitions and economic miracles in emerging Asia. The World Bank Economic Review, 12(3), 419-455.
		
		\bibitem {kyc} Mullins, R. R., Ahearne, M., Lam, S. K., Hall, Z. R., \& Boichuk, J. P. (2014). Know your customer: How salesperson perceptions of customer relationship quality form and influence account profitability. Journal of Marketing, 78(6), 38-58.
		
		\bibitem{Vaportzis} Vaportzis, E., Giatsi Clausen, M., \& Gow, A. J. (2017). Older adults perceptions of technology and barriers to interacting with tablet computers: a focus group study. Frontiers in psychology, 8, 1687.
		
		\bibitem{Finkelstein} Finkelstein, M., \& Fishman, R. (2019). Demographics and Innovation: Evidence from Patent Data
		
		\bibitem{stavins} Stavins, J. (2016). The effect of demographics on payment behavior: panel data with sample selection (No. 16-5). Working Papers.
		
		\bibitem{wooldridge} Wooldridge, J. M. 1995. “Selection Corrections for Panel Data Models Under Conditional Mean
		Independence Assumptions.” Journal of Econometrics 68 (1): 115–132.
		
		\bibitem{stavins2} Greene, C., Perry, J., \& Stavins, J. (2024). Consumer Payment Behavior by Income and Demographics.
		
		\bibitem{malta} Camilleri, S. J., \& Agius, C. (2021). Choosing between innovative and traditional payment systems: an empirical analysis of European trends. Journal of Innovation Management, 9(4), 29-57.
		
		\bibitem{tishin} Тишин, А. (2020). Влияние демографии на развитие финансового сектора Российской Федерации. Аналитическая записка Департамента исследований и прогнозирования, Банк России.
		
		\bibitem{chawla} Chawla, D., \& Joshi, H. (2018). The moderating effect of demographic variables on mobile banking adoption: An empirical investigation. Global Business Review, 19(3\_suppl), S90-S113.
		
		\bibitem{vatsa} Vatsa, P., Ma, W., \& Zheng, H. (2024). Mobile payment adoption in China: Do demographic and socioeconomic factors matter?. Managerial and Decision Economics, 45(3), 1428-1434.
		
		
		\bibitem{graziano} Graziano, E. A., Musella, F., \& Petroccione, G. (2024). Cashless payment: behavior changes and gender dynamics during the COVID-19 pandemic. EuroMed Journal of Business, 20(5), 54-74.
		
		
		\bibitem{arkhipova} Arkhipova, N. (2022). Impact of Demographic Trends on Retail Banking. Procedia Computer Science, 214, 831-836.
		
		
		
	\end{thebibliography}
	
\end{document}
	
	